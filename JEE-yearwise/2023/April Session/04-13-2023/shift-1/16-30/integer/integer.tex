\iffalse
\let\negmedspace\undefined
\let\negthickspace\undefined
\documentclass[journal]{IEEEtran}
\usepackage[a5paper, margin=10mm, onecolumn]{geometry}
%\usepackage{lmodern} % Ensure lmodern is loaded for pdflatex
\usepackage{tfrupee} % Include tfrupee package

\setlength{\headheight}{1cm} % Set the height of the header box
\setlength{\headsep}{0mm}     % Set the distance between the header box and the top of the text

\usepackage{gvv-book}
\usepackage{gvv}
\usepackage{cite}
\usepackage{amsmath,amssymb,amsfonts,amsthm}
\usepackage{algorithmic}
\usepackage{graphicx}
\usepackage{textcomp}
\usepackage{xcolor}
\usepackage{txfonts}
\usepackage{listings}
\usepackage{enumitem}
\usepackage{mathtools}
\usepackage{gensymb}
\usepackage{comment}
\usepackage[breaklinks=true]{hyperref}
\usepackage{tkz-euclide} 
\usepackage{listings}
% \usepackage{gvv}                                        
\def\inputGnumericTable{}                                 
\usepackage[latin1]{inputenc}                                
\usepackage{color}                                            
\usepackage{array}                                            
\usepackage{longtable}                                       
\usepackage{calc}                                             
\usepackage{multirow}                                         
\usepackage{hhline}                                           
\usepackage{ifthen}                                           
\usepackage{lscape}
\bibliographystyle{IEEEtran}
\vspace{3cm}

\title{2023}
\author{EE24BTECH11061}
\maketitle

\renewcommand{\thefigure}{\theenumi}
\renewcommand{\thetable}{\theenumi}
\section{integer}
\fi

%\begin{enumerate}
\item The sum to 20 terms of the series $2.2^2 - 3^2 + 2.4^2 - 5^2 + 2.6^2-\dots$ is equal to
\hfill{\sbrak{\text{April 2023}}}

\item Let the mean of the data
\begin{center}
    \begin{tabular}{|c|c|c|c|c|c|}
\hline
	x & 1  & 3  & 5  & 7  & 9  \\ 
\hline
	Frequency (f) & 4  & 24 & 28 & $\alpha$ & 8  \\ 
\hline
\end{tabular}
\end{center}
be 5. If $m$ and $\sigma^2$ are respectively the mean deviation about the mean and the variance of the data, then $\frac{3\alpha}{m+\sigma^2}$ is equal to
\hfill{\sbrak{\text{April 2023}}}

\item Let $\alpha$ be the constant term in the binomial expansion of $\brak{\sqrt{x} - \frac{6}{x^{\frac{3}{2}}}}^n$, $n \leq 15$. If the sum of the coefficients of the remaining terms in the expansion is 649 and the coefficient of $x^{-n}$ is $\lambda\alpha$, then $\lambda$ is equal to
\hfill{\sbrak{\text{April 2023}}}

\item Let $\omega = z\overline{z} + k_1 z + k_2 \iota z + \lambda\brak{1+\iota}$, $k_1, k_2 \in \mathbb{R}$. Let $Re\brak{\omega} = 0$ be the circle of radius 1 in the first quadrant touching the line $y=1$ and the y-axis. If the curve $Im\brak{\omega} = 0$ intersects $C$ at $A$ and $B$, then $30\brak{AB}^2$ is equal to
\hfill{\sbrak{\text{April 2023}}}

\item Let $\vec{a} = 3i + j -k$ and $\vec{c} = 2i-3j+3k$. If $\vec{b}$ is a vector such that $\vec{a} = \vec{b} \times \vec{c}$ and $\abs{\vec{b}}^2 = 50$, then $\abs{72-\abs{\vec{b} + \vec{c}}^2}$ is equal to
\hfill{\sbrak{\text{April 2023}}}

\item Let $m_1$ and $m_2$ be the slopes of the tangents drawn from the point $\vec{P}(4, 1)$ to the hyperbola $H: \frac{y^2}{25} - \frac{x^2}{16} = 1$. If $\vec{Q}$ is the point from which the tangents drawn to $H$ have slopes $m_1$ and $m_2$, and they make positive intercepts $\alpha$ and $\beta$ on the x-axis, then $\frac{\brak{\vec{P}\vec{Q}}^2}{\alpha \beta}$ is equal to
\hfill{\sbrak{\text{April 2023}}}

\item Let the image of the point $\brak{\frac{5}{3}, \frac{5}{3}, \frac{8}{3}}$ in the plane $x - 2y + z - 2 = 0$ be $\vec{P}$. If the distance of the point $\vec{Q}\brak{6, -2, \alpha}$, $\alpha > 0$, from $\vec{P}$ is 13, then $\alpha$ is equal to
\hfill{\sbrak{\text{April 2023}}}

\item Let for $x \in \mathbb{R}$, $S_0\brak{x} = x$, $S_k\brak{x} = C_kx + k\int_0^x S_{k-1}\brak{t}, dt$ where $C_0 = 1$, $C_k = 1-\int_0^1 S_{k-1}\brak{x}, dx$, $k = 1, 2, 3, \dots$. Then $S_2\brak{3} + 6C_3$ is equal to
\hfill{\sbrak{\text{April 2023}}}

\item If S = $\{x \in \mathbb{R} \colon \sin^{-1}\brak{\frac{x+1}{\sqrt{x^2+2x+2}}} - \sin^{-1}\brak{\frac{x}{\sqrt{x^2+1}}} = \frac{\pi}{4}\}$, then $\sum_{x \in \S} \brak{\sin{\brak{\brak{x^2+x+5}\frac{\pi}{2}}}} - \brak{\cos{\brak{\brak{x^2+x+5}}\pi}}$ is equal to
\hfill{\sbrak{\text{April 2023}}}

\item The number of seven digit positive integers formed using the digits 1, 2, 3 and 4 only and sum of the digits equal to 12 is
\hfill{\sbrak{\text{April 2023}}}
%\end{enumerate}