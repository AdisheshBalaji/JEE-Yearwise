\iffalse
\title{2023}
\author{EE24BTECH11033}
\section{mcq-single}
\fi
%\begin{enumerate}

\item The number of ways, in which 5 girls and 7 boys can be seated at a round table so that no two girls sit together, is
\hfill(Apr-2023)
\begin{multicols}{4}
    \begin{enumerate}
        \item $7\brak{720}^2$
        \item $720$
        \item $7\brak{360}^2$
        \item $126\brak{5!}^2$
    \end{enumerate}
\end{multicols}	

\item Let $f(x) = \frac{\sin x + \cos x - \sqrt{2}}{\sin x - \cos x}$, $x \in \sbrak{0, \pi}-\cbrak{\frac{\pi}{4}}$. Then $f\brak{\frac{7\pi}{12}}f^{\prime\prime}\brak{\frac{7\pi}{12}}$  is equal to
\hfill(Apr-2023)
\begin{multicols}{4}
    \begin{enumerate}
    \item $\frac{-2}{3}$
    \item $\frac{2}{9}$
    \item $\frac{-1}{3\sqrt3}$
    \item $\frac{2}{3\sqrt{3}}$
    \end{enumerate}
\end{multicols}

\item If the equation of the plane containing the line $x$ + $2y$ + $3z$ - 4 =0, $2x$ + $y$ - $z$ + 5 =0 and perpendicular to the plane $\vec{r}$= $\brak{\hat{i}-\hat{j}}$ + $\lambda\brak{\hat{i}+\vec{j}+\vec{k}}$ +$\mu\brak{\hat{i}-2\hat{j}+3\hat{k}}$ is $ax + by+cz=4$, then $(a-b+c)$ is equal to
\hfill(Apr-2023)
\begin{multicols}{4}
    \begin{enumerate}
    \item 22
    \item 24
    \item 20
    \item 18
    \end{enumerate}
\end{multicols}

\item Let A =  \myvec{2 & 1 & 0 \\ 1 & 2 & -1 \\ 0 & -1 & 2 } . If  $\abs{\operatorname{adj}\abs{\operatorname{adj}\abs{\operatorname{adj}2A}}}$= $\brak{16}^n$, then $n$ is equal to
\hfill(Apr-2023)
\begin{multicols}{4}
    \begin{enumerate}
    \item 8
    \item 9
    \item 12
    \item 10
    \end{enumerate}
\end{multicols}

\item Let \( I(x) = \int \frac{\brak{x+1}}{x\brak{1 + x e^x}^2} \, dx, \, x > 0. \) If \( \lim_{x \to \infty} I\brak{x} = 0, \) then \( I\brak{1} \) is equal to
\hfill(Apr-2023)
\begin{multicols}{2}
    \begin{enumerate}
    \item  $\frac{e+1}{e+2} - \log_e\brak{e+1} $
    \item  $\frac{e+2}{e+1} + \log_e\brak{e+1}$ 
    \item  $\frac{e+2}{e+1} - \log_e\brak{e+1} $
    \item  $\frac{e+1}{e+2} + \log_e\brak{e+1} $
\end{enumerate}
\end{multicols}
%\end{enumerate}
