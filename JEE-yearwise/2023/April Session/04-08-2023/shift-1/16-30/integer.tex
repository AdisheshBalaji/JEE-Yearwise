\iffalse
\title{2023}
\author{EE24BTECH11033}
\section{integer}
\fi
%\begin{enumerate}
\item   Let $A = \sbrak{0, 3, 4, 6, 7, 8, 9, 10}$ and $R$ be the relation defined on $A$ such that $R = \cbrak{(x, y) \in A \times A : x - y \text{ is an odd positive integer or } x - y = 2 }$. The minimum number of elements that must be added to the relation $R$ so that it is a symmetric relation is equal to 
\hfill(Apr-2023)
\item Let $\sbrak{t}$ denote the greatest integer $\leq t$. If the constant term in the expansion of $\brak{3x^2 - \frac{1}{2x^5}}^7$ is $\alpha$, then $\sbrak{\alpha}$  is equal to 
\hfill(Apr-2023)
\item Let $\lambda_1, \lambda_2$ be the values of $\lambda$ for which the points 
$\brak{\frac{5}{2}, -1, \lambda}$ and $\brak{-2, 0, 1}$ 
are at equal distance from the plane $2x + 3y - 6z + 7 = 0$. 
If $\lambda_1 > \lambda_2$, then the distance of the point $(\lambda_1 - \lambda_2, \lambda_2, \lambda_1)$ 
from the line $\frac{x - 5}{1} = \frac{y - 1}{2} = \frac{z + 7}{2}$
\hfill(Apr-2023)
\item If the solution curve of the differential equation $\brak{y - 2 \log_e x}dx + \brak{x \log_e x^2}dy = 0$, $x > 1$ passes through the points $\brak{e, \frac{4}{3}}$ and  $\brak{e^4, \alpha}$, then $\alpha$ is equal to 
\hfill(Apr-2023)
\item Let \(\vec{a} = 6\hat{i} + 9\hat{j} + 12\hat{k}, \ \vec{b} = \alpha\hat{i} + 11\hat{j} - 2\hat{k}\) and \(\hat{c}\) be vectors such that \(\vec{a} \times \vec{c} = -\vec{a} \times \vec{b}\). If \(\vec{a} \cdot \vec{c} = -12, \ \vec{c} \cdot (\hat{i} - 2\hat{j} + \hat{k}) = 5\), then \(\vec{c} \cdot (\hat{i} + \hat{j} + \hat{k})\) is equal to 
\hfill(Apr-2023)
\item The largest natural number $n$ such that $3^n$ divides 66! is
\hfill(Apr-2023)
\item If $a_a$ is the greatest term in the sequence $a_n=\frac{n^3}{n^4 + 147}$, $n$=1, 2, 3,..... then $a$ is equal to \hfill(Apr-2023)
\item Let the mean and variance of 8 numbers $x$, $y$, 10, 12, 6, 12, 4, 8 be 9 and 9.25 respectively. If $x>y$, then $3x-2y$ is equal to 
\hfill(Apr-2023)
\item Consider a circle $C_1 : x^2+y^2-4x-2y=\alpha-5$. Let its mirror  image in the line $y=2x+1$ be another circle $C_2 : 5x^2+5y^2 -10fx-10gy+36=0 $.Let $r$ be the radius of $C_2$. Then $\alpha$ + $r$ is equal to
\hfill(Apr-2023)
\item Let $\sbrak{t}$ denote the greatest integer $\leq t$. Then $\frac{2}{\pi} \int_{\pi/6}^{5\pi/6} \brak{8\sbrak{\cosec x} - 5\sbrak{\cot x} } dx$ is equal to 
\hfill(Apr-2023)
%\end{enumerate}

