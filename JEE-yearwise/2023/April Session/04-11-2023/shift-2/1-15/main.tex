\iffalse
\title{Assignment}
\author{K.AKSHAY TEJA}
\section{mcq-single}
\fi
% Question 1
\item The angle of elevation of the top $P$ of a tower from the feet of one person standing due South of the tower is $45^\circ$ and from the feet of another person standing due West of the tower is $30^\circ$. If the height of the tower is $5$ meters, then the distance \brak{in \, meters} between the two persons is equal to \hfill \brak{Apr 2023}

\begin{multicols}{4}
\begin{enumerate}
    \item $10$
    \item $5 \sqrt{5}$
    \item $\frac{5}{2}\sqrt{5}$
    \item $5$
\end{enumerate}
\end{multicols}

% Question 2
\item Let $a, b, c$ and $d$ be positive real numbers such that $a + b + c + d = 11$. If the maximum value of $a^5 b^3 c^2 d$ is 3750$\beta$, then the value of $\beta$ is \hfill \brak{Apr 2023}
\begin{multicols}{4}
\begin{enumerate}
    \item $55$
    \item $108$
    \item $90$
    \item $110$
\end{enumerate}
\end{multicols}



% Question 3
\item Let $f: R \to R$ be a continuous function satisfying $\int_0^{\frac{\pi}{2}} f\brak{\sin 2x} \sin x \, dx + \alpha\int_0^{\frac{\pi}{4}} f\brak{\cos 2x} \cos x \, dx = 0,$ then the value of $\alpha$ is \hfill \brak{Apr 2023}
\begin{multicols}{4}
\begin{enumerate}
    \item $-\sqrt{3}$ 
    \item $\sqrt{3}$ 
    \item $-\sqrt{2}$ 
    \item $\sqrt{2}$ 
\end{enumerate}
\end{multicols}



% Question 4
\item Let $f$ and $g$ be two functions defined by $f\brak{x} = \begin{cases} 
x + 1, & x < 0 \\ 
\abs{x - 1}, & x \geq 0 
\end{cases}$ and $g\brak{x} = \begin{cases} 
x + 1, & x < 0 \\ 
1, & x \geq 0 
\end{cases}$ Then $\brak{gof}\brak{x}$ is \hfill \brak{Apr 2023}
\begin{enumerate}
    \item continuous everywhere but not differentiable at $x = 1$
    \item continuous everywhere but not differentiable exactly at one point
    \item differentiable everywhere
    \item not continuous at $x = -1$
\end{enumerate}


% Question 5
\item If the radius of the largest circle with center $\brak{2, 0}$ inscribed in the ellipse $x^2 + 4y^2 = 36$ is $r$, then $12r^2$ is equal to \hfill \brak{Apr 2023}
\begin{multicols}{4}
\begin{enumerate}
    \item $69$
    \item $72$
    \item $115$
    \item $92$
\end{enumerate}
\end{multicols}


% Question 6
\item Let the mean of 6 observations $1, 2, 4, 5, x,$ and $y$ be $5$ and their variance be $10$. Then their mean deviation about the mean is equal to \hfill \brak{Apr 2023}
\begin{multicols}{4}
\begin{enumerate}
    \item $\frac{7}{3}$
    \item $\frac{10}{3}$
    \item $\frac{8}{3}$
    \item $3$
\end{enumerate}
\end{multicols}


% Question 7
\item Let $A = \{1, 3, 4, 6, 9\}$ and $B = \{2, 4, 5, 8, 10\}$. Let $R$ be a relation defined on $A \times B$ such that $R = \{\brak{\brak{a_1, b_1}, \brak{a_2, b_2}}: a_1 \leq b_2$ and $b_1 \leq a_2\}.$ Then the number of elements in the set $R$ is \hfill \brak{Apr 2023}
\begin{multicols}{4}
\begin{enumerate}
    \item 52 
    \item 160 
    \item 26 
    \item 180 
\end{enumerate}
\end{multicols}



% Question 8
\item Let $P$ be the plane passing through the points $\brak{5, 3, 0}$, $\brak{13, 3, -2}$, and $\brak{1, 6, 2}$. For $\alpha \in \mathbb{N}$, if the distances of the points $A\brak{3, 4, \alpha}$ and $B\brak{2, \alpha, a}$ from the plane $P$ are $2$ and $3$ respectively, then the positive value of $a$ is \hfill \brak{Apr 2023}
\begin{multicols}{4}
\begin{enumerate}
    \item $5$
    \item $6$
    \item $4$
    \item $3$
\end{enumerate}
\end{multicols}


% Question 9
\item If the letters of the word MATHS are permuted and all possible words so formed are arranged as in a dictionary with serial number, then the serial number of the word THAMS is  \hfill \brak{Apr 2023}
\begin{multicols}{4}
\begin{enumerate}
    \item 102 
    \item 103 
    \item 101 
    \item 104 
\end{enumerate}
\end{multicols}

% Question 10
\item f four distinct points with position vectors $\overrightarrow{a},\, \overrightarrow{b},\, \overrightarrow{c} $ and $\overrightarrow{d}$ are coplanar, then $\sbrak{\overrightarrow{a} \, \overrightarrow{b} \,\overrightarrow{c}}$ is equal to \hfill \brak{Apr 2023}
\begin{multicols}{2}
\begin{enumerate}
    \item  $\sbrak{\overrightarrow{d} \, \overrightarrow{c} \, \overrightarrow{a}} + \sbrak{\overrightarrow{b} \, \overrightarrow{d} \, \overrightarrow{a}} + \sbrak{\overrightarrow{c} \, \overrightarrow{d} \, \overrightarrow{b}}$ 
    \item $\sbrak{\overrightarrow{d} \, \overrightarrow{b} \, \overrightarrow{a}} + \sbrak{\overrightarrow{a} \, \overrightarrow{c} \, \overrightarrow{d}} + \sbrak{\overrightarrow{d} \, \overrightarrow{b} \, \overrightarrow{c}}$
    \item $\sbrak{\overrightarrow{a} \, \overrightarrow{d} \, \overrightarrow{b}} + \sbrak{\overrightarrow{d} \, \overrightarrow{c} \, \overrightarrow{a}} + \sbrak{\overrightarrow{d} \, \overrightarrow{b} \, \overrightarrow{c}}$
    \item $\sbrak{\overrightarrow{b} \, \overrightarrow{c} \, \overrightarrow{d}} + \sbrak{\overrightarrow{d} \, \overrightarrow{a} \, \overrightarrow{c}} + \sbrak{\overrightarrow{d} \, \overrightarrow{b} \, \overrightarrow{a}}$
\end{enumerate}
\end{multicols}


% Question 11
\item The sum of the coefficients of three consecutive terms in the binomial expansion of $(1 + x)^{n+2}$, which are in the ratio $1 : 3 : 5$, is equal to \hfill \brak{Apr 2023}
\begin{multicols}{4}
\begin{enumerate}
    \item $63$
    \item $92$
    \item $25$
    \item $41$
\end{enumerate}
\end{multicols}

% Question 12
\item Let $y = y(x)$ be the solution of the differential equation $\frac{dy}{dx} + \frac{5}{x\brak{x^5 + 1}}y = \frac{\brak{x^5 +1}^2}{x^2}, x > 0.$ If $y(1) = 2$, then $y(2)$ is equal to  \hfill \brak{Apr 2023}
\begin{multicols}{4}
\begin{enumerate}
    \item $\frac{693}{128}$
    \item $\frac{637}{128}$
    \item $\frac{697}{128}$
    \item $\frac{679}{128}$
\end{enumerate}
\end{multicols}



% Question 13
\item The converse of $\sim(p \land q) \implies r$ is  \hfill \brak{Apr 2023}
\begin{multicols}{2}
\begin{enumerate}
    \item $\brak{p \lor \brak{\sim q}} \implies \brak{ \sim r}$ 
    \item $\brak{\brak{\sim p} \lor q} \implies  r$
    \item $\brak{\sim r} \implies \brak{ \brak{\sim p} \land q}$ 
    \item $\brak{\sim r} \implies  p \land q$
\end{enumerate}
\end{multicols}



% Question 14
\item If the $1011$th term from the end in the binomial expansion of $\brak{\frac{4x}{5} - \frac{5}{2x} }^{2022}$ is $1024$ times the $1011$th term from the beginning, then $\abs{x}$ is equal to  \hfill \brak{Apr 2023}
\begin{multicols}{4}
\begin{enumerate}
    \item $8$ 
    \item $12$ 
    \item $10$ 
    \item $15$ 
\end{enumerate}
\end{multicols}


% Question 15
\item If the system of linear equations 

$7x + 11y + \alpha z = 13\\5x + 4y + 7z = \beta \\175x + 194y + 57z = 361$ has infinitely many solutions, then $\alpha + \beta + 2$ is equal to: \hfill \brak{Apr 2023}
\begin{multicols}{4}
\begin{enumerate}
    \item $3$ 
    \item $6$ 
    \item $5$ 
    \item $4$ 
\end{enumerate}
\end{multicols}

%\end{enumerate}
%\end{document}

