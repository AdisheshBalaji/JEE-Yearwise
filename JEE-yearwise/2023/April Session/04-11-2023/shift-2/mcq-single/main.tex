\iffalse
\title{Assignment-5}
\author{EE24BTECH11048-NITHIN.K}
\section{mcq-single}
\fi
%\begin{enumerate}
%16
\item Let the line passing through the point P$\brak{2,-1,2}$ and Q$\brak{5,3,4}$ meet the plane $x-y+z=4$ at the point T. Then the distance of the point R from the plane $x+2y+3z+2=0$ measured parallel
to the line $\frac{x-7}{2} = \frac{y+3}{2} = \frac{z-2}{1}$ is equal to
\begin{enumerate}
\item 3
\item $\sqrt{61}$
\item $\sqrt{31}$
\item $\sqrt{189}$
\end{enumerate}

%17
\item Let the function f : $\sbrak{0,2} \rightarrow$ R be defined as \\
	$f\brak{x} = \begin{cases} e^{min\brak{x^2,x-\sbrak{x}}} & \text{, } x \in \lsbrak{0},\rbrak{1} \\ e^{\sbrak{x-\log_e{x}}} & \text{,} x \in \lsbrak{1},\rbrak{2} \end{cases}$ \\
		where $\sbrak{t}$ denotes the greatest integer less than or equal to t. Then the value of the integral $\int_{0}^{2}xf\brak{x}dx$ is
\begin{enumerate}
\item $\brak{e-1}\brak{e^2+\frac{1}{2}}$
\item $1+\frac{3e}{2}$
\item $2e-\frac{1}{2}$
\item 2e - 1
\end{enumerate}

%18
\item For a$\in$C, let A = $\cbrak{z\in C:Re\brak{a+\vec{z}} > Im\brak{\vec{a}+z}}$ and B = $\cbrak{z\in C:Re\brak{a+\vec{z}} < Im\brak{\vec{a}+z}}$. Then among the two statements: \\
$\brak{S_1} : If Re\brak{a}, Im\brak{a} > 0$, then the set A contains all the real numbers \\
$\brak{S_2} : If Re\brak{a}, Im\brak{a} < 0$, then the set B contains all the real numbers
\begin{enumerate}
\item only $S_1$ is true
\item both are false
\item only $S_2$ is true
\item both are true
\end{enumerate}

%19
\item If $\mydet{x+1 & x & x \\
	x & x+\lambda & x \\
	x & x & x+\lambda^2} = \frac{9}{8}\brak{103x+81}$, then $\lambda,\frac{\lambda}{3}$ are the roots of the equation
\begin{enumerate}
\item $4x^2-24x-27=0$         
\item $4x^2+24x+27=0$
\item $4x^2-24x+27=0$
\item $4x^2+24x-27=0$
\end{enumerate}

%20
\item The domain of the function $f\brak{x}=\frac{1}{\sqrt{\sbrak{x}^2-3\sbrak{x}-10}}$ is $\brak{\text{where} \sbrak{x} \text{denotes the greatest integer less than or equal to x}}$
\begin{enumerate}
\item $\lbrak{-\infty},\rsbrak{-3} \cup \lsbrak{6},\rbrak{\infty}$             
\item $\lbrak{-\infty},\rsbrak{-2} \cup \lsbrak{5},\rbrak{\infty}$
\item $\lbrak{-\infty},\rsbrak{-3} \cup \lsbrak{5},\rbrak{\infty}$
\item $\lbrak{-\infty},\rsbrak{-2} \cup \lsbrak{6},\rbrak{\infty}$
\end{enumerate}
%\end{enumerate}
