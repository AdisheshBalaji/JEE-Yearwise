\iffalse
\title{2023}
\author{EE24BTECH11063}
\section{mcq-single}
\fi
    \item Let $x_1,x_2,\cdots ,x_{100}$ be in an arithmetic progression, with $x_1=2$ and their mean equal to 200. If $y_i=i\brak{x_i-i},1 \le i \le 100$, then the mean of $y_1,y_2,\cdots y_{100}$ is : \hfill{[April 2023]}
    \begin{enumerate}
    \begin{multicols}{4}
    \item 10051.50
    \item 10100
    \item 10101.50
    \item 10049.50
    \end{multicols}
        \end{enumerate}
        \bigskip
        \item The number of elements in the set $S=\{\theta \in \sbrak{0,2\pi}:3\cos^{4}{\theta}-5\cos^{2}{\theta}-2\sin^{6}{\theta}+2=0\}$ is: \hfill{[April 2023]}
        \begin{enumerate}
        \begin{multicols}{4}
            \item 10
            \item 9
            \item 8
            \item 12
            \end{multicols}
        \end{enumerate}
        \bigskip
\item The value of the integral $\int_{-\log_{e}{2}}^{\log_{e}{2}} e^x\brak{\log_{e}{\brak{e^x+\sqrt{1+e^{2x}}}}}  \, dx$ is equal to \hfill{[April 2023]}
        \begin{enumerate}
        \begin{multicols}{2}
        \item $\log_{e}{\frac{\brak{2+\sqrt{5}}^2}{\sqrt{1+\sqrt{5}}}}+\frac{\sqrt{5}}{2}$
          \item $\log_{e}{\frac{2\brak{2+\sqrt{5}}^2}{\sqrt{1+\sqrt{5}}}}-\frac{\sqrt{5}}{2}$ 
          \item $\log_{e}{\frac{\sqrt{2}\brak{3-\sqrt{5}}^2}{\sqrt{1+\sqrt{5}}}}+\frac{\sqrt{5}}{2}$ 
          \item $\log_{e}{\frac{\sqrt{2}\brak{2+\sqrt{5}}^2}{\sqrt{1+\sqrt{5}}}}-\frac{\sqrt{5}}{2}$
        \end{multicols}
        \end{enumerate}
        \bigskip
    \item Let $S=\{M=\sbrak{a_{ij}},a_{ij} \in \{0,1,2\},1 \le i,j \le 2\}$ be a sample space and $A=\{M \in S : M\text{ is invertible}\}$ be an event. Then $\mathbb{P}\brak{A}$ is equal to: \hfill{[April 2023]}
    \begin{enumerate}
        \begin{multicols}{4}
            \item $\frac{16}{27}$
            \item $\frac{50}{81}$
            \item $\frac{47}{81}$
            \item $\frac{49}{81}$
        \end{multicols}
        \end{enumerate}
\bigskip
 
 \item Let $f\;:\;\sbrak{2,4} \rightarrow R$ be a differentiable function such that $\brak{x\log_{e}{x}}f'\brak{x}\;+\;\brak{\log_{e}{x}}f\brak{x}+f\brak{x}\; \ge \; 1, x\in \sbrak{2,4} $ with $f\brak{2}=\frac{1}{2}$ and $f\brak{4}=\frac{1}{4}$. Consider the following two statements: \\
 \brak{A} : $f\brak{x}\le 1$, for all $x \in \sbrak{2,4}$ \\
 \brak{B} : $f\brak{x}\ge \frac{1}{8}$, for all $x \in \sbrak{2,4}$\\
 Then, \hfill{[April 2023]}
 \begin{enumerate}
     \item Only statement \brak{B} is true
     \item Only statement \brak{A} is true 
     \item Neither statement \brak{A} nor statement \brak{B} is true
     \item Both the statements \brak{A} and \brak{B} are true 
 \end{enumerate}
 
 \bigskip
 \item Let $A$ be a $2 \times 2$ matrix with real entries such that $A^{T}=\alpha A + I$, where $a \in R -\{-1,1\}$. If det $\brak{A^2-A}=4$, then the sum of all possible values of $\alpha$ is equal to: \hfill{[April 2023]}
 \begin{enumerate}
     \begin{multicols}{4}
         \item 0
         \item $\frac{5}{2}$
         \item 2
         \item $\frac{3}{2}$
     \end{multicols}
 \end{enumerate}
\bigskip
 \item The number of integral solutions x of $\log_{\brak{x+\frac{7}{2}}}{\brak{\frac{x-7}{2x-1}}}^2$ \hfill{[April 2023]}
 \begin{enumerate}
     \begin{multicols}{4}
         \item 5
         \item 7
         \item 8
         \item 6
     \end{multicols}
 \end{enumerate}
 \bigskip
 \item Let $\overset{\rightarrow}{a}=a_1\hat{i}+a_2\hat{j}+a_3\hat{k}$ and $0<|a_i|<1\;,\;i=\{1,2,3\}$\\
\textbf{Statement-A} : $|a|\ge \text{max}\{|a_1|,|a_2|,|a_3|\}$ \\
\textbf{Statement-B} : $|a|< \text{3max}\{|a_1|,|a_2|,|a_3|\}$ \hfill{[April 2023]}
\begin{enumerate}
    \item Both A and B are true
    \item Both A and B are false
    \item A is true, B is false
    \item A is false, B is true
\end{enumerate}
\bigskip
 \item The number of triplets \brak{x,y,z}, where x,y,z are distinct non-negative integers satisying $x+y+z=15$, is : \hfill{[April 2023]}
 \begin{enumerate}
    \begin{multicols}{4}
    \item 136
    \item 114
    \item 80
    \item 92
    \end{multicols}
        \end{enumerate}
\bigskip
\item Let sets $A$ and $B$ have 5 elements each. Let mean of the elements in sets $A$ and $B$ be 5 and 8 respectively and the variance of the elements in sets $A$ and $B$ be 12 and 20 respectively. A new set $C$ of 10 elements is formed by subtracting 3 from each element of $A$ and adding 2 to each element of $B$. Then the sum of the mean and variance of the elements of $C$ is \hfill{[April 2023]}
 \begin{enumerate}
    \begin{multicols}{4}
    \item 36
    \item 40
    \item 32
    \item 38
    \end{multicols}
        \end{enumerate}
\bigskip


\item Area of the region $\{\brak{x,y}:x^2+\brak{y-2}^2 \le 4,x^2 \ge 2y\}$ is : \hfill{[April 2023]}
\begin{enumerate}
    \begin{multicols}{4}
        \item $\pi+\frac{8}{3}$
        \item $2\pi+\frac{16}{3}$
        \item $2\pi-\frac{16}{3}$
        \item $\pi-\frac{8}{3}$
    \end{multicols}
\end{enumerate}
\bigskip
\item Let $R$ be a rectangle given by the line $x=0,x=2,y=0 \text{ and } y=5$. Let $A\brak{\alpha,0}$ and $B\brak{0,\beta}$, $\alpha \in \sbrak{0,2}$ and $\beta \in \sbrak{0,5}$, be such that the line segment $AB$ divides the area of the rectangle $R$ in the ratio 4:1. Then, the mid-point of $AB$ lies on a : \hfill{[April 2023]}
\begin{enumerate}
    \begin{multicols}{4}
        \item straight line
        \item parabola
        \item circle 
        \item hyperbola
    \end{multicols}
\end{enumerate}
\bigskip 
\item Let $\overset{\rightarrow}{a}$ be a non-zero vector parallel to the line of intersection of the two planes described by $\hat{i}+\hat{j},\hat{i}+\hat{k}$ and $\hat{i}-\hat{j},\hat{j}-\hat{k}$. If $\theta$ is the angle between the vector $\overset{\rightarrow}{a}$ and the vector $\overset{\rightarrow}{b}=2\hat{i}-2\hat{j}+\hat{k}$ and $\overset{\rightarrow}{a}\cdot\overset{\rightarrow}{b}=6$, then ordered pair $\brak{\theta,|\overset{\rightarrow}{a}\times\overset{\rightarrow}{b}|}$ is equal to: \hfill{[April 2023]}
\begin{enumerate}
    \begin{multicols}{4}
    \item $\brak{\frac{\pi}{3},6}$
    \item $\brak{\frac{\pi}{4},3\sqrt{6}}$
    \item $\brak{\frac{\pi}{3},3\sqrt{6}}$
        \item $\brak{\frac{\pi}{4},6}$
    \end{multicols}
\end{enumerate}
\bigskip
\item Let $w_1$ be the point obtained by the rotation of $z_1=5+4i$ about the origin through a right angle in the anti-clockwise direction, and $w_2$ be the point obtained by the rotation of $z_2=3+5i$ about the origin through a right angle in the clockwise direction, Then the principal argument of $w_1-w_2$ is equal to : \hfill{[April 2023]}
\begin{enumerate}
    \begin{multicols}{4}
        \item $\pi-\tan^{-1}{\frac{8}{9}}$
        \item $-\pi+\tan^{-1}{\frac{8}{9}}$
        \item $\pi-\tan^{-1}{\frac{33}{5}}$
        \item $-\pi+\tan^{-1}{\frac{33}{5}}$
    \end{multicols}
\end{enumerate}
\bigskip
\item Consider ellipse $E_{k}:kx^2\;+\;ky^2=1,k=1,2,\cdots,20$. Let $C_k$ be the circle which touches the four chords joining the end points(one on minor axis and another on major axis) of the ellipse $E_k$. If $r_k$ is the radius of the circle $C_k$ then the value of $\sum_{k=1}^{20} \frac{1}{{r_k}^2}$ is \hfill{[April 2023]}
\begin{enumerate}
    \begin{multicols}{4}
        \item 3320
        \item 3210
        \item 3080
        \item 2870
    \end{multicols}
\end{enumerate}
