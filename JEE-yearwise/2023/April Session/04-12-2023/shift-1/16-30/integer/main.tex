\iffalse
\title{2023}
\author{EE24BTECH11008}
\section{integer}
\fi
%\begin{enumerate}
           \item A fair $n\brak{n \textgreater 1}$ faces die rolled repeatedly until a number less than $n$ appears. If the mean of the number of tosses required is $\frac{n}{9},$ then $n$ is equal to $\dots$
		  \hfill{\brak{2023-Apr}} \\
    \item Let the digits $a,b,c$ be in A.P. Nine-digit numbers are to be formed using each of three such that three consecutive digits are in A.P at least once. How many such numbers can be formed $?$
	  \hfill{\brak{2023-Apr}}  \\
    \item Let $\sbrak{x}$ be the greatest integer $\le x.$ Then the number of points in the interval $\brak{-2,1},$ where the function $f\brak{x}=\abs{\sbrak{x}}+\sqrt{x-\sbrak{x}}$ is discontinuous is $\dots$
	   \hfill{\brak{2023-Apr}} \\
    \item Let the plane $x+3y-2z+6=0$ meet the coordinate axes at the points $A,B,C.$ If the orthocentre of the triangle $ABC$ is $\brak{\alpha,\beta,\frac{6}{7}},$ then $98\brak{\alpha+\beta}^2$ is equal to $\dots$
	   \hfill{\brak{2023-Apr}} \\
    \item Let $I\brak{x}=\int\sqrt{\frac{x+7}{x}}dx$ and $I\brak{9}=12+7\log_e7.$ If $I\brak{1}=\alpha+7\log_e\brak{1+2\sqrt{2}},$ then $\alpha ^4$ is equal to $\dots$
	   \hfill{\brak{2023-Apr}} \\
    \item Let $D_k=\myvec{1&2k&2k-1\\n&n^2+n+2&n^2\\n&n^2+n&n^2+n+2}.$ If $\sum_{k=1}^nD_k=96,$ then $n$ is equal to $\dots$
	  \hfill{\brak{2023-Apr}}  \\
    \item Let the positive numbers $a_1,a_2,a_3,a_4$ and $a_5$ be in G.P. Let their mean and variance be $\frac{31}{10}$ and $\frac{m}{n}$ respectively,where $m$ and $n$ are co-prime. If the mean of their reciprocals is $\frac{31}{40}$ and $a_3+a_4+a_5=14,$ then $m+n$ is equal to $\dots$
	   \hfill{\brak{2023-Apr}} \\
    \item The number of relations, on the set ${1,2,3}$ containing $\brak{1,2}$ and $\brak{2,3},$ which are reflexive and transitive but not symmetic, is $\dots$
	  \hfill{\brak{2023-Apr}}  \\
    \item If $\int_{-0.15}^{0.15}\abs{100x^2-1}dx=\frac{k}{3000},$ then $k$ is equal to $\dots$
	   \hfill{\brak{2023-Apr}} \\
    \item Two circles in the first quadrant of radii $r_1$ and $r_2$ touch the coordinate axes. Each of them cuts off an intercept of $2$ units with the line $x+y=2.$ Then $r_1^2+r_2^2-r_1r_2$ is equal to $\dots$
	    \hfill{\brak{2023-Apr}} \\
%\end{enumerate}

