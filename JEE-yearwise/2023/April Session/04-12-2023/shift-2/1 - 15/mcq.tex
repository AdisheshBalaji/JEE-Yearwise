\iffalse
\title{2023}
\author{AI24BTECH11009}
\section{mcq-single}
\fi
\item If $f\brak{x} = \frac{x^3}{x-1} + \frac{x^3}{x+1}$ and $g\brak{x} = \sqrt{x}$, then $fog: \sbrak{0, 2}-\{1\}\rightarrow\mathbb{R}$ is: \hfill[Apr 2023]
    \begin{enumerate}
        \item one-one but not onto
        \item one-one and onto
        \item onto but not one-one
        \item neither one-one nor onto \\
    \end{enumerate}
\item The sum of squares of the values of $\alpha\in\mathbb{R}$ such that the argument of $\frac{\alpha+i}{\alpha-i}$ is 60\degree, $i=\sqrt{-1}$, is: \hfill[Apr 2023]
\begin{enumerate}
    \item 3
    \item $\frac{10}{3}$
    \item $\frac{11}{3}$
    \item 4 \\
\end{enumerate}
\item For the system of equations
\begin{align*}
    x + \lambda y - z & = 1 \\
    x + 2y + \lambda z & = 2 \\
    x + 2y + z & = 2,
\end{align*}
which one of the following is \textbf{NOT} correct? \hfill[Apr 2023]
\begin{enumerate}
    \item it has unique solution if $\lambda$ is not a root of the equation $t^2 - 3t + 2 = 0$
    \item it has unique solution if $\lambda$ is not a root of the equation $t^2 - t - 2 = 0$
    \item it has infinitely many solutions if $\lambda = 1$
    \item it has no solution if $\lambda = 2$ \\
\end{enumerate}
 \item If $a_n = \brak{2n^2 - n + 2}\brak{n!}$ then $\sum\limits_{n=1}^{20}a_n$ is equal to: \hfill[Apr 2023]
 \begin{enumerate}
     \item $37 \brak{20!} - 1$
     \item $37 \brak{20!} + 1$
     \item $39 \brak{21!} + 1$
     \item $39 \brak{21!} - 1$ \\
 \end{enumerate}
\item \begin{align*}
    \lim\limits_{x \rightarrow 0^{+}}\frac{1}{\sqrt{x}}\brak{\frac{1}{\sin\brak{x}} - \frac{1}{x}}
\end{align*} \hfill[Apr 2023]
\begin{enumerate}
    \item is equal to 0
    \item is equal to $\frac{1}{5}$
    \item is equal to 1
    \item does not exist \\
\end{enumerate}
\item  Let $x = x\brak{t}$ be the solution curve of the differential equation $\frac{dx}{dt} = -kx$, and $x\brak{0} = 100$, $x\brak{\frac{1}{2}} = 80$. If $x\brak{t_{\alpha}} = 5$, then $t_{\alpha}$ is equal to: \hfill[Apr 2023]
\begin{enumerate}
    \item $\frac{\log_{e}5 + \log_{e}4}{2\brak{\log_{e}5 - \log_{e}4}}$
    \item $\frac{\log_{e}5 + \log_{e}4}{\log_{e}5 - \log_{e}4}$
    \item $\frac{\log_{e}5 - \log_{e}4}{2\brak{\log_{e}5 + \log_{e}4}}$
    \item $\frac{\log_{e}5 - \log_{e}4}{\log_{e}5 + \log_{e}4}$ \\
\end{enumerate}
\item The slope of the tangent to the curve
\begin{align*}
  = y\brak{x} = \int_{\sin^{-1}\brak{x}}^{\cos^{-1}\brak{x}}\sqrt{1 + 4\sin^2t}dt, 0 \leq x \leq 1
\end{align*}
at the point \brak{\frac{1}{\sqrt{2}}, 0} on the curve is: \hfill[Apr 2023]
\begin{enumerate}
    \item $-2\sqrt{6}$
    \item $2\sqrt{6}$
    \item $-4\sqrt{3}$
    \item $4\sqrt{3}$\\
\end{enumerate}
 \item Let $\alpha, \beta, \gamma \brak{0 < \alpha, \beta, \gamma < \frac{\pi}{2}}$ be the angles between non-zero vectors $\vec{a}$ and $\vec{b}$, $\vec{b}$ and $\vec{c}$, $\vec{c}$ and $\vec{a}$ respectively. If $\theta$ is the angle that the vector $\vec{a}$ makes with the plane containing $\vec{b}$ and $\vec{c}$, then \hfill[Apr 2023]
 \begin{enumerate}
     \item $\cos^2\theta = \cosec^2\beta\brak{\cos^2\alpha + \cos^2\gamma - 2\cos\alpha\cos\beta\cos\gamma}$
     \item $\cos^2\theta = \sec^2\beta\brak{\cos^2\alpha + \cos^2\gamma + 2\cos\alpha\cos\beta\cos\gamma}$
     \item $\sin^2\theta = \cosec^2\beta\brak{\cos^2\alpha + \cos^2\gamma - 2\cos\alpha\cos\beta\cos\gamma}$
     \item $\sin^2\theta = \sec^2\beta\brak{\cos^2\alpha + \cos^2\gamma + 2\cos\alpha\cos\beta\cos\gamma}$ \\
 \end{enumerate}
\item The domain of the function
\begin{align*}
    f\brak{x} = \sin^{-1}\brak{\log_2\brak{\brak{x-1}\brak{x-2}}}
\end{align*}
is: \hfill[Apr 2023]
\begin{enumerate}
     \item $\sbrak{0, 3}$
     \item $\sbrak{0, \frac{3-\sqrt{3}}{2}}\cup\sbrak{\frac{3+\sqrt{3}}{2}, 3}$
     \item $\brak{\frac{3-\sqrt{3}}{2}, 1}\cup\brak{2, \frac{3+\sqrt{3}}{2}}$
     \item $\sbrak{0, \frac{3-\sqrt{6}}{2}}\cup\sbrak{\frac{3+\sqrt{6}}{2}, 3}$ \\
 \end{enumerate}
\item Let $\alpha$ and $\beta$ be the roots of the equation $2x^2 - 5x - 1 = 0$. For $n\in\mathbb{N}$, let $P_n = \alpha^n + \beta^n$. Then $\frac{2P_{11}\brak{2P_{10}-5P_9}}{P_8\brak{5P_{10}+P_9}}$ is equal to: \hfill[Apr 2023]
\begin{enumerate}
    \item $-\frac{1}{2}$
    \item $\frac{1}{2}$
    \item -1
    \item 1\\
\end{enumerate}
\item If the image of the point \brak{1,1,2} in the plane $2x - y + z + 3 = 0$ is the point $P$, then the distance of $P$ from origin is \hfill[Apr 2023]
\begin{enumerate}
    \item $2\sqrt{3}$
    \item $3\sqrt{2}$
    \item 4
    \item 6 \\
\end{enumerate}
\item For three non-coplanar vectors $\vec{a}, \vec{b}, \vec{c}$, if $\brak{\vec{b}+\vec{c}}\cdot\left\{\brak{\vec{c}+\vec{a}}\times\brak{\vec{a}+\vec{b}}\right\} = \alpha\sbrak{\vec{a}\ \vec{b}\ \vec{c}}$ and $\brak{\vec{a}+\vec{b}}\cdot\left\{\brak{\vec{b}+\vec{c}}\times\brak{\vec{a}+\vec{b}+\vec{c}}\right\} = \beta\sbrak{\vec{a}\ \vec{b}\ \vec{c}}$, then $\alpha+\beta$ is equal to: \hfill[Apr 2023]
\begin{enumerate}
    \item -3
    \item -1
    \item 1
    \item 3 \\
\end{enumerate}
\item If $I\brak{x} = \int \frac{dx}{1 - 2\sin^2x\cos^2x}$, then $\tan\brak{\sqrt{2}\brak{I\brak{\frac{\pi}{8}} - I\brak{0}}}$ is equal to: \hfill[Apr 2023]
\begin{enumerate}
    \item $\frac{1}{\sqrt{2}}$
    \item 1
    \item $\sqrt{2}$
    \item 2 \\
\end{enumerate}
\item Let $X$ have a binomial distribution $B\brak{6,p}$. If the sum of the mean and the variance of $X$ is $\frac{21}{8}$, then $\frac{P\brak{2 \leq X < 4}}{P\brak{4 < X < 6}}$ is equal to: \hfill[Apr 2023]
\begin{enumerate}
   \item 65
   \item 195
   \item $\frac{195}{2}$
   \item $\frac{225}{2}$ \\
\end{enumerate}
\item If the statement 
\begin{align*}
    \brak{\brak{p*\brak{\sim q}}\land\brak{p \lor q}} \iff p
\end{align*}
is a tautology, then $*$ is: \hfill[Apr 2023]
\begin{enumerate}
    \item $\land$
    \item $\lor$
    \item $\implies$
    \item $\iff$ \\
\end{enumerate}
 
