\iffalse
    \title{2023}
    \author{EE24BTECH11001}
    \section{mcq-single}
\fi
    \item
        The statement $\brak{p \land \brak{\sim q}} \implies \brak{p \implies \brak{\sim q}}$
        \hfill{\brak{\textnormal{2023-Jan}}}
        \begin{multicols}{4}
            \begin{enumerate}
                \item equivalent to $\brak{\sim p} \lor \brak{\sim q}$
                    \columnbreak
                \item a tautology
                    \columnbreak
                \item equivalent $p \lor q$
                    \columnbreak
                \item a contradiction
            \end{enumerate}
        \end{multicols}

    \item Let $f : \brak{0, 1} \rightarrow \mathbb{R}$ a be a function defined by
        \begin{align}
            f\brak{x} = \frac{1}{1 - e^{-x}}
        \end{align} and, 
        \begin{align}
            g\brak{x} = \brak{f\brak{-x} - f\brak{x}}
        \end{align}. Consider the two statements
        \begin{enumerate}
            \item $g$ is an increasing function $\brak{0, 1}$
            \item $g$ is one-one in $\brak{0, 1}$
        \end{enumerate} Then,
        \hfill{\brak{\textnormal{2023-Jan}}}
        \begin{multicols}{4}
            \begin{enumerate}
                \item Only 1 is true \columnbreak
                \item Only 2 is true \columnbreak
                \item Neither 1 nor 2 is true \columnbreak
                \item Both 1 and 2 are true
            \end{enumerate}
        \end{multicols}


    \item The distance of the point $P\brak{4, 6, -2}$ from the line passing through
        the point $\brak{-3, 2, 3}$ and parallel to a line with direction ratios 3, 3, -1 is
        equal to :
        \hfill{\brak{\textnormal{2023-Jan}}}
        \begin{enumerate}
                \begin{multicols}{2}
                \item 3\columnbreak
                \item $\sqrt{6}$
                \end{multicols}
                \begin{multicols}{2}
                \item $2\sqrt{3}$ \columnbreak
                \item $\sqrt{14}$
                \end{multicols}
        \end{enumerate}
        \begin{figure}[ht]
            \centering
            \begin{tikzpicture}
                \draw (-1.5, 0) -- (1.5, 0)  ;
                \draw (0, -1.5) -- (0, 1.5) ;
                \fill[black] (0, 1.5) circle (1pt) node[right] {\small $\brak{4, 6, -2}$};
            \end{tikzpicture}
        \end{figure}
    \item  Let $x, y, z, > 1$ and   
        \begin{align}
            A = \mydet{1 & \log_x y & \log_x z \\ \log_y x & 2 & \log_y z \\
            \log_z x & \log_x y & 3}
        \end{align} 
        Then $\abs{\textnormal{adj}\brak{\textnormal{adj}A^2}}$ is equal to :
        \hfill{\brak{\textnormal{2023-Jan}}}
        \begin{enumerate}
                \begin{multicols}{4}
                \item $6^4$ \columnbreak
                \item $2^8$ \columnbreak
                \item $4^8$ \columnbreak
                \item $2^4$
                \end{multicols}
        \end{enumerate}

    \item If $a_r$ is the coefficient of $x^{10-r}$ in the Binomial expansion of
        $\brak{1 + x}^10$, then $\sum_{r = 1} ^ {10} r^3\brak{\frac{a_r}{a_{r-1}}^2}$ is equal to :
        \hfill{\brak{\textnormal{2023-Jan}}}
        \begin{multicols}{4}
            \begin{enumerate}
                \item 4895 \columnbreak
                \item 1210 \columnbreak 
                \item 5445 \columnbreak  
                \item 3025
            \end{enumerate}
        \end{multicols}
