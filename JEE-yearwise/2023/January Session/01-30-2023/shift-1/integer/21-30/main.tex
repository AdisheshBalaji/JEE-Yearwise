\iffalse
\title{2023}
\author{AI24BTECH11006}
\section{integer}
\fi
\item Let $S = \sbrak{1, 2, 3, 4, 5, 6}$. The number of one-to-one functions $f : S \to P\brak{S}$, such that $f\brak{n} \subset f\brak{m}$ where $n \textless m$, is equal to $\cdots$
	\hfill{\sbrak{January-2023}}
\item The number of four-digit numbers $\brak{repetition of digits allowed}$ made using the digits $1, 2, 3, and 5$ and divisible by $15$, is $\cdots$
	\hfill{\sbrak{January-2023}}
\item If $\lambda_1$ \textless$ \lambda_2$ are two values of $\lambda$ such that the angle between the planes $P_1:\Bar{r}\brak{3\hat{i} - 5\hat{j} + \hat{k} }= 7$ and $P_2: \Bar{r}\brak{\lambda \hat{i} +\hat{j} - 3\hat{k}} = 9$ is 
$\sin^{-1}\brak{\frac{2\sqrt{6}}{5}}$, then the square of the length of the perpendicular from the point $\brak{38\lambda_1, 10\lambda_2, 2}$ to the plane $P_1$ is $\cdots$
\hfill{\sbrak{January-2023}}
\item Let $\sum_{n=0}^{\infty} \frac{n^3\brak{\brak{2n}!}+\brak{2n-1}\brak{n!}}{\brak{n!}\brak{\brak{2n}!}} = ae+\frac{b}{e}+c$, where $a, b, c \in \mathbb{Z}$ and $e=\sum_{n=0}^{\infty}\frac{1}{n!}$.Then $a^2 - b + c$ is equal to $\cdots$
	\hfill{\sbrak{January-2023}}
\item Let $ z = 1 + i $ and $z_1=\frac{1+i\Bar{z}}{\Bar{z}\brak{1-z}+\frac{1}{z}}$. Then $\frac{12}{\pi}\arg\brak{z_1}$ is equal to $\cdots$
	\hfill{\sbrak{January-2023}}
\item Let $f^1\brak{x} =\frac{3x+2}{2x+3} , x \in \mathbb{R}-\{\frac{-3}{2}\}$.For $ n \geq 2 $, define $f^n\brak{x} = f^1of^{n-1}\brak{x}.$If$ f^5\brak{x} = \frac{ax + b}{bx + a},$gcd$\brak{a, b} = 1$,
then  a + b  is equal to $\cdots$
\hfill{\sbrak{January-2023}}
\item $\lim_{x \to 0} \frac{48}{x^4} \int_0^{x} \frac{t^3}{t^6+1}dt$ is equal to $\cdots$
	\hfill{\sbrak{January-2023}}
\item The mean and variance of $7$ observations are $8$ and
$16$respectively. If one observation $14$ is omitted
and a and b are respectively mean and variance of
remaining $6$ observation, then $a + 3b -5$ is equal to $\cdots$
\hfill{\sbrak{January-2023}}
\item If the equation of the plane passing through the
point $\brak{1, 1, 2}$ and perpendicular to the line$\brak{ x - 3y +
2z - 1 = 0 = 4x - y + z}$ is Ax + By + Cz = 1, then
		$140\brak{C - B + A}$ is equal to $\cdots$
		\hfill{\sbrak{January-2023}}
\item Let $\alpha$ be the area of the larger region bounded by the curve $y^2 = 8x$, the line $y = x$, and $x = 2$, which lies in the first quadrant. Then the value of $3\alpha$ is equal to $\cdots$
	\hfill{\sbrak{January-2023}}

