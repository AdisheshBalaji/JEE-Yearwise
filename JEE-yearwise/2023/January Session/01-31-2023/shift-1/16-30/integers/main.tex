\iffalse
   \title{Assignment}
   \author{EE24BTECH11034}
   \section{integer}
\fi 
    \item Let $5$ digit numbers be constructed using the digits $0, 2, 3, 4, 7, 9$ with repetition allowed, and are arranged in ascending order with serial numbers. Then the serial number of the number $42923$ is:\hfill{jan 2023}

       
    \item Let $a_{1}, a_{2},\ldots,a_{n}$ be in A.P. If $a_{5}=2a_{7}$ and $a_{11}=18$, then
    $
    12\left(\frac{1}{\sqrt{a_{10}}+\sqrt{a_{11}}}+\frac{1}{\sqrt{a_{11}}+\sqrt{a_{12}}}+\cdots+\frac{1}{\sqrt{a_{17}}+\sqrt{a_{18}}}\right)
    $
    is equal to:\hfill{jan 2023}
    
      
   
    \item Let $\theta$ be the angle between the planes $P_{1}=\vec{r}\cdot\brak{\hat{i}+\hat{j}+2\hat{k}}=9$ and $P_{2}=\vec{r}\cdot\brak{2\hat{i}-\hat{j}+\hat{k}}=15$. Let $L$ be the line that meets $P_{2}$ at the point $\brak{4,-2, 5}$ and makes an angle $\theta$ with the normal of $P_{2}$. If $\alpha$ is the angle between $L$ and $P_{2}$, then $\brak{\tan^{2}\theta}\brak{\cot^{2}\alpha}$ is equal to:\hfill{jan 2023}

    \item Let $\alpha > 0$ be the smallest number such that the expansion of $\brak{x^{\frac{2}{3}}+\frac{2}{x^{3}}}^{30}$ has a term $\beta x^{-\alpha}$, $\beta\in\mathbb{N}$. Then $\alpha$ is equal to:\hfill{jan 2023}


    \item Let $\vec{a}$ and $\vec{b}$ be two vectors such that $\abs{\vec{a}}=\sqrt{14}$, $\abs{\vec{b}}=\sqrt{6}$, and $\abs{\vec{a}\times\vec{b}}=\sqrt{48}$. Then $\brak{\vec{a}\cdot\vec{b}}^{2}$ is equal to:\hfill{jan 2023}

       
    \item Let the line $L:\frac{x-1}{2}=\frac{y+1}{-1}=\frac{z-3}{1}$ intersect the plane $2x+y+3z=16$ at the point $P$. Let the point $Q$ be the foot of perpendicular from the point $R\brak{1,-1,-3}$ on the line $L$. If $\alpha$ is the area of triangle $PQR$, then $\alpha^{2}$ is equal to:\hfill{jan 2023}

      

    \item The remainder on dividing $5^{99}$ by $11$ is:\hfill{jan 2023}

        
    \item If the variance of the frequency distribution
    \begin{tabular}{|c|c|c|c|c|c|c|c|}
    \hline
    $X_{i}$ & 2 & 3 & 4 & 5 & 6 & 7 & 8 \\ \hline
    Frequency $f_{i}$ & 3 & 6 & 16 & $\alpha$ & 9 & 5 & 6 \\ \hline
    \end{tabular}
    is $2.5$, then $\alpha$ is equal to:
    
       

    \item Let for $x\in\mathbb{R}$
     $f\brak{x}=\frac{x+\abs{x}}{2}$
     and $g\brak{x}=
     \begin{cases}
     x, & x<0 \\
     x^{2}, & x\geq0
     \end{cases}$.
     Then the area bounded by the curve $y=\brak{f\circ g}\brak{x}$ and the lines $y=0$, $2y-x=15$ is equal to: \hfill{jan 2023}
    

    \item Number of $4$-digit numbers that are less than or equal to $2800$ and either divisible by $3$ or by $11$, is equal to:\hfill{jan 2023}
