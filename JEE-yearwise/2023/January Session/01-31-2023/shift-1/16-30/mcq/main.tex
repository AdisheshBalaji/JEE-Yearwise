\iffalse
   \title{Assignment}
   \author{EE24BTECH11034}
   \section{mcq-single}
\fi 
    \item If $\sin^{-1}\brak{\frac{\alpha}{17}} + \cos^{-1}\brak{\frac{4}{5}} - \tan^{-1}\brak{\frac{77}{36}} = 0$, $0 < \alpha <13$, then $\sin^{-1}\brak{\sin \alpha} + \cos^{-1}\brak{\cos \alpha}$ is equal to:\hfill{jan 2023}
        
        \begin{multicols}{4}
        \begin{enumerate}
        \item $\pi$
        \item $16$
        \item $0$
        \item $16 - 5\pi$
        \end{enumerate}
        \end{multicols}

    \item Let a circle $C_{1}$ be obtained on rolling the circle $x^{2}+y^{2}-4x-6y+11=0$ upwards $4$ units on the tangent $T$ to it at the point $\brak{3, 2}$. Let $C_{2}$ be the image of $C_{1}$ in $T$. Let $A$ and $B$ be the centers of circles $C_{1}$ and $C_{2}$ respectively, and $M$ and $N$ be respectively the feet of perpendiculars drawn from $A$ and $B$ on the x-axis. Then the area of the trapezium $AMNB$ is:\hfill{jan 2023}
    
        \begin{multicols}{4}
        \begin{enumerate}
        \item $2\brak{2+\sqrt{2}}$
        \item $4\brak{1+\sqrt{2}}$
        \item $3+2\sqrt{2}$
        \item $2\brak{1+\sqrt{2}}$
        \end{enumerate}
        \end{multicols}
        
    \item S1: $\brak{p\Rightarrow q}\lor\brak{p\land\brak{\neg q}}$ is a tautology.

      S2: $\brak{\brak{\neg p}\Rightarrow\brak{\neg q}}\land\brak{\brak{\neg p}\lor q}$ is a contradiction. Then \hfill{jan 2023}
    
        \begin{enumerate}
        \item only $\brak{S2}$ is correct
        \item both $\brak{S1}$ and $\brak{S2}$ are correct
        \item both $\brak{S1}$ and $\brak{S2}$ are wrong
        \item only $\brak{S1}$ is correct
        \end{enumerate}
    
    \item The value of $\int_{\frac{\pi}{3}}^{\frac{\pi}{2}}\frac{\brak{2+3\sin{x}}}{\sin{x}\brak{1+\cos{x}}}dx$ is equal to: \hfill{jan 2023}

        \begin{enumerate}
        \item $\frac{7}{2}-\sqrt{3}-\log_{e}\sqrt{3}$
        \item $-2+3\sqrt{3}+\log_{e}\sqrt{3}$
        \item $\frac{10}{3}-\sqrt{3}+\log_{e}\sqrt{3}$
        \item $\frac{10}{3}-\sqrt{3}-\log_{e}\sqrt{3}$
        \end{enumerate}

    \item A bag contains $6$ balls. Two balls are drawn from it at random and both are found to be black. The probability that the bag contains at least $5$ black balls is: \hfill{jan 2023}

        \begin{multicols}{4}
        \begin{enumerate}
        \item $\frac{5}{7}$
        \item $\frac{2}{7}$
        \item $\frac{3}{7}$
        \item $\frac{5}{6}$
        \end{enumerate}
        \end{multicols}
