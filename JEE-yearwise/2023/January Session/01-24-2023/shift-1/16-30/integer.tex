\iffalse
\title{2023}
\author{EE24Btech11024}
\section{integer}
\fi

\item Let $C$ be the largest circle centred at $\brak{2,0}$ and inscribed in the ellipse $\frac{x^2}{36}+\frac{y^2}{16}=1$. If \brak{1,\alpha} lies on $C$, then $10\alpha^2$ is equal to \rule{1cm}{0.15mm}.

\hfill{\brak{\text{Jan 2023}}}

\item Suppose $\sum_{r=0}^{2023}r^2\times\comb{2023}{r} = 2023\times\alpha\times 2^{2022}$. Then the value of $\alpha$ is \rule{1cm}{0.15mm}.

\hfill{\brak{\text{Jan 2023}}}

\item The value of $12\int_{0}^{3}\abs{x^2-3x+2}dx$ is \rule{1cm}{0.15mm}.

\hfill{\brak{\text{Jan 2023}}}

\item The number of $9$ digit numbers, that can be formed using all the digits of the number $123412341$ so that the even digits occupy only even places is \rule{1cm}{0.15mm}.

\hfill{\brak{\text{Jan 2023}}}

\item Let $\lambda \in \mathbb{R}$ and let the equation $E$ be $\abs{x}^2-2\abs{x}+\abs{\lambda-3}=0$. Then the largest element in set $S=\cbrak{x+\lambda:x\text{ is an integer solution of }E}$ is \rule{1cm}{0.15mm}.

\hfill{\brak{\text{Jan 2023}}}

\item A boy needs to select $5$ courses from $12$ available courses, out of which $5$ courses are language courses. If  he can choose at most $2$ language courses, then the number of ways he can choose five courses is \rule{1cm}{0.15mm}.

\hfill{\brak{\text{Jan 2023}}}

\item Let a tangent to the curve $9x^2+16y^2=144$ intersect coordinate axes at points $\vec{A}$ and $\vec{B}$. Then, the minimum length of the line segment $AB$ is \rule{1cm}{0.15mm}.

\hfill{\brak{\text{Jan 2023}}}

\item The value of $\frac{8}{\pi}\int_{0}^{\frac{\pi}{2}}\frac{\brak{\cos x}^{2023}}{\brak{\sin x}^{2023}+\brak{\cos x}^{2023}}dx$ is \rule{1cm}{0.15mm}. 

\hfill{\brak{\text{Jan 2023}}}

\item The shortest distance between the lines $\frac{x-2}{3}=\frac{y+1}{2}=\frac{z-6}{2}$ and $\frac{x-6}{3}=\frac{1-y}{2}=\frac{z+8}{0}$ is equal to \rule{1cm}{0.15mm}.

\hfill{\brak{\text{Jan 2023}}}

\item The $4$\textsuperscript{th} term of GP is $500$ and its common ratio is $\frac{1}{m}$, $m\in \mathbb{N}$. Let $S_n$ denote the sum of the first $n$ terms of this GP. If $S_6>S_5+1$ and $S_7>S_6+\frac{1}{2}$, then the number of possible values of $m$ is \rule{1cm}{0.15mm}.

\hfill{\brak{\text{Jan 2023}}}
