\iffalse
\let\negmedspace\undefined
\let\negthickspace\undefined
\documentclass[journal]{IEEEtran}
\usepackage[a5paper, margin=10mm, onecolumn]{geometry}
%\usepackage{lmodern} % Ensure lmodern is loaded for pdflatex
\usepackage{tfrupee} % Include tfrupee package

\setlength{\headheight}{1cm} % Set the height of the header box
\setlength{\headsep}{0mm}     % Set the distance between the header box and the top of the text

\usepackage{gvv-book}
\usepackage{gvv}
\usepackage{cite}
\usepackage{amsmath,amssymb,amsfonts,amsthm}
\usepackage{algorithmic}
\usepackage{graphicx}
\usepackage{textcomp}
\usepackage{xcolor}
\usepackage{txfonts}
\usepackage{listings}
\usepackage{enumitem}
\usepackage{mathtools}
\usepackage{gensymb}
\usepackage{comment}
\usepackage[breaklinks=true]{hyperref}
\usepackage{tkz-euclide} 
\usepackage{listings}
% \usepackage{gvv}                                        
\def\inputGnumericTable{}                                 
\usepackage[latin1]{inputenc}                                
\usepackage{color}                                            
\usepackage{array}                                            
\usepackage{longtable}                                       
\usepackage{calc}                                             
\usepackage{multirow}                                         
\usepackage{hhline}                                           
\usepackage{ifthen}                                           
\usepackage{lscape}
\bibliographystyle{IEEEtran}
\vspace{3cm}

\title{2024}
\author{EE24BTECH11061}
\maketitle

\renewcommand{\thefigure}{\theenumi}
\renewcommand{\thetable}{\theenumi}

\section{mcq-single}
\fi

%\begin{enumerate}
\item Let $f\brak{x}$ be a positive function such that the area bounded by $y=f\brak{x}$, $y=0$ from $x=0$ to $x=a>0$ is $e^{-a}+4a^2+a-1$. Then the differential equation, whose general solution is $y=c_1f\brak{x} + c_2$, where $c_1$ and $c_2$ are arbitrary constants, is
\hfill{\sbrak{\text{April 2024}}}
\begin{multicols}{2}
    \begin{enumerate}
        \item $\brak{8e^x+1}\frac{d^2y}{dx^2} + \frac{dy}{dx} = 0$
        \item $\brak{8e^x-1}\frac{d^2y}{dx^2} - \frac{dy}{dx} = 0$
        \item $\brak{8e^x+1}\frac{d^2y}{dx^2} - \frac{dy}{dx} = 0$
        \item $\brak{8e^x-1}\frac{d^2y}{dx^2} + \frac{dy}{dx} = 0$
    \end{enumerate}
\end{multicols}

\item The number of critical points of the function $f\brak{x} = \brak{x-2}^{\frac{2}{3}}\brak{2x+1}$ is
\hfill{\sbrak{\text{April 2024}}}
\begin{multicols}{2}
    \begin{enumerate}
        \item 0
        \item 1
        \item 3
        \item 2
    \end{enumerate}
\end{multicols}

\item Let the sum of two positive integers be 24. If the probability, that their product is not less than $\frac{3}{4}$ times their greatest possible product, is $\frac{m}{n}$, where $gcd(m,n) = 1$, then $n-m$ equals
\hfill{\sbrak{\text{April 2024}}}
\begin{multicols}{2}
    \begin{enumerate}
        \item 11
        \item 10
        \item 9
        \item 8
    \end{enumerate}
\end{multicols}

\item Let $H:\frac{-x^2}{a^2} + \frac{y^2}{b^2} = 1$ be the hyperbola, whose eccentricity is $\sqrt{3}$ and the length of the latus rectum is $4\sqrt{3}$. Suppose the point $\brak{\alpha,6}$, $\alpha>0$ lies on $H$. If $\beta$ is the product of the focal distances of the point $\brak{\alpha,6}$, then $\alpha^2 + \beta$ is equal to
\hfill{\sbrak{\text{April 2024}}}
\begin{multicols}{2}
    \begin{enumerate}
        \item 169
        \item 171
        \item 170
        \item 172
    \end{enumerate}
\end{multicols}

\item The sum of all the solutions of the equation $8^{2x} - 16 . 8^x + 48 = 0$ is:
\hfill{\sbrak{\text{April 2024}}}
\begin{multicols}{2}
    \begin{enumerate}
        \item $\log_8 {6}$
        \item $\log_8 {4}$
        \item $1+\log_8 {6}$
        \item $1+\log_6 {8}$
    \end{enumerate}
\end{multicols}
%\end{enumerate}