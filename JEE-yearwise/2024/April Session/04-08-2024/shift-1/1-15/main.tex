\iffalse
\title{April 2024}
\author{AI24BTECH11012}
\section{mcq-single}
\fi
    \item If the image of the point (-4, 5) in the line $x + 2y = 2$ lies on the circle $(x + 4)^2 + (y - 3)^2 = r^2$, then $r$ is equal to:
    \hfill{[April 2024]}
    \begin{enumerate}
        \item $1$
        \item $2$
        \item $75$
        \item $3$
    \end{enumerate}
    
    \item Let $\overrightarrow{a} = \hat{i} + 2\hat{j} + 3\hat{k}, \ \overrightarrow{b} = 2\hat{i} + 3\hat{j} - 5\hat{k}, \overrightarrow{c} = 3\hat{i} - \hat{j} + \lambda\hat{k}$ be three vectors. Let $\overrightarrow{r}$ be a unit vector along $\overrightarrow{b} + \overrightarrow{c}$. If $\overrightarrow{r} \cdot \overrightarrow{a} = 3$, then $3\lambda$ is equal to:
    \hfill{[April 2024]}
    \begin{enumerate}
        \item $27$
        \item $25$
        \item $15$
        \item $21$
    \end{enumerate}
    \item If $\alpha \neq a$, $\beta \neq b$, $\gamma \neq c$, and
$
\begin{vmatrix}
\alpha & \beta & 1 \\
a & b & 1 \\
c & c & 0 \\
\end{vmatrix} = 0,
$
then $\frac{a}{\alpha-a}+\frac{b}{\beta-b}+\frac{\gamma}{\gamma-c}$ is equal to:
\hfill{[April 2024]}
    \begin{enumerate}
        \item $2$
        \item $3$
        \item $0$
        \item $1$
    \end{enumerate}
    \item In an increasing geometric progression of positive terms, the sum of the second and sixth terms is $\frac{70}{3}$ and the product of the third and fifth terms is 49. Then the sum of the 4th, 6th, and 8th terms is:
    \hfill{[April 2024]}
    \begin{enumerate}
        \item $96$
        \item $78$
        \item $91$
        \item $84$
    \end{enumerate}
    \item The number of ways five alphabets can be chosen from the alphabets of the word "MATHEMATICS", where the chosen alphabets are not necessarily distinct, is:
    \hfill{[April 2024]}
    \begin{enumerate}
        \item $175$
        \item $181$
        \item $177$
        \item $179$
    \end{enumerate}
    \item The sum of all possible values of $\theta \in [-\pi, 2\pi]$ for which $\frac{1 + i\cos \theta}{1 - 2i\cos \theta}$ is purely imaginary is equal to:
    \hfill{[April 2024]}
    \begin{enumerate}
        \item $2\pi$
        \item $3\pi$
        \item $5\pi$
        \item $4\pi$
    \end{enumerate}
    \item If the system of equations $x + 4y - z = \lambda$, $7x + 9y + \mu z = -3$, $5x + y + 2z = -1$ has infinitely many solutions, then $2\mu + 3\lambda$ is equal to:
    \hfill{[April 2024]}
    \begin{enumerate}
        \item $2$
        \item $-3$
        \item $3$
        \item $-2$
    \end{enumerate}
    \item If the shortest distance between the lines $\frac{x - \lambda}{2} = \frac{y - 4}{3} = \frac{z - 3}{4}$ and $\frac{x - 2}{4} = \frac{y - 4}{6} = \frac{z - 7}{8}$ is $\frac{13}{\sqrt{29}}$, then a value of $\lambda$ is:
    \hfill{[April 2024]}
    \begin{enumerate}
        \item $-\frac{13}{25}$
        \item $\frac{13}{25}$
        \item $1$
        \item $-1$
    \end{enumerate}
    \item If the value of $\frac{3 \cos 36^\circ + 5 \sin 18^\circ}{5 \cos 36^\circ - 3 \sin 18^\circ}$ is $\frac{a \sqrt{5} - b}{c}$, where $a$, $b$, $c$ are natural numbers and gcd($a, c$) = 1, then $a + b + c$ is equal to:
    \hfill{[April 2024]}
    \begin{enumerate}
        \item $50$
        \item $40$
        \item $52$
        \item $54$
    \end{enumerate}
    \item Let $y = y(x)$ be the solution curve of the differential equation $\sec y \frac{dy}{dx} + 2x \sin y = x^3 \cos y$, $y(1) = 0$. Then $y(\sqrt{3})$ is equal to:
    \hfill{[April 2024]}
    \begin{enumerate}
        \item $\frac{\pi}{3}$
        \item $\frac{\pi}{6}$
        \item $\frac{\pi}{4}$
        \item $\frac{\pi}{12}$
    \end{enumerate}
    \item The area of the region in the first quadrant inside the circle $x^2 + y^2 = 8$ and outside the parabola $y^2 = 2x$ is equal to:
    \hfill{[April 2024]}
    \begin{enumerate}
        \item $\frac{\pi}{2} - \frac{1}{3}$
        \item $\pi - \frac{2}{3}$
        \item $\frac{\pi}{2} - \frac{2}{3}$
        \item $\pi - \frac{1}{3}$
    \end{enumerate}
    \item If the line segment joining the points $(5, 2)$ and $(2, a)$ subtends an angle $\frac{\pi}{4}$ at the origin, then the absolute value of the product of all possible values of $a$ is:
    \hfill{[April 2024]}
    \begin{enumerate}
        \item $6$
        \item $8$
        \item $2$
        \item $4$
    \end{enumerate}
    \item Let $\overrightarrow{a} = 4\hat{i} - \hat{j} + \hat{k}$, $\overrightarrow{b} = 11\hat{i} - \hat{j} + \hat{k}$, and $\overrightarrow{c}$ be a vector such that $(\overrightarrow{a} + \overrightarrow{b}) \times \overrightarrow{c} = \overrightarrow{c} \times (2\overrightarrow{a} + 3\overrightarrow{b})$. If $2\overrightarrow{a} + 3\overrightarrow{b} \cdot \overrightarrow{c} = 1670$, then $2|\overrightarrow{c}|^2$ is equal to:
    \hfill{[April 2024]}
    \begin{enumerate}
        \item $1627$
        \item $1618$
        \item $1600$
        \item $1609$
    \end{enumerate}
    \item If the function $f(x) = 2x^3 - 9ax^2 + 12a^2x + 1$ has a local maximum at $x = \alpha$ and a local minimum at $x = \alpha^2$, then $\alpha$ and $\alpha^2$ are the roots of the equation:
    \hfill{[April 2024]}
    \begin{enumerate}
        \item $x^2 - 6x + 8 = 0$
        \item $8x^2 + 6x - 8 = 0$
        \item $8x^2 - 6x + 1 = 0$
        \item $x^2 + 6x + 8 = 0$
    \end{enumerate}
    \item There are three bags X, Y, and Z. Bag X contains 5 one-rupee coins and 4 five-rupee coins; Bag Y contains 4 one-rupee coins and 5 five-rupee coins, and Bag Z contains 3 one-rupee coins and 6 five-rupee coins. A bag is selected at random, and a coin drawn from it at random is found to be a one-rupee coin. Then the probability that it came from bag Y is:
    \hfill{[April 2024]}
    \begin{enumerate}
        \item $\frac{1}{3}$
        \item $\frac{1}{2}$
        \item $\frac{1}{4}$
        \item $\frac{5}{12}$
    \end{enumerate}

