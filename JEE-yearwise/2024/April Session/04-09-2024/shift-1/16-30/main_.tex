\iffalse
\title{April 2024}
\author{EE24Btech11058}
\section{mcq-single}
\fi

%\begin{enumerate}
    \item Let $\int \frac{2 - \tan x}{3+ \tan x} \,dx = \frac{1}{2}\brak{\alpha x + \log_e |\beta \sin x + \gamma \cos x |} + C,$ where $C$ is the constant of integration. Then $\alpha + \frac{\gamma}{\beta}$ is equal to :
    \hfill(April 2024)
    \begin{enumerate}
        \item $1$
        \item $7$
        \item $4$
        \item $3$\\
    \end{enumerate}


    \item A ray of light coming from the point $\vec{P}\brak{1,2}$ gets reflected from $\vec{Q}$ on the $x-axis$ and then passes through the point $\vec{R}\brak{4,3}.$ If the point $\vec{S}\brak{h,k}$ is such that $PQRS$ is a parallelogram, then $hk^2$ is equal to :
    \hfill(April 2024)
    \begin{enumerate}
        \item $80$
        \item $70$
        \item $60$
        \item $90$\\
    \end{enumerate}

    \item $\overrightarrow{OA} = 2 \overrightarrow{a},\overrightarrow{OB} = 6\overrightarrow{a}+ 5\overrightarrow{b}$ and $\overrightarrow{OC} = 3\overrightarrow{b},$ where $O$ is the origin. If the area of the parallelogram with adjacent sides $\overrightarrow{OA}$ and $\overrightarrow{OC}$ is $15sq.units,$ then the area $\brak{in sq.units}$ of the quadrilateral $OABC$ is equal to :
    \hfill(April 2024)
    \begin{enumerate}
        \item $38$
        \item $32$
        \item $40$
        \item $35$\\
    \end{enumerate}

    \item Let $f\brak{x}=x^2+9, g\brak{x}= \frac{x}{x-9}$ and $a=(f \circ g)(10), b= (g \circ f)(3).$ If $e$ and $l$ denote the eccentricity and the length of the latus rectum of the ellipse $\frac{x^2}{a} + \frac{y^2}{b} = 1,$ then $8e^2 + l^2$ is equal to :
    \hfill(April 2024)
    \begin{enumerate}
        \item $16$
        \item $12$
        \item $8$
        \item $6$ \\
    \end{enumerate}


    \item The parabola $y^2=4x$ divides the area of the circle $x^2+y^2=5$ in two parts. The area of the smaller part is equal to :
    \hfill(April 2024)
    \begin{enumerate}
        \item $\frac{1}{3} + \sqrt{5} \sin^{-1} \brak{\frac{2}{\sqrt{5}}}$
        \item $\frac{2}{3} + \sqrt{5} \sin^{-1} \brak{\frac{2}{\sqrt{5}}}$
        \item $\frac{1}{3} + 5 \sin^{-1} \brak{\frac{2}{\sqrt{5}}}$
        \item $\frac{2}{3} + 5 \sin^{-1} \brak{\frac{2}{\sqrt{5}}}$\\
    \end{enumerate}
%\end{enumerate}

  


