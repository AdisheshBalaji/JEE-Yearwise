\iffalse
\title{Assignment}
\author{AI24BTECH11020}
\section{mcq-single}
\fi

%\begin{enumerate}

	\item If the sum of the series $\frac{1}{1.\brak{1+d}}+\frac{1}{\brak{1+d}\brak{1+2d}}+\cdots+\frac{1}{\brak{1+9d}\brak{1+10d}}$ is equal to 5,then 50d is equal to:   \hfill\brak{Apr 2024}
     \begin{enumerate}
     \item $5$ \item $10$ \item $20$ \item $15$
     \end{enumerate}
\item The solution of the differential equation $\brak{x^2+y^2}$d$x$-$5xy$d$y=0$, $y\brak{1}=0,$ is :\hfill\brak{Apr 2024}
      \begin{enumerate}
      \item $\abs{x^2-2y^2}^6=x$
      \item $\abs{x^2-2y^2}^5=x^2$
      \item $\abs{x^2-4y^2}^6=x$
      \item $\abs{x^2-4y^2}^5=x^2$
      \end{enumerate}
\item A variable line $L$ passes through the point $\brak{3,5}$ and intersects the positive coordinate axes at the points $A$ and $B$. The minimum area of the triangle $OAB$, where $O$ is the origin, is: \hfill\brak{Apr 2024}
     \begin{enumerate}
     \item $25$ \item $40$ \item $35$ \item $30$
     \end{enumerate}
\item Let $\alpha ,\beta $ be the roots of the equation $x^2+2\sqrt{2}x-1=0.$ The quadratic equation, whose roots are $\alpha^4+\beta^4$ and $\frac{1}{10}\brak{\alpha^6+\beta^6},$ is : \hfill\brak{Apr 2024}
     \begin{enumerate}
     \item $x^2-180x+9506=0$
     \item $x^2-195x+9506=0$
     \item $x^2-195x+9466=0$
     \item $x^2-190x+9466=0$
     \end{enumerate}
\item Let $f\brak{x}=ax^3+bx^2+cx+41$ be such that $f\brak{1}=40, f'\brak{1}=2 and f''\brak{1}=4.$ Then $a^2+b^2+c^2$ is equal to : \hfill\brak{Apr 2024}
     \begin{enumerate}
     \item $73$ \item $54$ \item $51$ \item $62$
     \end{enumerate}
\item If the domain of the function $f\brak{x}=\sin ^{-1} \brak{\frac{x-1}{2x+3}}$ is $\mathbb{R}-\brak{\alpha,\beta}.$ Then $12\alpha \beta$ is equal to: \hfill\brak{Apr 2024}
     \begin{enumerate}
     \item $36$ \item $32$ \item $24$ \item $40$
     \end{enumerate}
\item Let three vectors $\overrightarrow{a}=\alpha \hat{i}+4\hat{j}+2\hat{k},\overrightarrow{b}=5\hat{i}+3\hat{j}+4\hat{k},\overrightarrow{c}=x\hat{i}+y\hat{j}+z\hat{k}$ form a triangle such that $\overrightarrow{c}=\overrightarrow{a}-\overrightarrow{b}$ and the area of the triangle is $5\sqrt{6}$. If $\alpha$ is a positive real number, then $\abs{\overrightarrow{c}}^2$ is equal to : \hfill\brak{Apr 2024}
     \begin{enumerate}
     \item $16$ \item $10$ \item $14$ \item $12$
     \end{enumerate}
\item Let $\abs{\cos \theta \cos \brak{60^{\circ} -\theta} \cos \brak{60^{\circ} + \theta}} \leq \frac{1}{8}, \theta \in \sbrak{0,2\pi}$. Find the sum of  all $\theta \in \sbrak{0,2\pi}$, where $\cos 3 \theta $ attains its maximum value,is : \hfill\brak{Apr 2024}
     \begin{enumerate}
     \item $18\pi$ \item $9\pi$ \item $6\pi$ \item $15\pi$
     \end{enumerate}
\item The coefficient of $x^{70}$ in $x^2\brak{1+x}^{98}+x^3\brak{1+x}^{97}+x^4\brak{1+x}^{96}+\cdots+x^{54}\brak{1+x}^{46}$ is $\binom{99}{p}-\binom{46}{q}$. Then a possible value of $p + q$ is: \hfill\brak{Apr 2024}
     \begin{enumerate}
     \item $61$ \item $55$ \item $83$ \item $68$
     \end{enumerate}
\item The shortest distance between the lines $\frac{x-3}{4}=\frac{y+7}{-11}=\frac{z-1}{5}$ and $\frac{x-5}{3}=\frac{y-9}{-6}=\frac{z+2}{1}$ is : \hfill\brak{Apr 2024}
     \begin{enumerate}
     \item $\frac{185}{\sqrt{563}}$ 
     \item $\frac{178}{\sqrt{563}}$
     \item $\frac{179}{\sqrt{563}}$
     \item $\frac{187}{\sqrt{563}}$
     \end{enumerate}
\item Let a circle passing through $\brak{2,0}$ have its centre at the point \brak{h,k}. Let $\brak{x_c,y_c}$ be the point of intersection of the lines $3x+5y=1$ and $\brak{2+c}x+5c^2y=1.$ If $\displaystyle \lim_{c \to 1}x_c$ and $\displaystyle \lim_{c \to 1}y_c$, then the equation of the circle is : \hfill\brak{Apr 2024}
     \begin{enumerate}
     \item $25x^2+25y^2-2x+2y-60=0$
     \item $5x^2+5y^2-4x+2y-12=0$
     \item $25x^2+25y^2-20x+2y-60=0$
     \item $5x^2+5y^2-4x-2y-12=0$
     \end{enumerate}
\item Let $\lambda,\mu \in \mathbb{R}.$ If the system of equations\\
	$3x+5y+\lambda z=3$\\
	$7x+11y-9z=2$\\
	$97x+155y+189z=\mu $\\
	has infinitely many solutions, then $\mu +2\lambda$ is equal to : \hfill\brak{Apr 2024}
     \begin{enumerate}
     \item $25$ \item $22$ \item $24$ \item $27$
     \end{enumerate}
\item Let the line $L$ intersect the lines $x-2=-y=z-1,2\brak{x+1}=2\brak{y-1}=z+1$ and be parallel to the line $\frac{x-2}{3}=\frac{y-1}{1}=\frac{z-2}{2}$. Then which of the following points lies on $L$? \hfill\brak{Apr 2024}
     \begin{enumerate}
	\item $\brak{\frac{-1}{3},-1,-1}$
	\item $\brak{\frac{-1}{3},1,-1}$
	\item $\brak{\frac{-1}{3},1,1}$ 
	\item $\brak{\frac{-1}{3},-1,1}$ 
     \end{enumerate}
\item The frequency distribution of the age of students in a class of $40$ students is given below.
	\begin{table}[h]
\centering
\begin{tabular}{|c|c|c|c|c|c|c|}
\hline
Age           & 15  & 16  & 17  & 18  & 19  & 20  \\ \hline
No of Students & 5   & 8   & 5   & 12  & $x$ & $y$ \\ \hline
\end{tabular}
\end{table}
If the mean deviation about the median is $1.25$, then $4x+5y$ is equal to: \hfill\brak{Apr 2024}
     \begin{enumerate}
     \item $47$ \item $43$ \item $46$ \item $44$
     \end{enumerate}
\item The solution curve, of the differential equation $2y \frac{dy}{dx} +3=5\frac{dy}{dx}$, passing through the point $\brak{0,1}$ is a conic, whose vertex lies on the line : \hfill\brak{Apr 2024}
	\begin{enumerate}
     \item $2x+3y=9$ \item $2x+3y=6$ \item $2x+3y=-6$ \item $2x+3y=-9$ 
     \end{enumerate}

%\end{enumerate}
 

 
