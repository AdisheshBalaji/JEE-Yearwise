\iffalse
\title{04-06-2024-shift-1-16-30}
\author{AI24BTECH11011}
\section{mcq-single}
\fi
    \item A company has two plants A and B to manufacture motorcycles. $60\%$ motorcycles are manufactured at plant A and the remaining are manufactured at plant B. $80\%$ of the motorcycles manufactured at plant A are rated of the standard quality, while $90\%$ of the motorcycles manufactured at plant B are rated of the standard quality. A motorcycle picked up randomly from the total production is found to be of the standard quality. If $p$ is the probability that it was manufactured at plant B, then 126p is
  \hfill{(April 2024)}
\begin{enumerate}
        \item 64
        \item 66
        \item 56
        \item 54
    \end{enumerate}
    \item For $\alpha,\beta \in \mathbb{R}$ and a natural number $n$,let $A_r=\myvec{r & 1 & \frac{n^2}{2}+\alpha \\ 2r & 2 & n^2-\beta \\ 3r-2 & 3 &\frac{n\brak{3n-1}}{2}}$ . Then $2A_{10}-A_8$ is
\hfill{(April 2024)} 
\begin{enumerate}
        \item $4\alpha+2\beta$
        \item 0
        \item $2\alpha+4\beta$
        \item $2n$
    \end{enumerate}
    \item Let the relations $\mathbb{R}_1$ and $\mathbb{R}_2$ on the set $X=\{1,2,3,\cdots,20\}$ be given by $\mathbb{R}_1=\cbrak{\brak{x,y}:2x-3y=2}$ and $\mathbb{R}_2=\cbrak{\brak{x,y}:-5x+4y=0}$.If M nd N be the minimum number of elements required to be added in $\mathbb{R}_1$ and $\mathbb{R}_2$ respectively, in order to make the relations symmetric, then $M+N$ equals
\hfill{(April 2024)} 
\begin{enumerate}
        \item 10
        \item 8
        \item 12
        \item 16
    \end{enumerate}
    \item Let $A=\cbrak{n \in [100,700]\cap N :\text{n is neither a multiple of 3 nor a multiple of 4}}$. Then the number of elements in A is 
\hfill{(April 2024)} 
\begin{enumerate}
        \item 290
        \item 300
        \item 280
        \item 310
    \end{enumerate}

    \item A circle is inscribed in an equilateral triangle of side of length 12. If the area and perimeter of any square inscribed in this circle are $m$ and $n$, respectively, then $m+n^2$ is equal to 
\hfill{(April 2024)} 
\begin{enumerate}
        \item 408
        \item 396
        \item 312
        \item 414
    \end{enumerate}
    
 
