\iffalse
\title{2024}
\author{ee24btech11009}
\section{integer}
\fi
\item If the shortest distance between the lines $\frac{x-\lambda}{3}=\frac{y-2}{-1}=\frac{z-1}{1}$ and $\frac{x+2}{2\-3}=\frac{y+5}{2}=\frac{z-4}{4}$ is $\frac{44}{\sqrt{30}}$, then the largest possible value of $\abs{\lambda}$ is equal to:\hfill{[April 2024]}
\item Let $\sbrak{t}$ denote the largest integer less than or equal to $t$. If $\int_{0}^{3}\brak{\sbrak{x^{2}}+\sbrak{\frac{x^{2}}{2}}}dx=a+b\sqrt{2}-\sqrt{3}-\sqrt{5}+c\sqrt{6}-\sqrt{7}$, where $a, b, c\in\mathbb{Z}$, then $a+b+c$ is equal to:\hfill{[April 2024]}

\item Let $\alpha, \beta$ be roots of $x^{2}+\sqrt{2}x-8=0$. If $U_{n}=\alpha^{n}+\beta^{n}$, then $\frac{U_{10}+\sqrt{2}U_{9}}{2U_{8}}$ is equal to:\hfill{[April 2024]}

\item In a triangle $ABC$, $BC=7$, $AC=8$, $AB=\alpha\in\mathbb{N}$ and $\cos A=\frac{2}{3}$. If $49\cos\brak{3C}+42=\frac{m}{n}$, where $\gcd(m,n)=1$, then $m+n$ is equal to:\hfill{[April 2024]}

\item The length of the latus rectum and directrices of a hyperbola with eccentricity $e$ are $9$ and $x=\pm\frac{4}{\sqrt{3}}$, respectively. Let the line $y-\sqrt{3}x+\sqrt{3}=0$ touch this hyperbola at $\brak{x_{0},y_{0}}$. If $m$ is the product of the focal distances of the point $\brak{x_{0},y_{0}}$, then $4e^{2}+m$ is equal to:\hfill{[April 2024]}
\item If $S\brak{x}=\brak{1+x}+2\brak{1+x}^{2}+3\brak{1+x}^{3}+\cdots+60\brak{1+x}^{60}$ and $\brak{60}^2 S\brak{60}=a\brak{b}^{b}+b$, where $a, b\in\mathbb{N}$, then $a+b$ is equal to:\hfill{[April 2024]}
\item If the system of equations
\[2x+7y+\lambda z=3\]
\[3x+2y+5z=4\]
\[x+\mu y+32z=-1\]
has infinitely many solutions, then $\brak{\lambda-\mu}$ is equal to:\hfill{[April 2024]}
\item Let $\sbrak{t}$ denote the greatest integer less than or equal to $t$. Let $f:[0,\infty)\rightarrow \mathbb{R}$ be a function defined by
$f(x)=\sbrak{\frac{x}{2}+3}-\sbrak{\sqrt{x}}$
Let $S$ be the set of all points in the interval $\sbrak{0, 8}$ at which $f$ is not continuous. Then $\displaystyle \sum_{a\in S}a$ is equal to:\hfill{[April 2024]}
\item If the solution $y(x)$ of the given differential equation $\brak{e^{y}+1}\cos x dx+e^{y}\sin x dy=0$ passes through the point $\brak{\frac{\pi}{2},0}$, then the value of $e^{y\brak{\frac{\pi}{6}}}$ is equal to:\hfill{[April 2024]}
\item From a lot of $12$ items containing $3$ defectives, a sample of $5$ items is drawn at random. Let the random variable $X$ denote the number of defective items in the sample. Let items in the sample be drawn one by one without replacement. If variance of $X$ is $\frac{m}{n}$, where $\gcd(m,n)=1$, then $n-m$ is equal to:\hfill{[April 2024]}
