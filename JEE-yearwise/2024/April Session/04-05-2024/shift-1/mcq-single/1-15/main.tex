\iffalse
    \title{2024}
    \author{EE24BTECH11001}
    \section{mcq-single}
\fi

\item 
	    Let a circle C of radius 1 and closer to the origin be such that the lines
        passing through the point $\brak{3, 2}$ and parallel to the coordinate axes touch it.
        Then the shortest distance of the circle from the point $\brak{5, 5}$ is :
		\hfill{\brak{2024-Apr}}
	\begin{multicols}{4}
		\begin{enumerate}
			\item 5
			\columnbreak
        \item $4\sqrt{2}$
			\columnbreak
			\item 4
			\columnbreak
        \item $2\sqrt{2}$
		\end{enumerate}
	\end{multicols}

	\item
        Let a rectangle $ABCD$ of sides 2 and 4 be inscribed in another rectangle $PQRS$
        such that the vertices of the rectangle $ABCD$ lie on the sides og the rectangle $PQRS$.
        Let $a$ and $b$ be the sides of the rectangle $PQRS$ when its area is maximum. Then
        $\brak{a + b}^2$ is equal to :
		\begin{multicols}{4}
		\begin{enumerate}
			\item 80 \columnbreak
			\item 60 \columnbreak
			\item 72 \columnbreak
			\item 64
		\end{enumerate}
	\end{multicols}


\item If
	\begin{align}
        \frac{1}{\sqrt{1} + \sqrt{2}} + \frac{1}{\sqrt{2} + \sqrt{3}} + \dots \frac{1}{\sqrt{99} + \sqrt{100}} = m
	\end{align} and
    \begin{align}
        \frac{1}{1 . 2} + \frac{1}{2 . 3} + \dots \frac{1}{99 . 100} = n
    \end{align} then the point $\brak{m, n}$ lies on the line
		\hfill{\brak{2024-Apr}}
		\begin{enumerate}
            \item $11\brak{x-1} - 100\brak{y-2}$ 
            \item $11\brak{x-2} - 100\brak{y-1}$ 
            \item $11\brak{x-1} - 100y$ 
            \item $11x - 100y$ 
		\end{enumerate}
		
	\item Let $d$ be this distance of the point of intersection of the lines
        \begin{align}
            \frac{x-6}{3} = \frac{y}{2} = \frac{z+1}{1} 
        \end{align} and 
        \begin{align}
            \frac{x-7}{4} = \frac{y-9}{3} = \frac{z-4}{2}
        \end{align} from the point $\brak{7, 8, 9}$. Then $d^2 + 6$ is equal to :
        \hfill{\brak{2024-Apr}}
		\begin{enumerate}
			\begin{multicols}{4}
				\item 72 \columnbreak
				\item 78 \columnbreak
				\item 69 \columnbreak
				\item 75
			\end{multicols}
		\end{enumerate}

	\item Let the line $2x + 3y - k = 0, k > 0$, intersect the x-axis and y-axis at the points 
        $A$ and $B$, respectively. If the equation of the circle having the line segment $AB$ as
        a diameter is $x^2 + 9y^2 = k^2$ is $\frac{m}{n}$, where $m$ and $n$ are coprime, then
        $2m + n$ is equal to :
		\hfill{\brak{2024-Apr}}
		\begin{enumerate}
                \begin{multicols}{4}
			\item 11 \columnbreak 
			\item 10 \columnbreak
			\item 13 \columnbreak
			\item 12  
                \end{multicols}
		\end{enumerate}
	\item
        The coefficients $a, b, c$ in the quadratic equation $ax^2 + bx +c = 0$ are chosen from the 
        set $\{ 1, 2 , 3, 4, 5, 6, 7, 8\}$. The probability of this equation having repeated roots is :
		\hfill{\brak{2024-Apr}}
		\begin{multicols}{4}
		\begin{enumerate}
            \item $\frac{3}{128}$ \columnbreak
            \item $\frac{1}{64}$ \columnbreak
            \item $\frac{1}{128}$ \columnbreak
            \item $\frac{3}{256}$
		\end{enumerate}
	\end{multicols}
\item Suppose $\theta \in \sbrak{0, \frac{\pi}{4}}$ is a solution of $4\cos \theta - 3 \sin \theta = 1$. 
    Then $\cos \theta$ is equal to :
		\hfill{\brak{2024-Apr}}
		\begin{enumerate}
            \item $\frac{4}{\brak{3\sqrt{6} - 2}}$ 
            \item $\frac{6 - \sqrt{6}}{\brak{3\sqrt{6} - 2}}$ 
            \item $\frac{4}{\brak{3\sqrt{6} + 2}}$ 
            \item $\frac{6 + \sqrt{6}}{\brak{3\sqrt{6} + 2}}$ 
		\end{enumerate}
\item
    For the function
	\begin{align}
        f\brak{x} = \sin x + 3x - \frac{2}{\pi}\brak{x^2 + x}, \textnormal{where} x \in \sbrak{0, \frac{\pi}{2}}
	\end{align} 
    Consider the follwing two statements,
    \begin{enumerate}
        \item[1.] $f$ is increasing in $\brak{0, \frac{\pi}{2}}$
        \item[2.] $f'$ is decreasing in $\brak{0, \frac{\pi}{2}}$
    \end{enumerate}
		\hfill{\brak{2024-Apr}}
		\begin{enumerate}
			\item Only 2 is true.
			\item  neither 1 nor 2 is true.
			\item  both 1 and 2 are true.
			\item only 1 is true.
		\end{enumerate}
    \item Let $f\brak{x} = x^5 + 2x^3 + 3x + 1, x \in \mathbb{R},$ and $g\brak{x}$ be a function 
        such that $g\brak{f\brak{x}} = x$ for all $x \in \mathbb{R}$. Then $\frac{g\brak{7}}{g'\brak{7}}$
        is equal to:
		\hfill{\brak{2024-Apr}}
        \begin{multicols}{4}
            
		\begin{enumerate}
			\item 7 \columnbreak
			\item 42 \columnbreak 
			\item 14 \columnbreak
			\item 1
		\end{enumerate}
        \end{multicols}
\item If the system of equations 
		\begin{align}
            11x + y + \lambda z = -5 \\
            2x + 3y + 5z = 3 \\
            8x - 19y - 39z = \mu
		\end{align}, has infinitely many solutions, then $\lambda ^ 4 - \mu$ is equal to :
        \hfill{\brak{2024-Apr}}
		\begin{multicols}{4}
		\begin{enumerate}
			\item 45 \columnbreak
			\item 51 \columnbreak 
			\item 47 \columnbreak
			\item 49
		\end{enumerate}
        \end{multicols}
		\
\item The value of 
    \begin{align}
        \int_{-\pi} ^{\pi} \frac{2y\brak{1 + \sin y}}{1 + \cos ^2 y} \, dy 
    \end{align} is :
		\hfill{\brak{2024-Apr}}
		
	\begin{multicols}{4}
		\begin{enumerate}
            \item $\frac{\pi}{2}$ \columnbreak
            \item $\frac{\pi ^ 2}{2}$ \columnbreak
            \item $\pi ^ 2$ \columnbreak
			\item $2\pi ^2 $
		\end{enumerate}
	\end{multicols}
\item If the line $\frac{2-x}{3} = \frac{3y-2}{4\lambda} = 4-z$ makes right angle with 
    the line $\frac{x+3}{3\mu} = \frac{1-2y}{6} = \frac{5-z}{7}$ the value of $4\lambda + 9\mu$ is :
		\hfill{\brak{2024-Apr}}
		\begin{multicols}{4}
		\begin{enumerate}
			\item 13 \columnbreak
			\item 5 \columnbreak 
			\item 4 \columnbreak
			\item 6
		\end{enumerate}
        \end{multicols}
    \item If $A\brak{1, -1, 2}, B\brak{5, 7, -6}, C\brak{3, 4, -10}$ and $D\brak{-1, -4, -2}$ 
        are the vertices of a quadrilateral $ABCD$, then its area is :
		\hfill{\brak{2024-Apr}}
	\begin{multicols}{4}
		\begin{enumerate}
            \item $12\sqrt{29}$ \columnbreak
            \item $24\sqrt{29}$ \columnbreak
            \item $48\sqrt{7}$ \columnbreak
            \item $24\sqrt{7}$
		\end{enumerate}
	\end{multicols}
\item Let $A$ and $B$ be the two square matrices of order 3 such that $\abs{A} = 3$ 
    and $\abs{B} = 8$. Then\\ 
    $\abs{A^{\top}A\brak{\textnormal{adj}\brak{2A}}^{-1}\brak{\textnormal{adj}\brak{4b}\brak{\textnormal{adj}\brak{AB}^{-1}AA^{\top}}}}$
    is equal to :
		\hfill{\brak{2024-Apr}}
	\begin{multicols}{4}
		\begin{enumerate}
            \item $64$ \columnbreak
            \item $81$ \columnbreak
            \item $108$ \columnbreak
            \item $32$
		\end{enumerate}
	\end{multicols}
\item Let $A = \{ 1, 3, 5, 7, 9\}$ and $B = \{ 2, 4, 5, 7, 8, 10, 12 \}$. Then the total
    number of one-one maps $f : A \rightarrow B$, such that $f\brak{1} + f\brak{3} = 14$ is :
		\hfill{\brak{2024-Apr}}
	\begin{multicols}{4}
		\begin{enumerate}
			\item $120$ \columnbreak
			\item $180$ \columnbreak
			\item $480$ \columnbreak
			\item $240$
		\end{enumerate}
	\end{multicols}
