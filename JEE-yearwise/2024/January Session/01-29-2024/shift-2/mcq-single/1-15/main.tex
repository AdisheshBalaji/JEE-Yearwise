\iffalse
    \title{2024}
    \author{EE24BTECH11011}
    \section{mcq-single}
\fi 
\item Let $\vec{OA} = \vec{a} , \vec{OB} = 12\vec{a} + 4\vec{b} \text{ and } \vec{OC} = \vec{b}$ , where $\vec{O}$ is the origin.If $S$ is the parallelogram with adjacent sides $\vec{OA}$ and $\vec{OC}$ , then $\frac{\text{area of the quadrilateral }OABC}{\text{area of }S}$ is equal to \hfill[2024-Jan]
\begin{multicols}{2}
\begin{enumerate}
\item $8$
\item $7$
\item $6$
\item $10$
\end{enumerate}
\end{multicols}
\item Let a unit vector $\vec{u} = x\hat{i}+y\hat{j}+z\hat{k}$ makes angles $\frac{\pi}{2} \frac{\pi}{3} \text{ and } \frac{2\pi}{3}$ with the vectors $\frac{1}{\sqrt{2}}\hat{i}+\frac{1}{\sqrt{2}}\hat{k} , \frac{1}{\sqrt{2}}\hat{j}+\frac{1}{\sqrt{2}}\hat{k} \text{ and } \frac{1}{\sqrt{2}}\hat{i}+\frac{1}{\sqrt{2}}\hat{j}$ respectively.If $\vec{v} = \frac{1}{\sqrt{2}}\hat{i}+\frac{1}{\sqrt{2}}\hat{j}+\frac{1}{\sqrt{2}}\hat{k}$ then $\abs{\vec{u} - \vec{v}}^2 $ is equal to \hfill[2024-Jan]
\begin{multicols}{2}
\begin{enumerate}
\item $9$\\
\item $\frac{5}{2}$
\item $7$\\
\item $\frac{11}{2}$
\end{enumerate}
\end{multicols}
\item The function $f\brak{x} = 2x + 3\brak{x}^{\frac{2}{3}} , x \in \mathbb{R} $ has \hfill[2024-Jan]
	\begin{enumerate}
		\item exactly one point of local minima and no point of local laxima
		\item exactly one point of local maxima and exactly one point of local minima
		\item exactly one point of local maxima and no point of local minima
		\item exactly two points of local maxima and exactly one point of local minima\\
	\end{enumerate}
\item If each term of a geometric peogression $a_1 , a_2 , a_3 ,\dots $ with $a_1 = \frac{1}{8}$ and $a_2 \neq a_1 $ is the arithmetic mean of the next two terms and $S_n = a_1 + a_2 + \dots + a_n$ then $S_{20} - S_{18}$ is equal to \hfill[2024-Jan]
	\begin{multicols}{2}
		\begin{enumerate}
			\item $2^{18}$
			\item $-2^{18}$
			\item $2^{15}$
			\item $-2^{15}$
		\end{enumerate}
	\end{multicols}
	\item If $\log_e a , \log_e b , log_e c $ are in A.P and $log_e{a} - \log_e{2b} , \log_e{2b} - \log_e{3c} , \log_e{3c} - \log_e{a}$ are also in an A.P then $a \colon b \colon c$ is equal to \hfill[2024-Jan]
	\begin{enumerate}
		\item $16 \colon 4 \colon 1$
		\item $6 \colon 3 \colon 2$
		\item $25 \colon 10 \colon 4$
		\item $9 \colon 6 \colon 4$\\
	\end{enumerate}
\item Let $\vec{A} \brak{ 3 , 2 ,3} , \vec{Q}\brak{4 , 6 ,2} \text{ and } \vec{R}\brak{ 7 , 3 ,2}$ be the vertices of $\triangle {PQR}$.Then the angle $\angle{QPR}$ is \hfill[2024-Jan]
	\begin{enumerate}
		\item $\cos ^{-1}\brak{\frac{1}{18}}$
		\item $\frac{\pi}{3}$
		\item $\frac{\pi}{6}$
		\item $\cos^{-1}\brak{\frac{7}{18}}$
	\end{enumerate}
\item Number of ways of arranging $8$ identical books into $4$ identical shelves where any number of shelves may remain empty is equal to \hfill[2024-Jan]
	\begin{multicols}{2}
		\begin{enumerate}
			\item $16$
			\item $18$
			\item $15$
			\item $12$
		\end{enumerate}
	\end{multicols}
\item The distance of the point $\brak{ 2 , 3 }$ from the line $2x - 3y + 28 =0 $ measured parallel to the line $\sqrt{3}x - y + 1 = 0$ is equal to \hfill[2024-Jan]
	\begin{multicols}{2}
		\begin{enumerate}
			\item $ 4 + 6\sqrt{3}$
			\item $ 3 + 4\sqrt{2}$
			\item $6\sqrt{3}$
			\item $4\sqrt{2}$
		\end{enumerate}
	\end{multicols}
\item Let $ y = log_e \brak{\frac{1 - x^2}{1 + x^2}} , -1 < x < 1$ . Then at $x = \frac{1}{2}$, then the value off $225\brak{y^\prime - y^{\prime\prime}}$ is equal to \hfill[2024-Jan]
	\begin{multicols}{2}
		\begin{enumerate}
			\item $736$
			\item $746$
			\item $732$
			\item $742$
		\end{enumerate}
	\end{multicols}
	\item If 
	\begin{align}
		\int \frac{ \sin^{\frac{3}{2}}x + \cos^{\frac{3}{2}}x}{\sqrt{ \sin^3x \cos^3 x \sin \brak{x - \theta}}} dx = A \sqrt{\cos\theta \tan x - \sin \theta} + B \sqrt{ \cos \theta - \sin \theta \cot x} + C 
	\end{align}
	where $C$ is the integration constant, then $AB$ is equal to \hfill[2024-Jan]
	\begin{multicols}{2}
		\begin{enumerate}
			\item $4 \cosec\brak{2\theta}$
			\item $4 \sec{\theta}$
			\item $2 \sec{\theta}$
			\item $8 \cosec\brak{2\theta}$
		\end{enumerate}
	\end{multicols}
\item If $\mathbf{R}$ is the smallest equivalence relation on the set $\cbrak{ 1,2 ,3 ,4}$ such that $\cbrak{\brak{1,2},\brak{1,3}} \subset \mathbf{R}$, then the number of elements in $\mathbf{R}$ is \hfill[2024-Jan]
	\begin{multicols}{2}
		\begin{enumerate}
			\item $12$
			\item $15$\\
			\item $8$
			\item $10$
		\end{enumerate}
	\end{multicols}
\item Let $x = \frac{m}{n} \brak{ m , n \text{ are co-prime natural numbers}}$ be a solution of the equation $\cos \brak{2\sin^{-1}x}=\frac{1}{9}$ and let $\alpha,\beta \brak{\alpha > \beta}$ be the roots of the equation $mx^2 - nx - m + n=0$.Then the point $\brak{\alpha,\beta}$ lies on the line \hfill[2024-Jan]
	\begin{enumerate}
		\item $5x + 8y = 9$
		\item $3x - 3y = -2$
		\item $5x - 8y = -9$
		\item $3x + 2y = 2$
	\end{enumerate}
\item The sum of the solutions $x \in \mathbb{R}$ of the equation 
	\begin{align}
		\frac{3\cos{2x}+\cos^3{2x}}{\cos^{6x}-\sin^3{6x}} = x^3 - x^2 + 6
	\end{align}
	is \hfill[2024-Jan]
	\begin{multicols}{2}
		\begin{enumerate}
			\item $-1$
			\item $1$
			\item $3$
			\item $0$
		\end{enumerate}
	\end{multicols}
	\item Let $\vec{A} = \myvec{ 2 & 1 & 2 \\ 6 & 2 & 11 \\ 3 & 3 & 2} \text{ and } \vec{P} = \myvec{ 1 & 2 & 0 \\ 5 & 0 & 2 \\ 7 & 1 & 5} $. The sum of the prime factors of $\abs{\vec{P}^{-1}\vec{AP}-2\vec{I}}$ is equal to \hfill[2024-Jan]
	\begin{multicols}{2}
		\begin{enumerate}
			\item $26$
			\item $66$
			\item $23$
			\item $27$
		\end{enumerate}
	\end{multicols}
\item An integer is chosen at random from the integers $ 1 , 2 , 3 ,\dots , 50$. The probability that the chosen integer is a multiple of atleast one of $ 4 , 6 \text{ and } 7$ is \hfill[2024-Jan]
	\begin{enumerate}
		\item $\frac{9}{50}$\\
		\item $\frac{8}{25}$\\
		\item $\frac{21}{50}$\\
		\item $\frac{14}{25}$
	\end{enumerate}
