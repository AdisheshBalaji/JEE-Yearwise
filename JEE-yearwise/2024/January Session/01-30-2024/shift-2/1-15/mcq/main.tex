\iffalse
   \title{Assignment}
   \author{EE24BTECH11034}
   \section{mcq-single}
\fi 

    \item Consider the system of linear equations
    $x + y + z = 5$, $x + 2y + \lambda^{2}z = 9$,
    $x + 3y + \lambda z = \mu$ where $\lambda, \mu \in \mathbb{R}$. Then, which of
    the following statement is NOT correct?\hfill{Jan 2024}
        
        \begin{enumerate}
        \item System has infinite number of solutions if $\lambda = 1$
        and $\mu = 13$
        \item System is inconsistent if $\lambda = 1$ and $\mu \ne 13$
        \item System is consistent if $\lambda \ne 1$ and $\mu = 13$
        \item System has unique solution if $\lambda \ne 1$ and $\mu \ne 13$
        \end{enumerate}

    \item For $\alpha,\beta\in\sbrak{0,\frac{\pi}{2}}$, let $3\sin\brak{\alpha+\beta}=2\sin\brak{\alpha-\beta}$ and a real number $k$ be such that $\tan\alpha=k\tan\beta$. Then the value of $k$ is equal to:\hfill{Jan 2024}

        \begin{multicols}{4}
        \begin{enumerate}
        \item $-\frac{2}{3}$
        \item $-5$
        \item $\frac{2}{3}$
        \item $5$
        \end{enumerate}
        \end{multicols}

    \item Let $A\brak{\alpha,0}$ and $B\brak{0,\beta}$ be the points on the line $5x+7y=50$. Let the point $P$ divide the line segment $AB$ internally in the ratio $7:3$. Let $3x-25=0$ be a directrix of the ellipse $E:\frac{x^{2}}{a^{2}}+\frac{y^{2}}{b^{2}}=1$ and the corresponding focus be $S$. If from $S$, the perpendicular on the x-axis passes through $P$, then the length of the latus rectum of $E$ is equal to:\hfill{Jan 2024}

        \begin{multicols}{4}
        \begin{enumerate}
        \item $\frac{25}{3}$
        \item $\frac{32}{9}$
        \item $\frac{25}{9}$
        \item $\frac{32}{5}$
        \end{enumerate}
        \end{multicols}

    \item Let $\vec{a}=\hat{i}+\alpha\hat{j}+\beta\hat{k}$, $\alpha,\beta\in\mathbb{R}$. Let a vector $\vec{b}$ be such that the angle between $\vec{a}$ and $\vec{b}$ is $\frac{\pi}{4}$ and $\abs{\vec{b}}=6$. If $\vec{a}\cdot\vec{b}=3\sqrt{2}$, then the value of $\brak{\alpha^{2}+\beta^{2}}\abs{\vec{a}\times\vec{b}}$ is equal to:\hfill{Jan 2024}

        \begin{multicols}{4}
        \begin{enumerate}
        \item $90$
        \item $75$
        \item $95$
        \item $85$
        \end{enumerate}
        \end{multicols}
        
    \item Let $f\brak{x}=\brak{x+3}\brak{x-2}^{3}$, $x\in\sbrak{-4,4}$. If $M$ and $m$ are the maximum and minimum values of $f$, respectively in $\sbrak{-4,4}$, then the value of $M-m$ is:\hfill{Jan 2024}

        \begin{multicols}{4}
        \begin{enumerate}
        \item $600$
        \item $392$
        \item $608$
        \item $108$
        \end{enumerate}
        \end{multicols}

    \item Let $a$ and $b$ be two distinct positive real numbers. Let $11^{th}$ term of a GP, whose first term is $a$ and third term is $b$, is equal to $p^{th}$ term of another GP, whose first term is $a$ and fifth term is $b$. Then $p$ is equal to:\hfill{Jan 2024}

        \begin{multicols}{4}
        \begin{enumerate}
        \item $20$
        \item $25$
        \item $21$
        \item $24$
        \end{enumerate}
        \end{multicols}

    \item If $x^{2}-y^{2}+2hxy+2gx+2fy+c=0$ is the locus of a point, which moves such that it is always equidistant from the lines $x+2y+7=0$ and $2x-y+8=0$, then the value of $g+c+h-f$ equals:\hfill{Jan 2024}

        \begin{multicols}{4}
        \begin{enumerate}
        \item $14$
        \item $6$
        \item $8$
        \item $29$
        \end{enumerate}
        \end{multicols}

    \item Let $\vec{a}$ and $\vec{b}$ be two vectors such that $\abs{\vec{b}}=1$ and $\abs{\vec{b}\times\vec{a}}=2$. Then $\abs{\brak{\vec{b}\times\vec{a}}-\vec{b}}^{2}$ is equal to:\hfill{Jan 2024}

        \begin{multicols}{4}
        \begin{enumerate}
        \item $3$
        \item $5$
        \item $1$
        \item $4$
        \end{enumerate}
        \end{multicols}

    \item Let $y=f\brak{x}$ be a thrice differentiable function in $\brak{-5,5}$. Let the tangents to the curve $y=f\brak{x}$ at $\brak{1,f\brak{1}}$ and $\brak{3,f\brak{3}}$ make angles $\frac{\pi}{6}$ and $\frac{\pi}{4}$, respectively with positive x-axis. If $27\int_{1}^{3}\brak{\brak{f^{\prime}\brak{t}}^{2}+1}f^{\prime\prime}\brak{t}dt=\alpha+\beta\sqrt{3}$, where $\alpha,\beta$ are integers, then the value of $\alpha+\beta$ equals \hfill{Jan 2024}

        \begin{multicols}{4}
        \begin{enumerate}
        \item $-14$
        \item $26$
        \item $-16$
        \item $36$
        \end{enumerate}
        \end{multicols}


    \item Let $P$ be a point on the hyperbola $H:\frac{x^{2}}{9}-\frac{y^{2}}{4}=1$, in the first quadrant such that the area of triangle formed by $P$ and the two foci of H is $2\sqrt{13}$. Then, the square of the distance of $P$ from the origin is \hfill{Jan 2024}

        \begin{multicols}{4}
        \begin{enumerate}
        \item $18$
        \item $26$
        \item $22$
        \item $20$
        \end{enumerate}
        \end{multicols}

    \item Bag $A$ contains $3$ white, $7$ red balls and bag $B$ contains $3$ white, $2$ red balls. One bag is selected at random and a ball is drawn from it. The probability of drawing the ball from the bag $A$, if the ball drawn in white, is: \hfill{Jan 2024}

        \begin{multicols}{4}
        \begin{enumerate}
        \item $\frac{1}{4}$
        \item $\frac{1}{9}$
        \item $\frac{1}{3}$
        \item $\frac{3}{10}$
        \end{enumerate}
        \end{multicols}

    \item Let $f:\mathbb{R}\rightarrow\mathbb{R}$ be defined $f\brak{x}=ae^{2x}+be^{x}+cx$. If $f\brak{0}=-1$, $f^{\prime}\brak{0}=1$, and $f^{\prime\prime}\brak{0}=0$, then the value of $\frac{b}{a}$ is equal to: \hfill{Jan 2024}

        \begin{multicols}{4}
        \begin{enumerate}
        \item $-3$
        \item $2$
        \item $-2$
        \item $3$
        \end{enumerate}
        \end{multicols}


