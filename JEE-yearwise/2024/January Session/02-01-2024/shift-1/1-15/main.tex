\iffalse
\title{Assignment}
\author{K.AKSHAY TEJA}
\section{mcq-single}
\fi
% Question 1
\item If 3, a, b, x are in A.P, and 2, a-1, b+1 are in G.P. Then arithmetic mean of a, b and c is   \hfill \brak{Jan 2024}

\begin{multicols}{4}
\begin{enumerate}
    \item $11$
    \item $10$
    \item $9$
    \item $13$
\end{enumerate}
\end{multicols}

% Question 2
\item The value of $\int_0^{\frac{\pi}{4}} \frac{xdx}{\sin ^4\brak{2x} + \cos ^4\brak{2x}}$ is equal to  \hfill \brak{Jan 2024}

\begin{multicols}{4}
\begin{enumerate}
    \item $\frac{\pi ^2}{16\sqrt{2}}$
    \item $\frac{\pi ^2}{64}$
    \item $\frac{\pi ^2}{32}$
    \item $\frac{\pi ^2}{8\sqrt{2}}$
\end{enumerate}
\end{multicols}



% Question 3
\item If $A = \begin{bmatrix}\sqrt{2} & 1\\-1 & \sqrt{2}
\end{bmatrix}, B = \begin{bmatrix} 1 & 0\\1 & 1
\end{bmatrix}, C = ABA^T,$ then $\abs{X}$ is equal to  \hfill \brak{Jan 2024}
\begin{multicols}{4}
\begin{enumerate}
    \item $729$ 
    \item $283$ 
    \item $27$ 
    \item $23$ 
\end{enumerate}
\end{multicols}



% Question 4
\item If $3, 7, 11, \ldots, 403 = AP_1\,\, 2, 5, 8, \ldots, 401 = AP_2$ Find sum of common term of $AP_1$ and $AP_2$  \hfill \brak{Jan 2024}

\begin{multicols}{4}
    \begin{enumerate}
        \item 3366
        \item 6699
        \item 9999
        \item 6666
    \end{enumerate}
\end{multicols}

% Question 5
\item $\int_\frac{-\pi}{2}^\frac{\pi}{2} \frac{8\sqrt{2}\cos x}{\brak{1 + e^{\sin x}}\brak{1 + \sin^4x}}dx = a\pi + b\log\brak{3 + 2\sqrt{2}}$ then find a + b.  \hfill \brak{Jan 2024}

\begin{multicols}{4}
\begin{enumerate}
    \item $4$
    \item $6$
    \item $8$
    \item $2$
\end{enumerate}
\end{multicols}


% Question 6
\item If $\brak{t+ 1}\,dx = \brak{2x + \brak{t+ 1}^3}\,dt$ and $x\brak{0} = 2$, then $x\brak{1}$ is equal to  \hfill \brak{Jan 2024}

\begin{multicols}{4}
\begin{enumerate}
    \item $5$
    \item $12$
    \item $6$
    \item $8$
\end{enumerate}
\end{multicols}



% Question 7
\item Five people are distributed in four identical rooms. A room can also contain zero people. Find the number of ways to distribute them. \hfill \brak{Jan 2024}
\begin{multicols}{4}
\begin{enumerate}
    \item $47$
    \item $53$
    \item $43$
    \item $51$
\end{enumerate}
\end{multicols}


% Question 8
\item $5f\brak{x} + 4f\brak{\frac{1}{x}} = x^2 -4$ and $y = 9f\brak{x}x^2$. If $y$ is a strictly increasing function, find the interval of $x$.  \hfill \brak{Jan 2024}
\begin{multicols}{2}
\begin{enumerate}
    \item $\brak{ -\infty, \frac{-1}{\sqrt{5}} } \cup \brak{ \frac{-1}{\sqrt{5}}, 0 }$
    \item $\brak{-\frac{-1}{\sqrt{5}},0} \cup \brak{0, \frac{-1}{\sqrt{5}}}$
    \item $\brak{0, \frac{-1}{\sqrt{5}}} \cup \brak{\frac{-1}{\sqrt{5}},\infty}$
    \item $\brak{-\sqrt{\frac{2}{5}}, 0} \cup \brak{\sqrt{\frac{2}{5}}, \infty}$
\end{enumerate}
\end{multicols}


% Question 9
\item If the hyperbola $x^2 - y^2 \cosec^2 \theta = 5$ and the ellipse $x^2 \cosec^2 \theta + y^2 = 5$ has eccentricities $e_H$ and $e_e$ respectively, and $e_H =  \sqrt{7}{e_e}$, then $\theta$ is equal to:  \hfill \brak{Jan 2024}

\begin{multicols}{4}
\begin{enumerate}
    \item $\frac{\pi}{3}$
    \item $\frac{\pi}{6}$
    \item $\frac{\pi}{2}$
    \item $\frac{\pi}{4}$
\end{enumerate}
\end{multicols}

% Question 10
\item A bag contains 8 balls \brak{black and white}. If four balls are chosen without replacement and 2 white \brak{W} and 2 black \brak{B} balls are found, then the probability that the number of white and black balls are the same in the bag is equal to: \hfill \brak{Jan 2024}


\begin{multicols}{4}
\begin{enumerate}
    \item $\frac{1}{7}$
    \item $\frac{2}{7}$
    \item $\frac{3}{5}$
    \item $\frac{1}{2}$
\end{enumerate}
\end{multicols}


% Question 11
\item If two circles $x^2 + y^2 = 4$ and $x^2 + y^2 - 4\lambda x + 9 = 0$ intersect at two distinct points, then find the range of $\lambda$.  \hfill \brak{Jan 2024}
\begin{multicols}{2}
\begin{enumerate}
    \item $\brak{ -\infty, -\frac{13}{2}} \cup \brak{ -\frac{13}{2}, \infty}$
    \item $\brak{ -\infty, -\frac{13}{8}} \cup \brak{ -\frac{13}{8}, \infty}$
    \item $\sbrak{-\frac{13}{8},\frac{13}{8}}$
    \item $\lambda \in \brak{\frac{3}{2},\infty}$
\end{enumerate}
\end{multicols}

% Question 12
\item If $S = \{x \in R : 3\brak{\sqrt{3} + \sqrt{2}}^x + \brak{\sqrt{3} - \sqrt{2}}^x = \frac{10}{3} \}$, then the number of elements in set $S$ is  \hfill \brak{Jan 2024}

\begin{multicols}{4}
\begin{enumerate}
    \item Zero
    \item $1$
    \item $2$
    \item $3$
\end{enumerate}
\end{multicols}




% Question 13
\item $f(x) = \begin{cases} 
e^x, & x < 0 \\
\ln x, & x > 0 
\end{cases} g(x) = \begin{cases} 
e^x, & x < 0 \\
x, & x > 0 
\end{cases}$ The $gof: A \to R$ is   \hfill \brak{Jan 2024}

\begin{multicols}{2}
\begin{enumerate}
    \item Onto but not one-one
    \item Into and many-one
    \item Onto and one-one
    \item Into and one-one
\end{enumerate}
\end{multicols}



% Question 14
\item If $\tan A = \frac{1}{\sqrt{x^2 + x + 1}, \tan B = \frac{\sqrt{x}}{\sqrt{x^2 + x +1}} }$ and $\tan C = \frac{1}{\sqrt{x\brak{x^2 + x +1}}}$, then A + B =  \hfill \brak{Jan 2024}

\begin{multicols}{4}
\begin{enumerate}
    \item $0$ 
    \item $\pi - C$ 
    \item $\frac{\pi}{2} - C$ 
    \item None 
\end{enumerate}
\end{multicols}


% Question 15
\item $\lim_{x \to 0} \frac{\cos^{-1}\brak{1 - \{x\}^2}\sin^{-1}\brak{1 - \{x\}}}{\{x\} - \{x\}^3}$ where \{\} is fractional part function. If L.H.L = L and R.H.L = R, then the correct relation between L and R is \hfill 01-02-2024

\begin{multicols}{4}
\begin{enumerate}
    \item $\sqrt{2}R = 4L$ 
    \item $\sqrt{2}L = 4R$ 
    \item $R = l$ 
    \item $R = 2L$ 
\end{enumerate}
\end{multicols}

%\end{enumerate}
%\end{document}
