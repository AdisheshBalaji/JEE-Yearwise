\iffalse
  \title{Assignment}
  \author{EE24BTECH11038}
  \section{integer}
\fi 
\item If the integral 
\begin{align*}
    525\int_{0}^{\frac{\pi}{2}}\sin{2x}\cos^{\frac{11}{2}}{x}\brak{1+\cos^{\frac{5}{2}}{x}}\,dx
\end{align*}
is equal to $\brak{n\sqrt{2}-64}$ then n is equal to \hfill{Jan 2024}
\bigskip
\item Let $S=\brak{-1,\infty}$ and $f:S\rightarrow \mathbf{R}$ defined as 
\begin{align*}
    f\brak{x}=\int_{-1}^{x}\brak{e^{11}-1}^{11}\brak{2t-1}^5\brak{t-3}^{12}\brak{2t-10}^{61}\,dt
\end{align*}
Let p = Sum of square of the values of x, where f$\brak{x}$ attains local maxima on S. and q = Sum of the values of x, where f$\brak{x}$ attains local minima on S. Then, the value of $p^{2}$+2q is \hfill{Jan 2024}
\bigskip
\item The total number of words with or without meaning that can be formed out of the letters of the word `DISTRIBUTION' taken four at a time, 
is equal to,\hfill{Jan 2024}
\bigskip
\item Let Q and R be the feet of perpendiculars from the point $p\brak{a,a,a}$ on the lines x = y, z = 1 and x = -y,z = -1 respectively. If $\angle{QPR}$ is a right angle, then $12a^2$ is equal to \hfill{Jan 2024}
\bigskip
\item In the expansion of 
\begin{align*}
    \brak{1+x}\brak{1-x^2}\brak{1+\frac{3}{x}+\frac{3}{x^2}+\frac{1}{x^3}}^5, x\neq=0
\end{align*}
then the sum of coefficients of $x^3$ and $x^{-13}$ \hfill{Jan 2024}
\bigskip
\item if $\alpha$ denotes the number of solutions of $\abs{1-i}^{x}=2^x$ and $\beta=\brak{\frac{z}{arg\brak{z}}}$ where z$=\frac{\pi}{4}\brak{i+1}^4\brak{\brak{\frac{1-\sqrt{\pi}i}{\sqrt{\pi}+i}}+\brak{\frac{\sqrt{\pi}-1}{1+\sqrt{\pi}i}}},i=\sqrt{-1}$ then the  distance of the point $\brak{\alpha,\beta}$ from the line 4x-3y=7 \hfill{Jan 2024}
\bigskip
\item Let the foci and length of the latus rectum of an ellipse $\frac{x^2}{a^2}+\frac{y^2}{b^2}=1$, a$>$b be $\brak{\pm5,0}$ and $\sqrt{50}$ then the square of eccentrcity of hyperbola $\frac{x^2}{b^2}+\frac{y^2}{a^2b^2}=1$ \hfill{Jan 2024}
\bigskip
\item Let $\vec{a},\vec{b}$ be two vectors such that $\abs{a}=1,\abs{b}=4$ and a.b=2. If $\vec{c}=\brak{2\vec{a}\times\vec{b}}-3\vec{b}$ and the angle between $\vec{b}$ and $\vec{c}$ is $\alpha$ then 192$\sin^2{\alpha}$ \hfill{Jan 2024}
\bigskip
\item Let A = $\cbrak{1, 2, 3, 4}$ and R = $\cbrak{\brak{1, 2}, \brak{2, 3}, \brak{1, 4}}$ be a relation on A. Let  be the equivalence relation on A such that R $\subset$ S and the number of elements in S is n. Then, the minimum value of n is \hfill{Jan 2024}
\bigskip
\item Let f : $\mathbf{R} \rightarrow \mathbf{R}$ be a function defined by f$\brak{x}=\frac{4^x}{4^x+2}$ and 
\begin{align*}
    M=\int_{f\brak{a}}^{f\brak{1-a}} x\sin^{4}\brak{x\brak{1-x}}\,dx\\
    M=\int_{f\brak{a}}^{f\brak{1-a}} \sin^{4}\brak{x\brak{1-x}}\,dx   
\end{align*}
$\alpha M=\beta N$ then the value of $\alpha^2+\beta^2$ is \hfill{Jan 2024}

