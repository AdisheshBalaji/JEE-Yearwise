\iffalse
\title{2024}
\author{AI24BTECH11031}
\section{mcq-single}
\fi

\item For $0 < c < b < a$, let $\brak{a + b - 2c}x^2 + \brak{b + c - 2a}x
    + \brak{c + a - 2b} = 0$ and $\alpha \ne 1$ be one of its roots.
    Then, among the two statements
    \hfill{[Jan 2024]}

    (I) If $\alpha \in \brak{-1, 0}$, then $b$ cannot be the geometric
    mean of $a$ and $c$

    (II) If $\alpha \in \brak{0, 1}$, then $b$ may be the geometric
    mean of $a$ and $c$

    \begin{multicols}{2}
        \begin{enumerate}

            \item Both (I) and (II) are true
            \item Neither (I) nor (II) is true
            \item Only (II) is true
            \item Only (I) is true
        \end{enumerate}
    \end{multicols}

\item Let $a$ be the sum of all coefficients in the expansion of 
    \begin{align*}
    \brak{1 - 2x + 2x^2}^{2023} \brak{3 0 4x^2+2x^3}^{2024}
    \end{align*}
    and $b = \lim\limits_{x \to 0} \brak{\frac{\int_0^x \frac{\log\brak{1+t}}{t^{2024} + 1} dt} {x^2}}$.
    If the equations $cx^2 + dx + e = 0$ and $2bx^2 + ax + 4 = 0$
    have a common root, where $c, d, e \in R$, then $d : c : e$ equals
    \hfill{[Jan 2024]}

    \begin{multicols}{4}
        \begin{enumerate}

            \item $2:1:4$
            \item $4:1:4$
            \item $1:2:4$
            \item $1:1:4$
        \end{enumerate}
    \end{multicols}

\item If the foci of a hyperbola are same as that of the ellipse
    $\frac{x^2}{9} + \frac{y^2}{25} = 1$ and the eccentricity of the
    hyperbola is $\frac{15}{8}$ times the eccentricity of the
    ellipse, then the smaller focal distance of the point
    $\brak{\sqrt{2}, \frac{14}{3} \sqrt{\frac{2}{5}}}$ on the hyperbola,
    is equal to
    \hfill{[Jan 2024]}

    \begin{multicols}{4}
        \begin{enumerate}

            \item $7\sqrt{\frac{2}{5}} - \frac{8}{3}$
            \item $14\sqrt{\frac{2}{5}} - \frac{4}{3}$
            \item $14\sqrt{\frac{2}{5}} - \frac{16}{3}$
            \item $7\sqrt{\frac{2}{5}} + \frac{8}{3}$
        \end{enumerate}
    \end{multicols}

\item If one of the diameters of the circle $x^2 + y^2 - 10x +
    4y + 13 = 0$ is a chord of another circle $C$, whose
    center is the point of intersection of the lines $2x +
    3y = 12$ and $3x - 2y = 5$, then the radius of the
    circle $C$ is
    \hfill{[Jan 2024]}

    \begin{multicols}{4}
        \begin{enumerate}

            \item $\sqrt{20}$
            \item 4
            \item 6
            \item $3\sqrt{2}$
        \end{enumerate}
    \end{multicols}

\item The area of the region
    \begin{align*}
        \cbrak{\brak{x,y}: y^2\le 4x,x<4,\frac{xy\brak{x-1}\brak{x-2}}{\brak{x-3}\brak{x-4}}>0,x\ne3}
    \end{align*}
    is
    \hfill{[Jan 2024]}

    \begin{multicols}{4}
        \begin{enumerate}

            \item $\frac{16}{3}$
            \item $\frac{64}{3}$
            \item $\frac{8}{3}$
            \item $\frac{32}{3}$
        \end{enumerate}
    \end{multicols}

\item If $f\brak{x} = \frac{4x+3}{6x-4}$, $x \ne \frac{2}{3}$ and
    $\brak{fof}\brak{x} = g\brak{x}$, where $g:\mathbb{R} - \cbrak{\frac{2}{3}} \to
    \mathbb{R} - \cbrak{\frac{2}{3}}$ then $\brak{gogog}\brak{4}$ is equal to
    \hfill{[Jan 2024]}

    \begin{multicols}{4}
        \begin{enumerate}

            \item $-\frac{19}{20}$
            \item $\frac{19}{20}$
            \item -4
            \item 4
        \end{enumerate}
    \end{multicols}

\item $\lim\limits_{x \to 0} \frac{e^{2\abs{\sin x}} - 2\abs{\sin x} - 1}{x^2}$
    \hfill{[Jan 2024]}

    \begin{multicols}{4}
        \begin{enumerate}

            \item is equal to -1 
            \item does not exist
            \item is equal to 1 
            \item is equal to 2 
        \end{enumerate}
    \end{multicols}

\item If the system of linear equations
    \begin{align*}
        x - 2y + z = -4 \\
        2x + \alpha y + 3z = 5 \\
        3x - y + \beta z = 3
    \end{align*}
    has infinitely many solutions, then $12\alpha + 13\beta$ is equal to
    \hfill{[Jan 2024]}

    \begin{multicols}{4}
        \begin{enumerate}

            \item 60
            \item 64
            \item 54
            \item 58
        \end{enumerate}
    \end{multicols}

\item The solution curve of the differential equation $y\frac{dx}{dy}
    = x\brak{\log_e x - \log_e y + 1}, x > 0, y > 0$ passing through the point
    $\brak{e, 1}$ is
    \hfill{[Jan 2024]}

    \begin{multicols}{2}
        \begin{enumerate}

            \item $\abs{\log_e \frac{y}{x}} = x$
            \item $\abs{\log_e \frac{y}{x}} = y^2$
            \item $\abs{\log_e \frac{x}{y}} = y$
            \item $\abs{\log_e \frac{x}{y}} = y + 1$
        \end{enumerate}
    \end{multicols}

\item Let $\alpha,\beta,\gamma,\delta \in Z$ and let $A\brak{\alpha,\beta}$,
    $B\brak{1, 0}$, $C\brak{\gamma,\delta}$ and $D\brak{1, 2}$ be the
    vertices of a parallelogram $ABCD$. If $AB = 10$ and the points
    $A$ and $C$ lie on the line $3y = 2x + 1$, then
    $2\brak{\alpha + \beta + \gamma + \delta}$ is equal to
    \hfill{[Jan 2024]}

    \begin{multicols}{4}
        \begin{enumerate}

            \item 10
            \item 5
            \item 12
            \item 8
        \end{enumerate}
    \end{multicols}

\item Let $y = y\brak{x}$ be the solution of the differential equation
    \begin{align*}
        \frac{dy}{dx} = \frac{\brak{\tan x} + y}{\sin x\brak{\sec x - \sin x \tan x}},
        x \in \brak{0, \frac{\pi}{2}}
    \end{align*}
    satisfying the condition $y\brak{\frac{\pi}{4}} = 2$.
    Then $y\brak{\frac{\pi}{3}}$ is
    \hfill{[Jan 2024]}

    \begin{multicols}{2}
        \begin{enumerate}

            \item $\sqrt{3}\brak{2 + \log_e \sqrt{3}}$
            \item $\frac{\sqrt{3}}{2}\brak{2 + \log_e 3}$
            \item $\sqrt{3}\brak{1 + 2 \log_e 3}$
            \item $\sqrt{3}\brak{2 + \log_e 3}$
        \end{enumerate}
    \end{multicols}

\item Let $\overrightarrow{a} = 3\hat{i} + \hat{j} - 2\hat{k}, \overrightarrow{b} = 4\hat{i} + \hat{j} + 7\hat{k}$
    and $\overrightarrow{c} = \hat{i} - 3\hat{j} + 4\hat{k}$ be three vectors. If a
    vector $\overrightarrow{p}$ satisfies $\overrightarrow{p} \times \overrightarrow{b} = \overrightarrow{c} \times \overrightarrow{b}$
    and $\overrightarrow{p} \cdot \overrightarrow{a} = 0$, then $\overrightarrow{p} \cdot \brak{\hat{i} - \hat{j} - \hat{k}}$
    is equal to
    \hfill{[Jan 2024]}

    \begin{multicols}{4}
        \begin{enumerate}

            \item 24
            \item 36
            \item 28
            \item 32
        \end{enumerate}
    \end{multicols}

\item The sum of the series $\frac{1}{1-3\cdot 1^2+1^4}+\frac{2}{1-3\cdot 2^2+2^4}
    +\frac{3}{1-3\cdot 3^2+3^4} + \dots$ upto 10 terms is
    \hfill{[Jan 2024]}

    \begin{multicols}{4}
        \begin{enumerate}

            \item $\frac{45}{109}$
            \item $-\frac{45}{109}$
            \item $\frac{55}{109}$
            \item $-\frac{55}{109}$
        \end{enumerate}
    \end{multicols}

\item The distance of the point $Q\brak{0, 2, -2}$ from the line passing
    through the point $P\brak{5, -4, 3}$ and perpendicular to the lines 
    $\overrightarrow{r} = \brak{-3\hat{i} + 2\hat{k}} + \lambda\brak{2\hat{i} + 3\hat{j} + 5\hat{k}}, \lambda \in \mathbb{R}$
    and $\overrightarrow{r} = \brak{\hat{i} - 2\hat{j} + \hat{k}} + \mu\brak{-\hat{i} + 3\hat{j} + 2\hat{k}}, \mu \in \mathbb{R}$
    is
    \hfill{[Jan 2024]}

    \begin{multicols}{4}
        \begin{enumerate}

            \item $\sqrt{86}$
            \item $\sqrt{20}$
            \item $\sqrt{54}$
            \item $\sqrt{74}$
        \end{enumerate}
    \end{multicols}

\item For $\alpha, \beta, \gamma \ne 0$, if $\sin^{-1}\alpha + \sin^{-1}\beta + \sin^{-1}\gamma = \pi$
    and $\brak{\alpha+\beta+\gamma}\brak{\alpha-\gamma+\beta} = 3\alpha\beta$ then $\gamma$ is
    equal to
    \hfill{[Jan 2024]}

    \begin{multicols}{4}
        \begin{enumerate}

            \item $\frac{\sqrt{3}}{2}$
            \item $\frac{1}{\sqrt{2}}$
            \item $\frac{\sqrt{3} - 1}{2\sqrt{2}}$
            \item $\sqrt{3}$
        \end{enumerate}
    \end{multicols}
