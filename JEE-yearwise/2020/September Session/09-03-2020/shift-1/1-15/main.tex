\iffalse
\title{2020}
\author{AI24BTECH11012}
\section{mcq-single}
\fi
    \item The value of $ (2.^1P_0 - 3.^2P_1 + 4.^3P_2 - \cdots \text{ up to 51st term}) + (1! - 2! + 3! - \cdots \text{ up to 51st term}) $ 
    is equal to:
    \hfill{[Sep 2020]}
    \begin{enumerate}
        \item $ 1 - 51(51)! $
        \item $ 1 + (52)! $
        \item $ 1 $
        \item $ 1 + (51)! $
    \end{enumerate}

    \item Let $P$ be a point on the parabola $ y^2 = 12x $ and $ N $ be the foot of the perpendicular drawn from $ P $ on the axis of the parabola. A line is now drawn through the mid-point $ M $ of $ PN $, parallel to its axis which meets the parabola at $ Q $. If the y-intercept of the line $ NQ $ is $ \frac{4}{3} $, then:
    \hfill{[Sep 2020]}
    \begin{enumerate}
        \item $ PN = 4 $
        \item $ MQ = \frac{1}{3} $
        \item $ PN = 3 $
        \item $ MQ = \frac{1}{4} $
    \end{enumerate}

    \item If $ \Delta = \begin{vmatrix}
    x-2 & 2x-3 & 3x-4 \\
    2x-3 & 3x-4 & 4x-5 \\
    3x-5 & 5x-8 & 10x-17
    \end{vmatrix} = Ax^3 + Bx^2 + Cx + D, $ 
    then $ B + C $ is equal to:
    \hfill{[Sep 2020]}
    \begin{enumerate}
        \item $ 1 $
        \item $ -1 $
        \item $ -3 $
        \item $ 9 $
    \end{enumerate}

    \item The foot of the perpendicular drawn from the point $ (4, 2, 3) $ to the line joining the points $ (1, -2, 3) $ and $ (1, 1, 0) $ lies on the plane:
    \hfill{[Sep 2020]}
    \begin{enumerate}
        \item $ x - y - 2z = 1 $
        \item $ x - 2y + z = 1 $
        \item $ 2x + y - z = 1 $
        \item $ x + 2y - z = 1 $
    \end{enumerate}

    \item If $ y^2 + \log_e(\cos^2x) = y $, $ x \in \brak{-\frac{\pi}{2}, \frac{\pi}{2}} $, then:
    \hfill{[Sep 2020]}
    \begin{enumerate}
        \item $ |y'(0)| + |y''(0)| = 1 $
        \item $ y''(0) = 0 $
        \item $ |y'(0)| + |y''(0)| = 3 $
        \item $ |y''(0)| = 2 $
    \end{enumerate}

    \item $ 2\pi-\brak{\sin^{-1}\frac{4}{5} - \sin^{-1}\frac{5}{13} + \sin^{-1}\frac{16}{65}}$ 
    is equal to:
    \hfill{[Sep 2020]}
    \begin{enumerate}
        \item $ \frac{5\pi}{4} $
        \item $ \frac{3\pi}{2} $
        \item $ \frac{7\pi}{4} $
        \item $ \frac{\pi}{2} $
    \end{enumerate}

    \item A hyperbola having the transverse axis of length $\sqrt{2}$ has the same foci as that of the ellipse $ 3x^2 + 4y^2 = 12 $. Then this hyperbola does not pass through which of the following points:
    \hfill{[Sep 2020]}
    \begin{enumerate}
        \item $ \brak{ \sqrt{\frac{3}{2}}, \frac{1}{\sqrt{2}} } $
        \item $ \brak{ 1, \frac{-1}{\sqrt{2}} } $
        \item $ \brak{ \frac{1}{\sqrt{2}}, 0 } $
        \item $ \brak{ -\sqrt{\frac{3}{2}}, 1} $
    \end{enumerate}

    \item For the frequency distribution:
\[
\begin{array}{c c c c c c c}
\text{Variate:} & x_1 & x_2 & x_3 & \dots & x_{15} \\
\text{Frequency:} & f_1 & f_2 & f_3 & \dots & f_{15}
\end{array}
\]
    where $ 0 < x_1 < x_2 < \dots < x_{15} \leq 10 $ and $ \sum_{i=1}^{15} f_i > 0 $, the standard deviation cannot be:
    \hfill{[Sep 2020]}
    \begin{enumerate}
        \item $ 1 $
        \item $ 4 $
        \item $ 6 $
        \item $ 2 $
    \end{enumerate}

    \item A die is thrown two times and the sum of the scores appearing on the die is observed to be a multiple of 4. Then the conditional probability that the score 4 has appeared at least once is:
    \hfill{[Sep 2020]}
    \begin{enumerate}
        \item $ \frac{1}{3} $
        \item $ \frac{1}{4} $
        \item $ \frac{1}{8} $
        \item $ \frac{1}{9} $
    \end{enumerate}

    \item If the number of integral terms in the expansion of $ \brak{3^{\frac{1}{2}} + 5^{\frac{1}{8}}}^{n} $ 
    is exactly 33, then the least value of $ n $ is:
    \hfill{[Sep 2020]}
    \begin{enumerate}
        \item 128
        \item 248
        \item 256
        \item 264
    \end{enumerate}

    \item $ \int_{-\pi}^{\pi} |\pi-|x|| dx $ is:
    \hfill{[Sep 2020]}
    \begin{enumerate}
        \item $ \pi^2 $
        \item $ \frac{\pi^2}{2} $
        \item $ \sqrt{2} \pi^2$
        \item $ 2\pi^2 $
    \end{enumerate}

    \item Consider the two sets:\\
    $ A = \brak{ m \in \mathbb{R} : \text{both the roots of } x^2 - (m+1)x + m + 4 = 0 \text{ are real} }$and$\quad B = [-3, 5) $\\
    Which of the following is not true?
    \hfill{[Sep 2020]}
    \begin{enumerate}
        \item $ A - B = \brak{-\infty, -3} \cup \brak{5, \infty} $
        \item $ A \cap B = \{-3\} $
        \item $ B - A = (-3, 5) $
        \item $ A \cup B = \mathbb{R} $
    \end{enumerate}

    \item The proposition $ p \to \sim\brak{p \land \sim q} $ is equivalent to:
    \hfill{[Sep 2020]}
    \begin{enumerate}
        \item $ \brak{\sim p} \lor \brak{\sim q} $
        \item $ \brak{\sim p} \land q $
        \item $ q $
        \item $ \brak{\sim p} \lor q $
    \end{enumerate}

    \item The function $ f(x) = (3x - 7)x^{2/3} $, $ x \in \mathbb{R} $, is increasing for all $ x $ lying in:
    \hfill{[Sep 2020]}
    \begin{enumerate}
        \item $ \brak{ -\infty, -\frac{14}{15} } \cup (0, \infty) $
        \item $ \brak{ -\infty, \frac{14}{15} } $
        \item $ \brak{ -\infty, 0 } \cup \brak{ \frac{14}{15}, \infty } $
        \item $ \brak{ -\infty, 0} \cup \brak{ \frac{3}{7}, \infty } $
    \end{enumerate}

    \item If the first term of an A.P. is 3 and the sum of its first 25 terms is equal to the sum of its next 15 terms, then the common difference of this A.P. is:
    \hfill{[Sep 2020]}
    \begin{enumerate}
        \item $ \frac{1}{6} $
        \item $ \frac{1}{5} $
        \item $ \frac{1}{4} $
        \item $ \frac{1}{7} $
    \end{enumerate}




