\iffalse
\title{2020}
\author{EE24BTECH11066}
\section{integer}
\fi
\item Let $\{x\}$ and ${[x]}$ denote the fractional part of $x$ and the greatest integer $\leq x$ respectively of a real number $x$. If $\int\limits_0^n\{x\} \, \text{dx},\int\limits_0^n{[x]} \,\text{dx}$ and $10\brak{n^2-n},\brak{n \in N,n>1}$ are three consecutive terms of a G.P., then $n$ is equal to \underline{\hspace{1cm}} \hfill{[September 2020]}\\ 

\item A test consists of $6$ multiple choice questions, each having $4$ alternative answers of which only one is correct. The number of ways, in which a candidate answers all six questions such that exactly four of the answers are correct, is \underline{\hspace{1cm}} \hfill{[September 2020]}\\

\item If $\bar{a} = 2\hat{i} + \hat{j} + 2\hat{k}$, then the value of $\abs{\hat{i} \times \brak{\bar{a} \times \hat{i}}}^2 + \abs{\hat{j} \times \brak{\bar{a} \times \hat{j}}}^2 + \abs{\hat{k} \times \brak{\bar{a} \times \hat{k}}}^2$  is equal to \underline{\hspace{1cm}} \hfill{[September 2020]}\\

\item Let $PQ$ be a diameter of the circle $x^2+y^2=9$. If $\alpha$ and $\beta$ are the lengths of the perpendiculars from $\vec{P}$ and $\vec{Q}$ on the straight line, $x+y=2$ respectively, then the maximum value of $\alpha\beta$ is \underline{\hspace{1cm}} \hfill{[September 2020]}\\

\item If the variance of the following frequency distribution:

\begin{center}
\begin{tabular}{|c|c|c|c|}
\hline
\textbf{Class}     & 10-20 & 20-30 & 30-40 \\
\hline
\textbf{Frequency} & 2     & x     & 2     \\
\hline
\end{tabular}
\end{center}

is $50$, then  x is equal to \underline{\hspace{1cm}} \hfill{[September 2020]}

