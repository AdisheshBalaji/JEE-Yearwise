\iffalse
\let\negmedspace\undefined
\let\negthickspace\undefined
\documentclass[journal]{IEEEtran}
\usepackage[a5paper, margin=10mm, onecolumn]{geometry}
%\usepackage{lmodern} % Ensure lmodern is loaded for pdflatex
\usepackage{tfrupee} % Include tfrupee package

\setlength{\headheight}{1cm} % Set the height of the header box
\setlength{\headsep}{0mm}     % Set the distance between the header box and the top of the text

\usepackage{gvv-book}
\usepackage{gvv}
\usepackage{cite}
\usepackage{amsmath,amssymb,amsfonts,amsthm}
\usepackage{algorithmic}
\usepackage{graphicx}
\usepackage{textcomp}
\usepackage{xcolor}
\usepackage{txfonts}
\usepackage{listings}
\usepackage{enumitem}
\usepackage{mathtools}
\usepackage{gensymb}
\usepackage{comment}
\usepackage[breaklinks=true]{hyperref}
\usepackage{tkz-euclide} 
\usepackage{listings}
% \usepackage{gvv}                                        
\def\inputGnumericTable{}                                 
\usepackage[latin1]{inputenc}                                
\usepackage{color}                                            
\usepackage{array}                                            
\usepackage{longtable}                                       
\usepackage{calc}                                             
\usepackage{multirow}                                         
\usepackage{hhline}                                           
\usepackage{ifthen}                                           
\usepackage{lscape}
\bibliographystyle{IEEEtran}
\vspace{3cm}

\title{2020}
\author{EE24BTECH11061}
\maketitle

\renewcommand{\thefigure}{\theenumi}
\renewcommand{\thetable}{\theenumi}
\section{mcq-single}
\fi
%\begin{enumerate}
\item A line parallel to the straight line $2x-y = 0$ is tangent to the hyperbola $\frac{x^2}{4}-\frac{y^2}{2} = 1$ at the point $\brak{x_1, y_1}$. Then $x_1^2 + 5y_1^2$ is equal to:
\hfill{\sbrak{\text{Sep 2020}}}
\begin{multicols}{2}
    \begin{enumerate}
        \item $6$
        \item $10$
        \item $8$
        \item $5$
    \end{enumerate}
\end{multicols}

\item The domain of the function $f(x) = \sin^{-1}\brak{\frac{\abs{x}+5}{x^2+1}}$ is $(-\infty,-a] \cup [a,\infty) $. Then $a$ is equal to:
\hfill{\sbrak{\text{Sep 2020}}}
\begin{multicols}{2}
    \begin{enumerate}
        \item $\frac{\sqrt{17}-1}{2}$
        \item $\frac{\sqrt{17}}{2}$
        \item $\frac{1+\sqrt{17}}{2}$
        \item $\frac{\sqrt{17}}{2} + 1$
        \end{enumerate}
\end{multicols}

\item If a function $f(x)$ defined by
\begin{align*}
f(x) = \begin{cases}
ae^x + be^{-x}, & \text{if } -1\leq x < 1\\
cx^2, & \text{if } 1\leq x \leq 3\\
ax^2 + 2cx, & \text{if } 3 < x \leq 4
\end{cases}
\end{align*}
be continuous for some $a,b,c$ $\epsilon$ $\mathbb{R}$ and $f^{\prime}(0) + f^{\prime}(2) = e$, then the value of $a$ is:
\hfill{\sbrak{\text{Sep 2020}}}
\begin{multicols}{2}
    \begin{enumerate}
        \item $\frac{1}{e^2-3e+13}$
        \item $\frac{e}{e^2-3e-13}$
        \item $\frac{e}{e^2+3e+13}$
        \item $\frac{e}{e^2-3e+13}$
    \end{enumerate}
\end{multicols}


\item The sum of the first three terms of G.P is S and their product is 27. Then all such S lie in
\hfill{\sbrak{\text{Sep 2020}}}
\begin{multicols}{2}
    \begin{enumerate}
        \item $(-\infty, -9] \cup [3,\infty]$
        \item $[-3,\infty)$
        \item $(-\infty,-9]$
        \item $(-\infty,-3] \cup [9,\infty)$
    \end{enumerate}
\end{multicols}

\item If $R$ = \{$(x,y)$ $\colon$ $x,y$ $\epsilon$ $\mathbb{Z}$, $x^2+3y^2 \leq 8$\} is relation on the set of integers $\mathbb{Z}$, then the domain of R$^{-1}$ is:
\hfill{\sbrak{\text{Sep 2020}}}
\begin{multicols}{2}
    \begin{enumerate}
        \item \{-1, 0, 1\}
        \item \{-2, -1, 1, 2\}
        \item \{0, 1\}
        \item \{-2, -1, 0, 1, 2\}
    \end{enumerate}
\end{multicols}

\item The value of $\brak{\frac{1+\sin{\frac{2\pi}{9}} + \iota \cos{\frac{2\pi}{9}}}{{1+\sin{\frac{2\pi}{9}} - \iota \cos{\frac{2\pi}{9}}}}}^3$
\hfill{\sbrak{\text{Sep 2020}}}
\begin{multicols}{2}
    \begin{enumerate}
        \item $\frac{-1}{2}\brak{1-\iota\sqrt{3}}$
        \item $\frac{1}{2}\brak{1-\iota\sqrt{3}}$
        \item $\frac{-1}{2}\brak{\sqrt{3}-\iota}$
        \item $\frac{1}{2}\brak{\sqrt{3}-\iota}$
    \end{enumerate}
\end{multicols}

\item Let $\vec{P}\brak{h,k}$ be a point on the curve $y=x^2+7x+2$, nearest to the line, $y=3x-3$. Then the equation of the normal to the curve at $\vec{P}$ is:
\hfill{\sbrak{\text{Sep 2020}}}
\begin{multicols}{2}
    \begin{enumerate}
        \item $x+3y-62=0$
        \item $x-3y-11=0$
        \item $x-3y+22=0$
        \item $x+3y+26=0$
    \end{enumerate}
\end{multicols}

\item Let A be a $2\times2$ real matrix with entries from \{0,1\} and A$\neq$0. Consider the following two statements:\\
(P) If A$\neq$ $I_2$, then A = -1\\
(Q) If $\det{A}$ = 1, then tr(A) = 2,\\
where $I_2$ denotes $2\times2$ identity matrix and tr(A) denotes the sum of the diagonal entries of A. Then:
\hfill{\sbrak{\text{Sep 2020}}}
\begin{multicols}{2}
    \begin{enumerate}
        \item Both (P) and (Q) are false
        \item (P) is true and (Q) is false
        \item Both (P) and (Q) are false
        \item (P) is false and (Q) is true
    \end{enumerate}
\end{multicols}

\item Box I contains 30 cards numbered 1 to 30 and Box II contains 20 cards numbered 31 to 50. A box is selected at random and a card is drawn from it. The number on the card is found to be a non-prime number. The probability that the card was drawn from Box I is:
\hfill{\sbrak{\text{Sep 2020}}}
\begin{multicols}{2}
    \begin{enumerate}
        \item $\frac{4}{17}$
        \item $\frac{8}{17}$
        \item $\frac{2}{5}$
        \item $\frac{2}{3}$
    \end{enumerate}
\end{multicols}

\item If $p(x)$ be a polynomial of degree three that has a local maximum value 8 at $x=1$ and a local minimum value 4 at $x=2$; then $p(0)$ is equal to:
\hfill{\sbrak{\text{Sep 2020}}}
\begin{multicols}{2}
    \begin{enumerate}
        \item 12
        \item -12
        \item -24
        \item 6
    \end{enumerate}
\end{multicols}

\item The contra-positive of the statement "If I reach the station in time, then I will catch the train" is:
\hfill{\sbrak{\text{Sep 2020}}}
\begin{multicols}{2}
    \begin{enumerate}
        \item If I will catch the train, then I reach the station in time.
        \item If I do not reach the station in time, then I will catch the train.
        \item  If I do not reach the station in time, then I will not catch the train.
        \item If I will not catch the train, then I do not reach the station in time.
    \end{enumerate}
\end{multicols}

\item Let $\alpha$ and $\beta$ be the roots of the equation, $5x^2+6x-2=0$. If $S_n = {\alpha}^n + {\beta}^n$, $n=1,2,3,\dots$ then:
\hfill{\sbrak{\text{Sep 2020}}}
\begin{multicols}{2}
    \begin{enumerate}
        \item $5S_6+6S_5+2S_4=0$
        \item $6S_6+5S_5=2S_4$
        \item $6S_6+5S_5+2S_4=0$
        \item $5S_6+6S_5=2S_4$
    \end{enumerate}
\end{multicols}

\item If the tangent to the curve $y=x+\sin y$ at a point $\brak{a,b}$ is parallel to the line joining $\brak{0,\frac{3}{2}}$ and $\brak{\frac{1}{2},2}$, then:
\hfill{\sbrak{\text{Sep 2020}}}
\begin{multicols}{2}
    \begin{enumerate}
        \item $b=\brak{\frac{\pi}{2}+a}$
        \item $\abs{a+b} = 1$
        \item $\abs{b-a} = 1$
        \item $b=a$
    \end{enumerate}
\end{multicols}

\item Area (in sq. units) of the region outside $\frac{x}{2} + \frac{y}{3} = 1$ and inside the ellipse $\frac{x^2}{4} + \frac{y^2}{9} = 1$ is:
\hfill{\sbrak{\text{Sep 2020}}}
\begin{multicols}{2}
    \begin{enumerate}
        \item $3\brak{\pi - 2}$
        \item $6\brak{\pi - 2}$
        \item $6\brak{4 - \pi}$
        \item $3\brak{4 - \pi}$
    \end{enumerate}
\end{multicols}

\item If $\abs{x} < 1$, $\abs{y} < 1$, and $xy$, then the sum to infinity of the following series 
\begin{align*}
    \brak{x+y} + \brak{x^2+xy+y^2} + \brak{x^3 + x^2y + xy^2 + y^3} + \dots
\end{align*}
is:
\hfill{\sbrak{\text{Sep 2020}}}
\begin{multicols}{2}
    \begin{enumerate}
        \item $\frac{x+y+xy}{\brak{1-x}\brak{1-y}}$
        \item $\frac{x+y-xy}{\brak{1-x}\brak{1-y}}$
        \item $\frac{x+y+xy}{\brak{1+x}\brak{1+y}}$
        \item $\frac{x+y-xy}{\brak{1+x}\brak{1+y}}$
    \end{enumerate}
\end{multicols}
%\end{enumerate}