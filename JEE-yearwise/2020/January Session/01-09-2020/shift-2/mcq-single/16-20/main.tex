\iffalse
    \title{2020}
    \author{EE24BTECH11011}
    \section{mcq-single}
\fi 
 
    \item If $\frac{dy}{dx} = \frac{xy}{x^2+y^2}$ and $y\brak{1}=1$, then a value of $x$ satisfying $y\brak{x}=e$ is: \hfill[2020-Jan]
    \begin{multicols}{2} % Use multicol to divide options into two columns
    \begin{enumerate}
        \item $\sqrt{3} e$\\
        \item $\frac{1}{2} \sqrt{3} e$\\
        \item $\sqrt{2}e$\\
        \item $\frac{e}{\sqrt{2}}$
    \end{enumerate}
    \end{multicols}

    % Second question
    \item If one end of focal chord $AB$ of the parabola $y^2 = 8x$ is at $A\brak{\frac{1}{2},-2}$, then the equation of the tangent at $B$ is:\hfill[2020-Jan]
    \begin{multicols}{2}
    \begin{enumerate}
        \item $x+2y+8=0$
        \item $2x-y-24=0$
        \item $x-2y+8=0$
        \item $2x+y-24=0$
    \end{enumerate}
    \end{multicols}
  \item Let $a_n$ be the $n^{th}$ term of a G.P. of positive terms. If $\sum_{n=1}^{100} a_{2n+1} = 200$ and $\sum_{n=1}^{100}a_{2n} = 100$, then $\sum_{n=1}^{200}a_n$ is equal to:\hfill[2020-Jan]
    \begin{multicols}{2}
    \begin{enumerate}
        \item $300$\\
        \item $175$
        \item $225$\\
        \item $150$
    \end{enumerate}
    \end{multicols}

\item A random variable $X$ has the following probability distribution.
    \begin{table}[h!]    
        \centering
        \begin{tabular}{|c|c|c|c|c|c|} \hline $X$ & $1$ & $2$ & $3$ & $4$ & $5$ \\ \hline P$\brak{X}$ & $K^2$ & $2K$ & $K$ & $2K$ & $5 K^2$ \\ \hline \end{tabular}
    \end{table}
    Then $P\brak{X>2}$ is:\hfill[2020-Jan]
    \begin{multicols}{2}
    \begin{enumerate}
        \item $\frac{7}{12}$\\
        \item $\frac{23}{26}$
        \item $\frac{1}{36}$\\
        \item $\frac{1}{6}$\\
    \end{enumerate}
    \end{multicols}

    % Fifth question
    \item If $\int \frac{d\theta}{\cos^2 \theta (\tan 2\theta + \sec 2 \theta)} = \lambda \tan \theta + 2 \log_e \abs {f\brak{\theta}}  + C$, where $C$ is the constant of integration, then the ordered point $\brak{\lambda, f\brak{\theta}}$ is:\hfill[2020-Jan]
    \begin{multicols}{2}
    \begin{enumerate}
        \item $\brak{-1, 1-\tan \theta}$\\
        \item $\brak{-1, 1+\tan \theta}$\\
        \item $\brak{1, 1+\tan \theta}$\\
        \item $\brak{1, 1-\tan \theta}$
    \end{enumerate}
    \end{multicols}

