\iffalse
   \title{Assignment}
   \author{EE24BTECH11034}
   \section{mcq-single}
\fi   
   

    \item If \( A = \{x \in \mathbb{R} : |x| < 2\} \) and \( B = \{x \in \mathbb{R} : |x - 2| \geq 3\} \), then: \hfill{(JAN 2020)}
        \begin{multicols}{4}
        
        \begin{enumerate}
            \item \( A - B = [-1, 2] \)  
            \item \( B - A = \mathbb{R} - (-2, 5) \)  
            \item \( A \cup B = \mathbb{R} - (2, 5) \)  
            \item \( A \cap B = (-2, -1) \)
        \end{enumerate}
        \end{multicols}

    \item If 10 different balls have to be placed in 4 distinct boxes at random, then the probability that two of these boxes contain exactly 2 and 3 balls is: \hfill{(JAN 2020)}
        \begin{multicols}{4}
        \begin{enumerate}
            \item \( \frac{965}{2^{10}} \)  
            \item \( \frac{945}{2^{10}} \)  
            \item \( \frac{945}{2^{11}} \)  
            \item \( \frac{965}{2^{11}} \)
        \end{enumerate}
        \end{multicols}

    \item If \( x = 2 \sin \theta - \sin 2 \theta \) and \( y = 2 \cos \theta - \cos 2 \theta \), \( \theta \in [0, 2\pi] \), then \( \frac{d^2y}{dx^2} \) at \( \theta = \pi \) is: \hfill{(JAN 2020)}
        \begin{multicols}{4}
        \begin{enumerate}
            \item \( -\frac{3}{8} \)  
            \item \( \frac{3}{4} \)  
            \item \( \frac{3}{2} \)  
            \item \( -\frac{3}{4} \)
        \end{enumerate}
        \end{multicols}

    \item Let \( f \) and \( g \) be differentiable functions on \( \mathbb{R} \), such that \( f \circ g \) is the identity function. If for some \( a, b \in \mathbb{R} \), \( g'(a) = 5 \) and \( g(a) = b \), then \( f'(b) \) is equal to: \hfill{(JAN 2020)}
        \begin{multicols}{4}
        \begin{enumerate}
            \item \( \frac{2}{5} \)  
            \item \( 5 \)  
            \item \( 1 \)  
            \item \( \frac{1}{5} \)
        \end{enumerate}
        \end{multicols}

    \item In the expansion of 
    \[
    \left( \frac{x}{\cos \theta} + \frac{1}{x \sin \theta} \right)^{16},
    \]
    if \( I_1 \) is the least value of the term independent of \( x \) when \( \left( \frac{\pi}{8} \right) \leq \theta \leq \left( \frac{\pi}{4} \right) \) and \( I_2 \) is the least value of the term independent of \( x \) when \( \left( \frac{\pi}{16} \right) \leq \theta \leq \left( \frac{\pi}{8} \right) \), then the ratio \( I_2 : I_1 \) is equal to: \hfill{(JAN 2020)}
        \begin{multicols}{4}
        \begin{enumerate}
            \item \( 16:1 \)  
            \item \( 8:1 \)  
            \item \( 1:8 \)  
            \item \( 1:16 \)
        \end{enumerate}
        \end{multicols}
        
    \item Let $a,b\in\mathbb{R}, a\neq0$, such that the equation,

        \[ax^2-2bx+5=0\]
        
        has a repeated root $\alpha$, which is also a root of the equation $x^2-2bx-10=0$. If $\beta$ is the root of this equation, then $a^2+b^2$ is equal to: \hfill{(JAN 2020)}
        
        \begin{enumerate}
        \item 24
        \item 25
        \item 26
        \item 28
        \end{enumerate}
           
    \item Let a function $f:[0,5] \rightarrow \mathbb{R}$ be continuous, $f(1) = 3$ and $F$ be defined as:

        \[F(x) = \int_{1}^{x}t^{2}g(t)dt\]
        
        where
        
        \[g(t) = \int_{1}^{t}f(u)du\]
        
        Then for the function $F$, the point $x=1$ is \hfill{(JAN 2020)}
        
        \begin{enumerate}
        \item a point of inflection.
        \item a point of local maxima
        \item a point of local minima
        \item not a critical point
        \end{enumerate}

    \item Let $[t]$ denote the greatest integer $\leq t$ and

        \[\lim_{x\rightarrow0}x\left[\frac{4}{x}\right] = A\]
        
        . Then the function, $f(x)=[x^{2}] \sin \pi x$ is discontinuous, when $x$ is equal to \hfill{(JAN 2020)}
        
        \begin{enumerate}
        \item $\sqrt{(A+1)}$
        \item $\sqrt{A}$
        \item $\sqrt{(A+5)}$
        \item $\sqrt{(A+21)}$
        \end{enumerate}

    \item Let $a-2b+c=1$. If $f(x)=$

        \[\begin{vmatrix}
        x+a & x+2 & x+1 \\
        x+b & x+3 & x+2 \\
        x+c & x+4 & x+3
        \end{vmatrix}\]
        
        , then \hfill{(JAN 2020)}
        
        \begin{enumerate}
        \item $f(-50)=501$
        \item $f(-50)=-1$
        \item $f(50)=1$
        \item $f(-50)=-501$
        \end{enumerate}
        
    \item Given:
        \[f(x)=\begin{cases}
        x, & 0 \leq x < \frac{1}{2} \\
        \frac{1}{2}, & x = \frac{1}{2} \\
        1-x, & \frac{1}{2} < x \leq 1
        \end{cases}\]
        
        and $g(x) = (x-1/2)^{2}, x \in \mathbb{R}$. Then the area (in sq. units) of the region bounded by the curves $y = f(x)$ and $y = g(x)$ between the lines $2x=1$ to $2x=\sqrt{3}$ is: \hfill{(JAN 2020)}
        
        \begin{enumerate}
        \item $(\sqrt{3}/4)-(1/3)$
        \item $(1/3)+(\sqrt{3}/4)$
        \item $(1/2)+(\sqrt{3}/4)$
        \item $(1/2)-(\sqrt{3}/4)$
        \end{enumerate}

    \item The following system of linear equations
        \[7x+6y-2z=0\] 
        \[3x+4y+2z=0\]
        \[x-2y-6z=0\]
        has \hfill{(JAN 2020)}
        \begin{enumerate}
        \item infinitely many solutions, $(x,y,z)$ satisfying $y=2z$
        \item infinitely many solutions $(x,y,z)$ satisfying $x=2z$
        \item no solution
        \item only the trivial solution
        \end{enumerate}

    \item If \( p \to (p \land \neg q) \) is false, then the truth values of \( p \) and \( q \) are respectively: \hfill{(JAN 2020)}
        \begin{multicols}{4}
        \begin{enumerate}
            \item \( F, T \)  
            \item \( T, F \)  
            \item \( F, F \)  
            \item \( T, T \)
        \end{enumerate}
        \end{multicols}

    \item The length of minor axis (along y-axis) of an ellipse of the standard form is $\frac{4}{\sqrt{3}}$. If this ellipse touches the line $x+6y=8$, then its eccentricity is: \hfill{(JAN 2020)}

        \begin{enumerate}
        \item $\frac{1}{2}\left(\frac{\sqrt{5}}{3}\right)$
        \item $\frac{1}{2}\sqrt{\frac{11}{3}}$
        \item $\sqrt{\frac{5}{6}}$
        \item $\frac{1}{3}\sqrt{\frac{11}{3}}$
        \end{enumerate}  

    \item If $z$ be a complex number satisfying $|\text{Re}(z)|+|\text{Im}(z)|=4$, then $|z|$ cannot be: \hfill{(JAN 2020)}

        \begin{enumerate}
        \item $\sqrt{7}$
        \item $\sqrt{\frac{17}{2}}$
        \item $\sqrt{10}$
        \item $\sqrt{8}$
        \end{enumerate}

    \item If        
        \[x=\sum_{n=0}^{\infty}(-1)^{n}\tan^{2n}\theta\]  
        and
        \[y=\sum_{n=0}^{\infty}\cos^{2n}\theta\]
        where $0<\theta<\pi/4$, then: \hfill{(JAN 2020)}
        
        \begin{enumerate}
        \item $y(1+x)=1$
        \item $x(1-y)=1$
        \item $y(1-x)=1$
        \item $x(1+y)=1$
        \end{enumerate}

    


