\iffalse
    \title{2020}
    \author{EE24BTECH11029}
    \section{mcq-single}
\fi 

  \item The area of the region,enclosed by the circle $x^2+y^2=2$ which is not common to the region bounded by the parabola $y^2=x$ and straight line $y=x$, is
    \begin{enumerate}
        \item $\brak{\frac{1}{3}}\brak{12\pi-1}$
        \item $\brak{\frac{1}{6}}\brak{12\pi-1}$
        \item $\brak{\frac{1}{3}}\brak{6\pi-1}$
        \item $\brak{\frac{1}{6}}\brak{24\pi-1}$\\
    \end{enumerate}
    \item Total number of six-digit numbers in which only and all five digits $1,3,5,7$ and $9$ appear, is
    \begin{enumerate}
        \item $5^6$
        \item $\brak{\frac{1}{2}}\brak{6!}$
        \item $6!$
        \item $\brak{\frac{5}{2}}6!$\\
    \end{enumerate}
    \item An unbiased coin is tossed $5$ times. Suppose that a variable $X$ is assigned the value $k$ when $k$ consecutive heads are obtained for  $k= 3, 4, 5,$ otherwise $X$ takes the value $-1$. The expected value of $X$, is
    \begin{enumerate}
        \item $\frac{1}{8}$
        \item $\frac{3}{16}$
        \item $\frac{-1}{8}$
        \item $\frac{-3}{16}$\\
    \end{enumerate}
    \item If Re $\frac{\brak{z-1}}{\brak{2z+i}}=1,$ where $z=x+iy,$ then the point $\brak{x,y}$ lies on a 
    \begin{enumerate}
        \item circle whose centre is at $\brak{\frac{-1}{2},\frac{-3}{2}}$
        \item straight line whose slope is $\frac{3}{2}$
        \item circle whose diameter is $\frac{\sqrt{5}}{2}$
        \item straight line whose slope is $\frac{-2}{3}$\\
    \end{enumerate}
    \item If $f\brak{a+b+1-x}=f\brak{x}\forall{x},$ where a and b are fixed positive real numbers,then $\frac{1}{\brak{a+b}}\int_{a}^{b}x\brak{f\brak{x}+f\brak{x+1}}\,dx$
    is equal to
    \begin{enumerate}
        \item $\int_{a-1}^{b-1}f\brak{x}\,dx$
        \item $\int_{a+1}^{b+1}f\brak{x+1}\,dx$
        \item $\int_{a-1}^{b-1}f\brak{x+1}\,dx$
        \item  $\int_{a+1}^{b+1}f\brak{x}\,dx$\\
    \end{enumerate}
    \item  If the distance between the foci of an ellipse is $6$ and the distance between its directrices is $12,$ then the length of its latus rectum is
    \begin{enumerate}
        \item $2\sqrt{3}$
        \item $\sqrt{3}$
        \item $\frac{3}{\sqrt{2}}$
        \item $3\sqrt{2}$\\
    \end{enumerate}
    \item  The logical statement $\brak{p\implies q}\land \brak{q\implies\sim p}$ is equivalent to
    \begin{enumerate}
        \item $\sim p$
        \item $p$
        \item $q$
        \item $\sim q$\\
    \end{enumerate}
    \item Te greatest positive integer $k$, for which $49^k+1$ is a factor of the sum $49^{125} + 49^{124} + \dots+ 49^2 + 49 + 1,$ is
    \begin{enumerate}
        \item $32$
        \item $60$
        \item $65$
        \item $63$\\
    \end{enumerate}
    \item A vector $\vec{a} = \alpha\hat{i}+2\hat{j}+\beta\hat{k} \brak{\alpha,\beta \in \vec{R}}$ lies in the plane of the vectors,$\vec{b}=\hat{i}+\hat{j}$ and $\vec{c}=\hat{i}-\hat{j}+4\hat{k}.$If $\vec{a}$ bisects the angle between $\vec{b}$ and $\vec{c}$, then
    \begin{enumerate}
        \item $\vec{a}\hat{i}+3=0$
        \item $\vec{a}\hat{k}+4=0$
        \item $\vec{a}\hat{i}+1=0$
        \item $\vec{a}\hat{k}+2=0$\\
    \end{enumerate}
    \item If $y\brak{\alpha}=\sqrt{2\brak{\frac{\tan\alpha+\cot\alpha}{1+\tan^2\alpha}}+\frac{1}{\sin^2\alpha}}$ where $\alpha \in \brak{\frac{3\pi}{4},\pi}$ then $\frac{dy}{d\alpha}$ at $\alpha=\frac{5\pi}{6}$ is
    \begin{enumerate}
        \item $-\frac{1}{4}$
        \item $\frac{4}{3}$
        \item $4$
        \item $-4$\\
    \end{enumerate}
    \item $y = mx + 4$ is a tangent to both the parabolas, $y^2 = 4x$ and $x^2 = 2by$, then $b$ is equal to
    \begin{enumerate}
        \item $-64$
        \item $128$
        \item $-128$
        \item $-32$\\
    \end{enumerate}
    \item Let $\alpha$ be a root of the equation $x^2 + x+ 1= 0$ and the matrix $A=\frac{1}{\sqrt{3}}\myvec{1&1&1\\1&\alpha&\alpha^2\\1&\alpha^2&\alpha^4}$
    then the matrix $A^{31}$ is equal to
    \begin{enumerate}
        \item $A$
        \item $A^2$
        \item $A^3$
        \item $l_3$\\
   \end{enumerate}
    \item If $g\brak{x}=x^2+x-1$ and $\brak{gof}\brak{x}=4x^2-10x+5,$then $f\brak{\frac{5}{4}}$ is equal to
    \begin{enumerate}
        \item $\frac{-3}{2}$
        \item $\frac{-1}{2}$
        \item $\frac{1}{2}$
        \item $\frac{3}{2}$\\
    \end{enumerate}
    \item let $\alpha$ and $\beta$ are two real roots of the equation $\brak{k+1}\tan^2{x}-\sqrt{2}\lambda\tan{x}=1-k,$ where $\brak{k\neq-1}$ and are real numbers. If $\tan^2\brak{\alpha+\beta}=50,$ then value of $\lambda$ is 
    \begin{enumerate}
        \item $5\sqrt{2}$
        \item $10\sqrt{2}$
        \item $10$
        \item $5$\\
    \end{enumerate}
    \item  Let $P$ be a plane passing through the points $\brak{2, 1, 0}, \brak{4, 1, 1}$ and $\brak{5, 0, 1}$ and $R$ be any point $\brak{2, 1, 6}.$ Then the image of $R$ in the plane $P$ is:
    \begin{enumerate}
        \item $\brak{6,5,2}$
        \item $\brak{6,5,-2}$
        \item $\brak{4,3,2}$
        \item $\brak{3,4,-2}$\\
    \end{enumerate} 
