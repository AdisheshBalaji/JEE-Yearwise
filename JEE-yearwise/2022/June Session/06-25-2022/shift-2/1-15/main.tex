\iffalse
\title{2022}
\author{EE24BTECH11063}
\section{mcq-single}
\fi
\item Let $A=\{x\in R : |x+1| < 2\}$ and $B=\{x\in R : |x-1| \ge 2\}$. Then which one of the following statements is \textbf{NOT} true? \hfill{[June 2022]}
    \begin{enumerate}
    \begin{multicols}{2}
    \item $A-B=\brak{-1,1}$
    \columnbreak
    \item $B-A=R-\brak{-3,1}$
    \end{multicols}
    \begin{multicols}{2}
    \item $A \cap B=(-3,-1]$
    \item $A \cup B=R-[1,3)$
    \end{multicols}
        \end{enumerate}
        \bigskip
        \item Let $a,b \in R$ be such that the equation $ax^2-2bx+15=0$ has a repeated root $\alpha$. If $\alpha$ and $\beta$ are the roots of the equation $x^2-2bx+21=0$, then $\alpha^2+\beta^2$ is equal to: \hfill{[June 2022]}
        \begin{enumerate}
        \begin{multicols}{4}
            \item 37
            \item 58
            \item 68
            \item 92
            \end{multicols}
        \end{enumerate}
        \bigskip
\item Let $z_1$ and $z_2$ be two complex numbers such that  $\Bar{z_1}\;=i\Bar{z_2};$ and $arg\brak{\frac{z_1}{z_2}}=\pi$. Then \hfill{[June 2022]}
        \begin{enumerate}
        \begin{multicols}{2}
        \item $arg z_2=\frac{\pi}{4}$
        \columnbreak
        \item $arg z_2=-\frac{3\pi}{4}$
        \end{multicols}
        \begin{multicols}{2}
        \item $arg z_1=\frac{\pi}{4}$
        \item $arg z_1=-\frac{3\pi}{4}$
        \end{multicols}
        \end{enumerate}
        \bigskip
    \item The system of equations
    \begin{align*}
        -kx+3y-14z=25\\
        -15x+4y-kz=3\\
        -4x+y+3z=4
    \end{align*}
    is consistent for all k in the set  \hfill{[June 2022]}
    \begin{enumerate}
        \begin{multicols}{4}
        \item $R$
        \item $R-\{-11,13\}$
        \item $R-\{13\}$
        \item $R-\{-11,11\}$
        \end{multicols}
        \end{enumerate}
\bigskip
 
 \item $\lim \limits_{x\to 0}\brak{\tan^2{x}\brak{\brak{2\sin^2{x}+3\sin{x}+4}^{\frac{1}{2}}-\brak{\sin^2{x}+6\sin{x}+2}^{\frac{1}{2}}}}$ \hfill{[June 2022]}
 \begin{enumerate}
     \begin{multicols}{4}
         \item $\frac{1}{12}$
         \item $-\frac{1}{18}$
         \item $-\frac{1}{12}$
         \item $\frac{1}{6}$
     \end{multicols}
 \end{enumerate}
 \bigskip
 \item The area of the region enclosed between the parabolas $y^2=2x-1$ and $y^2=4x-3$ \hfill{[June 2022]}
 \begin{enumerate}
     \begin{multicols}{4}
         \item $\frac{1}{3}$
         \item $\frac{1}{6}$
         \item $\frac{2}{3}$
         \item $\frac{3}{4}$
     \end{multicols}
 \end{enumerate}
\bigskip
 \item The coefficient of $x^{101}$ in the expression 
 \begin{align*}
     \brak{5+x}^{500}+x\brak{5+x}^{499}+x^2\brak{5+x}^{498}+ \cdots + x^{500}, x>0
 \end{align*}
 is \hfill{[June 2022]}
 \begin{enumerate}
     \begin{multicols}{4}
         \item ${}^{501}C_{101}\brak{5}^{399}$
         \item ${}^{501}C_{101}\brak{5}^{400}$
         \item ${}^{501}C_{100}\brak{5}^{400}$
         \item ${}^{500}C_{101}\brak{5}^{399}$
     \end{multicols}
 \end{enumerate}
 \bigskip
 \item The sum $1\;+\;2.3\;+\;3.3^{2}+ \cdots +\;10.3^{9}$ is equal to : \hfill{[June 2022]}
 \begin{enumerate}
    \begin{multicols}{2}
    \item $\frac{2.3^{12}+10}{4}$
    \columnbreak
    \item $\frac{19.3^{10}+1}{4}$
    \end{multicols}
    \begin{multicols}{2}
    \item $5.3^{10}-2$
    \item $\frac{9.3^{10}+1}{2}$
    \end{multicols}
        \end{enumerate}
\bigskip
\item Let $P$ be the plane passing through the intersection of the planes $\overset{\rightarrow}{r}\cdot \brak{\hat{i}+3\hat{j}-\hat{k}}=5$ and $\overset{\rightarrow}{r}\cdot \brak{2\hat{i}-\hat{j}+\hat{k}}=3$, and the point \brak{2,1,-2}. Let the position vectors of the points $X$ and $Y$ be $\hat{i}-2\hat{j}+4\hat{k}$ and $5\hat{i}-\hat{j}+2\hat{k}$ respectively. Then the points \hfill{[June 2022]}
\begin{enumerate}
    \item $X$ and $X+Y$ are on the same side of $P$
    \item $Y$ and $Y-X$ are on the opposite sides of $P$
    \item $X$ and $Y$ are on the opposite sides of $P$
    \item $X+Y$ and $X-Y$ are the same side of $P$
\end{enumerate}
\bigskip
\item A circle touches both the y-axis and the line $x+y=0$. Then the locus of it's centre is : \hfill{[June 2022]}
\begin{enumerate}
\begin{multicols}{2}
\item $y=\sqrt{2}x$
\columnbreak
    \item $x=\sqrt{2}y$
\end{multicols}
\begin{multicols}{2}
\item $y^2-x^2=2xy$
\item $x^2 - y^2 =2xy$
\end{multicols}
\end{enumerate}
\bigskip
\item Water is being filled at the rate of 1 ${cm}^3/sec$ in a right circular conical vessel(vertex downwards) of height 35 cm and diameter 14 cm. When the height of the water level is 10cm, the rate (in ${cm}^2/sec$) at which the wet conical surface area of the vessel increase is \hfill{[June 2022]}
\begin{enumerate}
\begin{multicols}{4}
\item 5
\item $\frac{\sqrt{21}}{5}$
\item $\frac{\sqrt{26}}{5}$
\item $\frac{\sqrt{26}}{10}$
\end{multicols}
\end{enumerate}
\bigskip
\item If $b_n = \int_{0}^{\frac{\pi}{2}} \frac{\cos^2{nx}}{\sin{x}}\,dx\;n \in N$, then \hfill{[June 2022]}
\begin{enumerate}
    \item $b_3-b_2,b_4-b_3,b_5-b_4$ are in A.P. with common difference -2
    \item $\frac{1}{b_3-b_2},\frac{1}{b_4-b_3},\frac{1}{b_5-b_4}$ are in A.P. with common difference 2
    \item $b_3-b_2,b_4-b_3,b_5-b_4$ are in G.P.
    \item $\frac{1}{b_3-b_2},\frac{1}{b_4-b_3},\frac{1}{b_5-b_4}$ are in A.P. with common difference -2
\end{enumerate}
\bigskip
\item If $y=y\brak{x}$ is the solution of the differential equation $2x^2\frac{dy}{dx}-2xy+3y^2=0$ such that $y\brak{e}=\frac{e}{3}$, then y\brak{1} is equal to \hfill{[June 2022]}
\begin{enumerate}
    \begin{multicols}{4}
    \item $\frac{1}{3}$
       \item $\frac{2}{3}$
       \item $\frac{3}{2}$
       \item 3
    \end{multicols}
\end{enumerate}
\bigskip
\item If the angle made by the tangent at the point $\brak{x_0,y_0}$ on the curve 
\begin{align*}
    x=12\brak{t+\sin{t}\cos{t}},\\
    y=12\brak{1+\sin{t}}^2, 0<t<\frac{\pi}{2}
\end{align*}
with the positive x-axis is $\frac{\pi}{3}$, then $y_0$ is equal to: \hfill{[June 2022]}
\begin{enumerate}

    \begin{multicols}{2}
        \item $6\brak{3+2\sqrt{2}}$
        \columnbreak
        \item $3\brak{7+4\sqrt{3}}$
    \end{multicols}
    \begin{multicols}{2}
        \item 27
        \item 48
    \end{multicols}
\end{enumerate}
\bigskip
\item The value of $2\sin{\brak{12\degree}}-\sin{\brak{72\degree}}$ is: \hfill{[June 2022]}
\begin{enumerate}
    \begin{multicols}{2}
    \item $\frac{\sqrt{5}\brak{1-\sqrt{3}}}{4}$
        \columnbreak
        \item $\frac{1-\sqrt{5}}{8}$
    \end{multicols}
    \begin{multicols}{2}
    \item $\frac{\sqrt{3}\brak{1-\sqrt{5}}}{2}$
        \item $\frac{\sqrt{3}\brak{1-\sqrt{5}}}{4}$
    \end{multicols}
\end{enumerate}
