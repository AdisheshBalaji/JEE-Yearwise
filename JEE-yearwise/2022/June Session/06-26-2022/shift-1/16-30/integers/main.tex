\iffalse
   \title{Assignment}
   \author{EE24BTECH11034}
   \section{integer}
\fi 
    \item The sum of the cubes of all the roots of the equation $x^{4}-3x^{3}-2x^{2}+3x+1=10$ is:\hfill{June 2022}

     

    \item There are ten boys $B_1, B_2, \ldots, B_{10}$ and five girls $G_1, G_2, \ldots, G_5$ in a class. Then the number of ways of forming a group consisting of three boys and three girls, if both $B_1$ and $B_2$ together should not be the members of a group, is:\hfill{June 2022}

      

    \item Let the common tangents to the curves $4\brak{x^{2}+y^{2}}=9$ and $y^{2}=4x$ intersect at the point $Q$. Let an ellipse, centered at the origin $O$, has lengths of semi-minor and semi-major axes equal to $OQ$ and $6$, respectively. If $e$ and $l$ respectively denote the eccentricity and the length of the latusrectum of this ellipse, then $\frac{l}{e^{2}}$ is equal to:\hfill{June 2022}

      

    \item Let $f\brak{x}=\max\brak{\abs{x+1},\abs{x+2},\ldots,\abs{x+5}}$. Then $\int_{-6}^{0}f\brak{x}dx$ is equal to:\hfill{June 2022}


    \item Let the solution curve $y=y\brak{x}$ of the differential equation $\brak{4+x^{2}}dy-2x\brak{x^{2}+3y+4}dx=0$ pass through the origin. Then $y\brak{2}$ is equal to:\hfill{June 2022}

       

    \item If $\sin^{2}\brak{10^{\circ}}\sin\brak{20^{\circ}}\sin\brak{40^{\circ}}\sin\brak{50^{\circ}}\sin\brak{70^{\circ}}=\alpha-\frac{1}{16}\sin\brak{10^{\circ}}$, then $16+\alpha^{-1}$ is equal to:\hfill{June 2022}

    \item Let $A=\sbrak{n\in\mathbb{N}:\text{H.C.F.}\brak{n,45}=1}$ and $B=\sbrak{2k:k\in\sbrak{1,2,\ldots,100}}$. Then the sum of all the elements of $A\cap B$ is:\hfill{June 2022}

        
    \item The value of the integral $\frac{48}{\pi^{4}}\int_{0}^{\pi}\brak{\frac{3\pi x^{2}}{2}-x^{3}}\frac{\sin x}{1+\cos^{2}x}dx$ is equal to:\hfill{June 2022}
    
       
        
    \item Let $A=\sum_{i=1}^{10}\sum_{j=1}^{10}\min\brak{i,j}$ and $B=\sum_{i=1}^{10}\sum_{j=1}^{10}\max\brak{i,j}$. Then $A+B$ is equal to:\hfill{June 2022}
    
      
        
    \item Let $S=\left(0,2\pi\right)-\sbrak{\frac{\pi}{2},\frac{3\pi}{4},\frac{3\pi}{2},\frac{7\pi}{4}}$. Let $y=y\brak{x}$, $x\in S$, be the solution curve of the differential equation $\frac{dy}{dx}=\frac{1}{1+\sin 2x}$, $y\brak{\frac{\pi}{4}}=\frac{1}{2}$. If the sum of abscissas of all the points of intersection of the curve $y=y\brak{x}$ with the curve $y=\sqrt{2}\sin x$ is $\frac{k\pi}{12}$, then $k$ is equal to:\hfill{June 2022}
