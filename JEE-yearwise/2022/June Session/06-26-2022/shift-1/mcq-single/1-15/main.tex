\iffalse
\title{Assignment}
\author{EE24BTECH11038}
\section{mcq-single}
\fi
\item Let $f\brak{x}=\frac{x-1}{x+1},x\in \mathbf{R}-\cbrak{0,-1,1}$. If $f^{n+1}\brak{x}=f\brak{f^{n}\brak{x}}$ for all $n\in\mathbf{N}$, then $f^6\brak{6}+f^7\brak{7}$ is equal to :\hfill{June 2022}
\begin{enumerate}
    \item $\frac{7}{6}$
    \item $-\frac{3}{2}$
    \item $\frac{7}{12}$
    \item $-\frac{11}{12}$
\end{enumerate}
\bigskip
\item Let A=$\cbrak{z\in \mathbf{C}:\abs{\frac{z+1}{z-1}<1}}$ and B=$\cbrak{z\in \mathbf{C}:arg \brak{\frac{z-1}{z+1}}=\frac{2\pi}{3}}$ then $A\cap B$ is \hfill{June 2022}
\begin{enumerate}
    \item a portion of circle centred at $\brak{0,-\frac{1}{\sqrt{3}}}$ that lies in the second and third quadrants only
    \item a portion of circle centred at $\brak{0,-\frac{1}{\sqrt{3}}}$ that lies in the second quadrant only
    \item an empty set
    \item a portion of circle of radius $\frac{2}{\sqrt{3}}$ that lies in the third quadrant only
\end{enumerate}
\bigskip
\item Let A be a 3$\times$3 invertible matrix. If $\abs{adj\brak{24A}}=\abs{adj\brak{3adj\brak{2A}}}$ then $\abs{A}^2$ is equal to\hfill{June 2022}
\begin{enumerate}
    \item $6^6$
    \item $2^{12}$
    \item $2^6$
    \item 1
\end{enumerate}
\bigskip
\item The order pair $\brak{a,b}$, for which the system of linear equations 
\begin{align*}
    3x-2y+z=b\\
    5x-8y+9z=3\\
    2x+y+az=-1
\end{align*}
has no slolution is :\hfill{June 2022}
\begin{enumerate}
    \item $\brak{3,\frac{1}{3}}$
    \item $\brak{-3,\frac{1}{3}}$
    \item $\brak{-3,-\frac{1}{3}}$
    \item $\brak{3,-\frac{1}{3}}$
\end{enumerate}
\bigskip
\item The remainder when $\brak{2021}^{2023}$ is divided by 7 is :\hfill{June 2022}
\begin{enumerate}
    \item 1
    \item 5
    \item 5
    \item 6
\end{enumerate}
\bigskip
\item $\lim_{x\to {\frac{1}{\sqrt{2}}}} \frac{\sin{\brak{\cos^{-1}{x}}}-x}{1-\tan{\brak{\cos^{-1}{x}}}}$ is equal to :\hfill{June 2022}
\begin{enumerate}
    \item $\sqrt{2}$
    \item $-\sqrt{2}$
    \item $\frac{1}{\sqrt{2}}$
    \item $-\frac{1}{\sqrt{2}}$
\end{enumerate}
\bigskip
\item Let f,g:$\mathbf{R}\rightarrow \mathbf{R}$ be two real valued functions defined as
\begin{align*}
    f\brak{x}=
    \begin{cases}
        -\abs{x+3} , \,\, &x<0\\
        e^x , &x\geq0
    \end{cases}\\ \\
    g\brak{x}=
    \begin{cases}
        x^2+k_1x ,\,\, &x<0\\
        4x+k_2, &x\geq0
    \end{cases}
\end{align*}
where $k_1,k_2$ are real constants. If $\brak{gof}$ is differentiable at x=0, then $\brak{gof}\brak{-4}+\brak{gof}{4}$ is equal to : \hfill{June 2022}
\begin{enumerate}
    \item $4\brak{e^4+1}$
    \item $2\brak{2e^4+1}$
    \item $4e^4$
    \item $2\brak{2e^4-1}$
\end{enumerate}
\bigskip
\item The sum of absolute minimum and absolute maximum values of the function$f\brak{x}=\abs{3x-x^2+2}-x$ in the interval $\sbrak{-1,2}$ \hfill{June 2022}
\begin{enumerate}
    \item $\frac{\sqrt{17}+3}{2}$
    \item $\frac{\sqrt{17}+5}{2}$
    \item 5
    \item $\frac{9-\sqrt{17}}{2}$
\end{enumerate}
\bigskip
\item Let S be set of all natural numbers , for which the line $\frac{x}{a}+\frac{y}{b}=2$ is a tangent to the curve $\brak{\frac{x}{a}}^{n}+\brak{\frac{y}{b}}^{n}=2$ at the point $\brak{a,b}$,ab$\neq$0. Then:\hfill{June 2022}
\begin{enumerate}
    \item $S=\phi$
    \item $n\brak{S}=1$
    \item S=$\cbrak{2k:k\in\mathbf{N}}$
    \item S=$\mathbf{N}$
\end{enumerate}
\bigskip
\item The area bounded by the curve y=$\abs{x^2-9}$ and the line y=3 is :\hfill{June 2022}
\begin{enumerate}
    \item $4\brak{2\sqrt{3}+\sqrt{6}-4}$
    \item $4\brak{4\sqrt{3}+\sqrt{6}-4}$
    \item $8\brak{4\sqrt{3}+3\sqrt{6}-9}$
    \item $4\brak{4\sqrt{3}+\sqrt{6}-9}$
\end{enumerate}
\bigskip
\item Let $\vec{R}$ be point $\brak{3,7}$ and let $\vec{P}$ and $\vec{Q}$ be two points on the line x+y=5 such that PQR is an equilateral triangle. Then the area of$\triangle PQR$ \hfill{June 2022}
\begin{enumerate}
    \item $\frac{25}{4\sqrt{3}}$
    \item $\frac{25\sqrt{3}}{2}$
    \item $\frac{25}{\sqrt{3}}$
    \item $\frac{25}{2\sqrt{3}}$
\end{enumerate}
\bigskip
\item Let C be a circle passing through the points $\vec{A}\brak{2,-1}$ and $\vec{B}\brak{3,4}$. The line segment AB is not a diameter of C. If r is  radius of C and its centre lies on the circle $\brak{x-5}^2+\brak{y-1}^2=\frac{13}{2}$, then $r^2$ is equal to:\hfill{June 2022}
\begin{enumerate}
    \item 32
    \item $\frac{65}{2}$
    \item $\frac{61}{2}$
    \item 30
\end{enumerate}
\bigskip
\item Let the normal at the point $\vec{P}$ on the parabola $Y^2=6X$ pass through the point $\brak{5,-8}$. If the tangent at $\vec{P}$ to the parabola intersects at its directrix at the point $\vec{Q}$ the the ordinate of point $\vec{Q}$\hfill{June 2022}
\begin{enumerate}
    \item -3
    \item -$\frac{9}{4}$
    \item $-\frac{5}{2}$
    \item -2
\end{enumerate}
\bigskip
\item if the two lines $l_1:\frac{x-2}{3}=\frac{y+1}{-2},z=2$ and $l_2:\frac{x-1}{1}=\frac{2y+3}{\alpha}=\frac{z+5}{2}$ are perpendicular then the angle between the lines $l_2$ and $l_3:\frac{1-x}{3}=\frac{2y-1}{-4}=\frac{z}{4} $ is :\hfill{June 2022}
\begin{enumerate}
    \item $\cos^{-1}{\frac{29}{4}}$
    \item $\sec^{-1}{\frac{29}{4}}$
    \item $\cos^{-1}{\frac{2}{29}}$
    \item $\cos^{-1}{\frac{2}{\sqrt{29}}}$
\end{enumerate}
\bigskip
\item  Let the plane 2x+3y+z+20=0 be rotated through a right angle about its line of intersection with the plane x-3y+5z=8. If the mirror image of the point $\brak{2,-\frac{1}{2},2}$ in the rotated plane is $\vec{B}\brak{a,b,c}$ then :\hfill{June 2022}
\begin{enumerate}
    \item $\frac{a}{8}=\frac{b}{5}=\frac{c}{-4}$
    \item $\frac{a}{4}=\frac{b}{5}=\frac{c}{-2}$
    \item $\frac{a}{8}=\frac{b}{-5}=\frac{c}{4}$
    \item $\frac{a}{8}=\frac{b}{5}=\frac{c}{2}$
\end{enumerate}
