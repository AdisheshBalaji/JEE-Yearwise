\iffalse
\title{2022}
\author{EE24Btech11024}
\section{mcq-single}
\fi

\item Let $f:\mathbb{R}\rightarrow\mathbb{R}$ be defined as $f\brak{x}=x-1$ and $g:\mathbb{R}-\cbrak{-1,1}\rightarrow\mathbb{R}$ be defined as $g\brak{x}=\frac{x^2}{x^2-1}$. Then the function $f\circ g$ is:

\hfill{\brak{\text{Jun 2022}}}
\begin{enumerate}
\begin{multicols}{2}
\item one-one but not onto function
\item onto but not one-one function
\item both one-one and onto function
\item neither one-one nor onto function
\end{multicols}
\end{enumerate}


\item If the system of equations $\alpha x+y+z=5$, $x+2y+3z=4$, $x+3y+5z=\beta$, has infinitely many solutions, then the ordered pair $\brak{\alpha,\beta}$ is equal to:

\hfill{\brak{\text{Jun 2022}}}
\begin{enumerate}
\begin{multicols}{4}
\item $\brak{1,-3}$
\item $\brak{-1,3}$
\item $\brak{1,3}$
\item $\brak{-1,-3}$
\end{multicols}
\end{enumerate}

\item If $A = \sum_{n=1}^{\infty}\frac{1}{\brak{3+\brak{-1}^n}^n}$ and $B = \sum_{n=1}^{\infty}\frac{\brak{-1}^n}{\brak{3+\brak{-1}^n}^n}$, then $\frac{A}{B}$ is equal to:

\hfill{\brak{\text{Jun 2022}}}
\begin{enumerate}
\begin{multicols}{4}
\item $\frac{11}{9}$
\item $1$
\item $-\frac{11}{9}$
\item $-\frac{11}{3}$
\end{multicols}
\end{enumerate}

\item $\lim_{x\to 0}\frac{\cos\brak{\sin x}-\cos x}{x^4}$ is equal to:

\hfill{\brak{\text{Jun 2022}}}
\begin{enumerate}
\begin{multicols}{4}
\item $\frac{1}{3}$
\item $\frac{1}{4}$
\item $\frac{1}{6}$
\item $\frac{1}{12}$
\end{multicols}
\end{enumerate}

\item Let $f\brak{x}=\min\cbrak{1,1+x\sin x}$, $0\leq x\leq 2\pi$. If $m$ is the number of points where $f$ is not differentiable and $n$ is the number of points where $f$ is not continuous, then the ordered pair $\brak{m,n}$ is equal to

\hfill{\brak{\text{Jun 2022}}}
\begin{enumerate}
\begin{multicols}{4}
\item $\brak{2,0}$
\item $\brak{1,0}$
\item $\brak{1,1}$
\item $\brak{2,1}$
\end{multicols}
\end{enumerate}

\item Consider a cuboid of sides $2x$, $4x$ and $5x$ and a closed hemisphere of radius $r$. If the sum of their surface areas is constant $k$, then the ratio $x:r$, for which the sum of their volumes is maximum is 

\hfill{\brak{\text{Jun 2022}}}
\begin{enumerate}
\begin{multicols}{4}
\item $2:5$
\item $19:45$
\item $3:8$
\item $19:15$
\end{multicols}
\end{enumerate}

\item The area of region bounded by $y^2=8x$ and $y^2=16\brak{3-x}$ is equal to:

\hfill{\brak{\text{Jun 2022}}}
\begin{enumerate}
\begin{multicols}{4}
\item $\frac{32}{3}$
\item $\frac{40}{3}$
\item $16$
\item $19$
\end{multicols}
\end{enumerate}

\item If $\int\frac{1}{x}\sqrt{\frac{1-x}{1+x}}dx=g\brak{x}+c$, $g\brak{1}=0$, then $g\brak{\frac{1}{2}}$ is equal to:

\hfill{\brak{\text{Jun 2022}}}
\begin{enumerate}
\begin{multicols}{4}
\item $\log_e{\brak{\frac{\sqrt{3}-1}{\sqrt{3}+1}}}+\frac{\pi}{3}$
\item $\log_e{\brak{\frac{\sqrt{3}+1}{\sqrt{3}-1}}}+\frac{\pi}{3}$
\item $\log_e{\brak{\frac{\sqrt{3}+1}{\sqrt{3}-1}}}-\frac{\pi}{3}$
\item $\log_e{\brak{\frac{\sqrt{3}-1}{\sqrt{3}+1}}}-\frac{\pi}{6}$
\end{multicols}
\end{enumerate}

\item If $y=y\brak{x}$ is the solution of the differential equation $x\frac{dy}{dx}+2y=xe^x$, $y\brak{1}=0$ then the local maximum value of the function $z\brak{x}=x^2 y\brak{x} - e^x$, $x \in \mathbb{R}$ is:

\hfill{\brak{\text{Jun 2022}}}
\begin{enumerate}
\begin{multicols}{4}
\item $1-e$
\item $0$
\item $\frac{1}{2}$
\item $\frac{4}{e}-e$
\end{multicols}
\end{enumerate}

\item If the solution to the differential equation $\frac{dy}{dx}+e^{x}\brak{x^2-2}y=\brak{x^2-2x}\brak{x^2-2}e^{2x}$ satisfies $y\brak{0}=0$, then the value of $y\brak{2}$ is \rule{1cm}{0.15mm}.

\hfill{\brak{\text{Jun 2022}}}
\begin{enumerate}
\begin{multicols}{4}
\item $-1$
\item $1$
\item $0$
\item $e$
\end{multicols}
\end{enumerate}

\item If $m$ is the slope of a common tangent to curves $\frac{x^2}{16}+\frac{y^2}{9}=1$ and $x^2+y^2=12$, then $12m^2$ is equal to:

\hfill{\brak{\text{Jun 2022}}}
\begin{enumerate}
\begin{multicols}{4}
\item $6$
\item $9$
\item $10$
\item $12$
\end{multicols}
\end{enumerate}

\item The locus of the mid point of the line segment joining the point \brak{4,3} and the points on the ellipse $x^2+2y^2=4$ is an ellipse with eccentricity:

\hfill{\brak{\text{Jun 2022}}}
\begin{enumerate}
\begin{multicols}{4}
\item $\frac{\sqrt{3}}{2}$
\item $\frac{1}{2\sqrt{2}}$
\item $\frac{1}{\sqrt{2}}$
\item $\frac{1}{2}$
\end{multicols}
\end{enumerate}

\item The normal to the hyperbola $\frac{x^2}{a^2}-\frac{y^2}{9}=1$ at the point \brak{8,3\sqrt{3}} on it passes through the point:

\hfill{\brak{\text{Jun 2022}}}
\begin{enumerate}
\begin{multicols}{4}
\item \brak{15,-2\sqrt{3}}
\item \brak{9,2\sqrt{3}}
\item \brak{-1,9\sqrt{3}}
\item \brak{-1,6\sqrt{3}}
\end{multicols}
\end{enumerate}

\item If the plane $2x+y-5z=0$ is rotated about its line of intersection with the plane $3x-y+4z-7=0$ by an angle $\frac{\pi}{2}$. then the plane after the rotation passes through the point:

\hfill{\brak{\text{Jun 2022}}}
\begin{enumerate}
\begin{multicols}{4}
\item $\brak{2,-2,0}$
\item $\brak{-2,2,0}$
\item $\brak{1,0,2}$
\item $\brak{-1,0,-2}$
\end{multicols}
\end{enumerate}

\item If the lines $\vec{r}=\brak{\hat{i}-\hat{j}+\hat{k}}+\lambda\brak{3\hat{j}-\hat{k}}$ and $\vec{r}=\brak{\alpha\hat{i}-\hat{j}}+\mu\brak{2\hat{i}-3\hat{k}}$ are co-planar, then the distance of the plane containing these two lines from the point \brak{\alpha,0,0} is:

\hfill{\brak{\text{Jun 2022}}}
\begin{enumerate}
\begin{multicols}{4}
\item $\frac{2}{9}$
\item $\frac{2}{11}$
\item $\frac{4}{11}$
\item $2$
\end{multicols}
\end{enumerate}

