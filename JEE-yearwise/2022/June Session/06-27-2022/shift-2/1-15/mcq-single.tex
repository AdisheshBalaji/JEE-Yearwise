\iffalse
\let\negmedspace\undefined
\let\negthickspace\undefined
\documentclass[journal]{IEEEtran}
\usepackage[a5paper, margin=10mm, onecolumn]{geometry}
%\usepackage{lmodern} % Ensure lmodern is loaded for pdflatex
\usepackage{tfrupee} % Include tfrupee package

\setlength{\headheight}{1cm} % Set the height of the header box
\setlength{\headsep}{0mm}     % Set the distance between the header box and the top of the text

\usepackage{gvv-book}
\usepackage{gvv}
\usepackage{cite}
\usepackage{amsmath,amssymb,amsfonts,amsthm}
\usepackage{algorithmic}
\usepackage{graphicx}
\usepackage{textcomp}
\usepackage{xcolor}
\usepackage{txfonts}
\usepackage{listings}
\usepackage{enumitem}
\usepackage{mathtools}
\usepackage{gensymb}
\usepackage{comment}
\usepackage[breaklinks=true]{hyperref}
\usepackage{tkz-euclide} 
\usepackage{listings}
% \usepackage{gvv}                                        
\def\inputGnumericTable{}                                 
\usepackage[latin1]{inputenc}                                
\usepackage{color}                                            
\usepackage{array}                                            
\usepackage{longtable}                                       
\usepackage{calc}                                             
\usepackage{multirow}                                         
\usepackage{hhline}                                           
\usepackage{ifthen}                                           
\usepackage{lscape}
\bibliographystyle{IEEEtran}
\vspace{3cm}

\title{2022}
\author{EE24BTECH11061}
\maketitle

\renewcommand{\thefigure}{\theenumi}
\renewcommand{\thetable}{\theenumi}
\section{mcq-single}
\fi
%\begin{enumerate}
\item   The number of points of intersection of $\abs{z-\brak{4+3\iota}} = 2$ and $\abs{z} + \abs{z-4} = 6$, $z \in \mathbb{C}$ is:
\hfil{\sbrak{\text{June 2022}}}
\begin{multicols}{2}
    \begin{enumerate}
        \item 0
        \item 1
        \item 2
        \item 3
    \end{enumerate}
\end{multicols}

\item Let $f\brak{x} = \mydet{a & -1 & 0\\ax & a & -1\\ax^2 & ax & a}$, $a \in \mathbb{R}$. Then the sum of which the squares of all the values of $a$ for $2f^{\prime}\brak{10} - f^{\prime}\brak{5} + 100 =0$ is:
\hfil{\sbrak{\text{June 2022}}}
\begin{multicols}{2}
    \begin{enumerate}
        \item 117
        \item 106
        \item 125
        \item 136
    \end{enumerate}
\end{multicols}

\item Let for some real numbers $\alpha$ and $\beta$, $a = \alpha - \iota\beta$. If the system of equations $4\iota x + \brak{1+\iota}y = 0$ and $8\brak{\cos{\frac{2\pi}{3}} + \iota \sin{\frac{2\pi}{3}}}x + \overline{a} y = 0$ has more than one solution then $\frac{\alpha}{\beta}$ is equal to:
\hfil{\sbrak{\text{June 2022}}}
\begin{multicols}{2}
    \begin{enumerate}
        \item $-2+\sqrt{3}$
        \item $2-\sqrt{3}$
        \item $2+\sqrt{3}$
        \item $-2-\sqrt{3}$
    \end{enumerate}
\end{multicols}

\item Let A and B be two $3\times3$ matrices such that AB = I and $\det{A} = \frac{1}{8}$ then $\det\{adj\brak{Badj\brak{2A}}\}$ is equal to 
\hfil{\sbrak{\text{June 2022}}}
\begin{multicols}{2}
    \begin{enumerate}
        \item 16
        \item 32
        \item 64
        \item 128
    \end{enumerate}
\end{multicols}

\item Let S = $2+\frac{6}{7} + \frac{12}{7^2} + \frac{20}{7^3} + \frac{30}{7^4}+\dots$ then 4S is equal to
\hfil{\sbrak{\text{June 2022}}}
\begin{multicols}{2}
    \begin{enumerate}
        \item $\brak{\frac{7}{3}}^2$
        \item $\brak{\frac{7}{3}}^3$
        \item $\frac{7^3}{3^2}$
        \item $\frac{7^2}{3^3}$
    \end{enumerate}
\end{multicols}

\item If $a_1, a_2, a_3, \dots$ and $b_1, b_2, b_3, \dots$ are in A.P. and $a_1 = 2$, $a_{10} = 3$, $a_1b_1 = 1 = a_{10}b_{10}$ then $a_4b_4$ is equal to 
\hfil{\sbrak{\text{June 2022}}}
\begin{multicols}{2}
    \begin{enumerate}
        \item $\frac{35}{27}$
        \item $1$
        \item $\frac{27}{28}$
        \item $\frac{28}{27}$
    \end{enumerate}
\end{multicols}

\item If m and n respectively are the number of local maximum and local minimum points of the function $f\brak{x} = \int_0^{x^2} \frac{t^2-5t+4}{2+e^t}, dt$, then the ordered pair (m,n) is equal to
\hfil{\sbrak{\text{June 2022}}}
\begin{multicols}{2}
    \begin{enumerate}
        \item \brak{3,2}
        \item \brak{2,3}
        \item \brak{2,2}
        \item \brak{3,4}
    \end{enumerate}
\end{multicols}

\item Let $f$ be a differentiable function in $\brak{0,\frac{\pi}{2}}$. If $\int_{\cos{x}} ^{1} t^2f\brak{t}, dt = \sin^3{x} + \cos{x}$ then $\frac{1}{\sqrt{3}}f^{\prime}\brak{\frac{1}{\sqrt{3}}}$ is equal to:
\hfil{\sbrak{\text{June 2022}}}
\begin{multicols}{2}
    \begin{enumerate}
        \item $6-9\sqrt{2}$
        \item $6-\frac{9}{\sqrt{2}}$
        \item $\frac{9}{2}-6\sqrt{2}$
        \item $\frac{9}{\sqrt{2}} - 6$
    \end{enumerate}
\end{multicols}

\item The integral $\int_0^1 \frac{1}{7^{\sbrak{\frac{1}{x}}}}, dx$, is where $\sbrak{.}$ denotes the greatest integer function is equal to
\hfil{\sbrak{\text{June 2022}}}
\begin{multicols}{2}
    \begin{enumerate}
        \item $1+6\log_e{\frac{6}{7}}$
        \item $1-6\log_e{\frac{6}{7}}$
        \item $\log_e{\frac{7}{6}}$
        \item $1-7\log_e{\frac{6}{7}}$
    \end{enumerate}
\end{multicols}

\item If the solution curve of the differential equation $\brak{\brak{\tan^{-1}y}-x}dy = \brak{1+y^2}dx$ passes through the point $\brak{1,0}$ then the abscissa of the point on the curve whose ordinate is $\tan{1}$ is:
\hfil{\sbrak{\text{June 2022}}}
\begin{multicols}{2}
    \begin{enumerate}
        \item $2e$
        \item $\frac{2}{e}$
        \item $2$
        \item $\frac{1}{e}$
    \end{enumerate}
\end{multicols}

\item If the equation of the parabola, whose vertex is at $\brak{5,4}$ and the directrix is $3x+y-29=0$, id $x^2+ay^2+bxy+cx+dy+k=0$ then $a+b+c+d+k$ is equal to
\hfil{\sbrak{\text{June 2022}}}
\begin{multicols}{2}
    \begin{enumerate}
        \item 575
        \item -575
        \item 576
        \item -576
    \end{enumerate}
\end{multicols}

\item The set of values of $k$ for which the circle C $\colon$ $4x^2+4y^2-12x+8y+k=0$ lies inside the fourth quadrant and the point $\brak{1,-\frac{1}{3}}$ lies on or inside the circle C is:
\hfil{\sbrak{\text{June 2022}}}
\begin{multicols}{2}
    \begin{enumerate}
        \item An empty set
        \item $\lbrak{6}, \rsbrak{\frac{95}{9}}$
        \item $\lsbrak{\frac{80}{9}}, \rbrak{10}$
        \item $\lbrak{9}, \rsbrak{\frac{92}{9}}$
    \end{enumerate}
\end{multicols}

\item Let the foot of the perpendicular from the point \brak{1,2,4} on the line $\frac{x+2}{4} = \frac{y-1}{2} = \frac{z+1}{3}$ be P. Then the distance of P from the plane $3x+4y+12z+23=0$
\hfil{\sbrak{\text{June 2022}}}
\begin{multicols}{2}
    \begin{enumerate}
        \item 5
        \item $\frac{50}{13}$
        \item 4
        \item $\frac{63}{13}$
    \end{enumerate}
\end{multicols}

\item The shortest distance between the lines $\frac{x-3}{2} = \frac{y-2}{3} = \frac{z-1}{-1}$ and $\frac{x+3}{2} = \frac{y-6}{1} = \frac{z-5}{3}$ is:
\hfil{\sbrak{\text{June 2022}}}
\begin{multicols}{2}
    \begin{enumerate}
        \item $\frac{18}{\sqrt{5}}$
        \item $\frac{22}{3\sqrt{5}}$
        \item $\frac{46}{3\sqrt{5}}$
        \item $6\sqrt{3}$
    \end{enumerate}
\end{multicols}

\item Let $\vec{a}$ and $\vec{b}$ be the vectors along the diagonal of a parallelogram having area $2\sqrt{2}$. Let the angle between $\vec{a}$ and $\vec{b}$ be acute. $\abs{\vec{a}} = 1$ and $\abs{\vec{a} \cdot \vec{b}} = \abs{\vec{a} \times \vec{b}}$. If $\vec{c} = 2\sqrt{2}\brak{\vec{a} \times \vec{b}} - 2\vec{b}$, then an angle between $\vec{b}$ and $\vec{c}$ is:
\hfil{\sbrak{\text{June 2022}}}
\begin{multicols}{2}
    \begin{enumerate}
        \item $\frac{\pi}{4}$
        \item $-\frac{\pi}{4}$
        \item $\frac{5\pi}{6}$
        \item $\frac{3\pi}{4}$
    \end{enumerate}
\end{multicols}
%\end{enumerate}