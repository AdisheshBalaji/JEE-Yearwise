\iffalse
   \title{Assignment}
   \author{EE24BTECH11034}
   \section{mcq-single}
\fi 

    \item The minimum value of the sum of the squares of the roots of $x^2 + \brak{3-a}x + 1 = 2a$ is:\hfill{July 2022}
    
        \begin{multicols}{4}
        \begin{enumerate}
        \item $4$
        \item $5$
        \item $6$
        \item $8$
        \end{enumerate}
        \end{multicols}
        
    \item If $z = x + iy$ satisfies $\abs{z} - 2 = 0$ and $\abs{z - i} - \abs{z + 5i} = 0$, then\hfill{July 2022}

        \begin{multicols}{4}
        \begin{enumerate}
        \item $x + 2y - 4 = 0$
        \item $x^{2} + y - 4 = 0$
        \item $x + 2y + 4 = 0$
        \item $x^{2} - y + 3 = 0$
        \end{enumerate}
        \end{multicols}

    \item Let $A = \begin{bmatrix} 1\\ 1\\ 1 \end{bmatrix}$ and $B = \begin{bmatrix} 9^{2} & -10^{2} & 11^{2} \\ 12^{2} & 13^{2} - 14^{2} \\ -15^{2} & 16^{2} & 17^{2} \end{bmatrix}$, then the value of $A^{T} B A$ is:\hfill{July 2022}

        \begin{multicols}{4}
        \begin{enumerate}
        \item $1224$
        \item $1042$
        \item $540$
        \item $539$
        \end{enumerate}
        \end{multicols}
        
    \item $\sum_{i,j=0}^{n} {n}C_{i} {n}C_{j}$ is equal to\hfill{July 2022}
 
        \begin{enumerate}
        \item $2^{2n} - {2n}C_{n}$
        \item $2^{2n-1} - {2n-1}C_{n-1}$
        \item $2^{2n} - \frac{1}{2} {2n}C_{n}$
        \item $2^{n-1} + {2n-1}C_{n}$
        \end{enumerate}

    \item Let $P$ and $Q$ be any points on the curves $\brak{x - 1}^{2} + \brak{y + 1}^{2} = 1$ and $y = x^{2}$, respectively. The distance between $P$ and $Q$ is minimum for some value of the abscissa of $P$ in the interval \hfill{July 2022}

        \begin{multicols}{4}
        \begin{enumerate}
        \item $\brak{0, \frac{1}{4}}$
        \item $\brak{\frac{1}{2}, \frac{3}{4}}$
        \item $\brak{\frac{1}{4}, \frac{1}{2}}$
        \item $\brak{\frac{3}{4}, 1}$
        \end{enumerate}
        \end{multicols}

    \item If the maximum value of $a$, for which the function $f_{a} \brak{x} = \tan^{-1} 2x - 3ax + 7$ is non-decreasing in $\brak{-\frac{\pi}{6}, \frac{\pi}{6}}$, is $\overline{a}$, then $f_{\overline{a}} \brak{\frac{\pi}{8}}$ is equal to: \hfill{July 2022}

        \begin{multicols}{4}
        \begin{enumerate}
        \item $8 - \frac{9\pi}{4\brak{9 + \pi^{2}}}$
        \item $8 - \frac{4\pi}{9\brak{4 + \pi^{2}}}$
        \item $8 \brak{\frac{1 + \pi^{2}}{9 + \pi^{2}}}$
        \item $8 - \frac{\pi}{4}$
        \end{enumerate}
        \end{multicols}

    \item Let $\beta = \lim_{x \rightarrow 0} \frac{\alpha x \brak{e^{-3x} - 1}}{\alpha x \brak{e^{3x} - 1}}$ for some $\alpha \in \mathbb{R}$. Then the value of $\alpha + \beta$ is:\hfill{July 2022}

        \begin{multicols}{4}
        \begin{enumerate}
        \item $\frac{14}{5}$
        \item $\frac{3}{2}$
        \item $\frac{5}{2}$
        \item $\frac{7}{2}$
        \end{enumerate}
        \end{multicols}

   \item The value of $\log_{e} 2 \frac{d}{dx} \brak{\log_{\cos{x}} \csc{x}}$ at $x = \frac{\pi}{4}$ is \hfill{July 2022}

        \begin{multicols}{4}
        \begin{enumerate}
        \item $-2\sqrt{2}$
        \item $2\sqrt{2}$
        \item $-4$
        \item $4$
        \end{enumerate}
        \end{multicols}

    \item $\int_{0}^{20\pi} \brak{\abs{\sin{x}} + \abs{\cos{x}}}^{2} dx$ is equal to:\hfill{July 2022}

        \begin{multicols}{4}
        \begin{enumerate}
        \item $10 \brak{\pi + 4}$
        \item $10 \brak{\pi + 2}$
        \item $20 \brak{\pi - 2}$
        \item $20 \brak{\pi + 2}$
        \end{enumerate}
        \end{multicols}

    \item Let the solution curve $y = f \brak{x}$ of the differential equation $\frac{dy}{dx} + \frac{xy}{x^{2} - 1} = \frac{x^{4} + 2x}{\sqrt{1 - x^{2}}}, x \in \brak{-1, 1}$ pass through the origin. Then $\int_{\frac{\sqrt{5}}{2}}^{1} f \brak{x} dx$ is equal to:\hfill{July 2022}

        \begin{multicols}{4}
        \begin{enumerate}
        \item $\frac{\pi}{3} - \frac{1}{4}$
        \item $\frac{\pi}{3} - \frac{\sqrt{3}}{4}$
        \item $\frac{\pi}{6} - \frac{\sqrt{3}}{4}$
        \item $\frac{\pi}{6} - \frac{\sqrt{3}}{2}$
        \end{enumerate}
        \end{multicols}
                        
    \item The acute angle between the pair of tangents drawn to the ellipse $2x^{2} + 3y^{2} = 5$ from the point $\brak{1, 3}$ is\hfill{July 2022}

        \begin{multicols}{4}
        \begin{enumerate}
        \item $\tan^{-1} \brak{\frac{16}{7\sqrt{5}}}$
        \item $\tan^{-1} \brak{\frac{24}{7\sqrt{5}}}$
        \item $\tan^{-1} \brak{\frac{32}{7\sqrt{5}}}$
        \item $\tan^{-1} \brak{\frac{3 + 8\sqrt{5}}{35}}$
        \end{enumerate}
        \end{multicols}

    \item The equation of a common tangent to the parabolas $y = x^{2}$ and $y = -\brak{x - 2}^{2}$ is \hfill{July 2022}

        \begin{multicols}{4}
        \begin{enumerate}
        \item $y = 4 \brak{x - 2}$
        \item $y = 4 \brak{x - 1}$
        \item $y = 4 \brak{x + 1}$
        \item $y = 4 \brak{x + 2}$
        \end{enumerate}
        \end{multicols}
        
    \item Let the abscissae of the two points $P$ and $Q$ on a circle be the roots of $x^{2} - 4x - 6 = 0$ and the ordinates of $P$ and $Q$ be the roots of $y^{2} + 2y - 7 = 0$. If $PQ$ is a diameter of the circle $x^{2} + y^{2} + 2ax + 2by + c = 0$, then the value of $a + b - c$ is \hfill{July 2022}

        \begin{multicols}{4}
        \begin{enumerate}
        \item $12$
        \item $13$
        \item $14$
        \item $16$
        \end{enumerate}
        \end{multicols}

    \item If the line $x - 1 = 0$ is a directrix of the hyperbola $kx^{2} - y^{2} = 6$, then the hyperbola passes through the point \hfill{July 2022}

        \begin{multicols}{4}
        \begin{enumerate}
        \item $\brak{-2\sqrt{5}, 6}$
        \item $\brak{-\sqrt{5}, 3}$
        \item $\brak{\sqrt{5}, -2}$
        \item $\brak{2\sqrt{5}, 3\sqrt{6}}$
        \end{enumerate}
        \end{multicols}
        

    \item A vector $\vec{a}$ is parallel to the line of intersection of the plane determined by the vectors $\hat{i}, \hat{i} + \hat{j}$ and the plane determined by the vectors $\hat{i} - \hat{j}, \hat{i} + \hat{k}$. The obtuse angle between $\vec{a}$ and the vector $\vec{b} = \hat{i} - 2\hat{j} + 2\hat{k}$ is \hfill{July 2022}

        \begin{multicols}{4}
        \begin{enumerate}
        \item $\frac{3\pi}{4}$
        \item $\frac{2\pi}{3}$
        \item $\frac{4\pi}{5}$
        \item $\frac{5\pi}{6}$
        \end{enumerate}
        \end{multicols}

    

