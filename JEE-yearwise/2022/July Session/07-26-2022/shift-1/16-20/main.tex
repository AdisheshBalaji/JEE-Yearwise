\iffalse
    \title{2022}
    \author{EE24BTECH11029}
    \section{mcq-single}
\fi 

\item The mean and variance of a binomial distribution are $\alpha$ and $\frac{\alpha}{3}$ respectively. If $P\brak{X=1}=\frac{4}{243},$ then $P\brak{X=4 or 5}$ is equal to:
    \begin{enumerate}
        \item $\frac{5}{9}$
        \item $\frac{64}{81}$
        \item $\frac{16}{27}$
        \item $\frac{145}{243}$\\
    \end{enumerate}
    \item Let $E_1,E_2,E_3$ be three mutually exclusive events such that $P\brak{E_1}=\frac{2+3p}{6},P\brak{E_2}=\frac{2-p}{8}$ and $P\brak{E_3}=\frac{1-p}{2}.$ If the maximum and minimum values of $p$ are $p_1$ and $p_2,$ then $\brak{p_1+p_2}$ is equal to :
    \begin{enumerate}
        \item $\frac{2}{3}$
        \item $\frac{5}{3}$
        \item $\frac{5}{4}$
        \item $1$\\
    \end{enumerate}
    \item Let $S\{\theta\in\sbrak{0,2\pi};8^{2\sin^2\theta}+8^{2\cos^2\theta}=16\}.$ Then $n\brak{S}+\sum_{\theta\in S}\brak{\sec\brak{\frac{\pi}{4}+2\theta}\cosec\brak{\frac{\pi}{4}+2\theta}}$ is equal to :
    \begin{enumerate}
        \item $0$
        \item $-2$
        \item $-4$
        \item $12$\\
    \end{enumerate}
    \item $\tan\brak{2\tan^{-1}\frac{1}{5}+\sec^{-1}\frac{\sqrt{5}}{2}+2\tan^{-1}\frac{1}{8}}$ is equal to:
    \begin{enumerate}
        \item $1$
        \item $2$
        \item $\frac{1}{4}$
        \item $\frac{5}{4}$\\
    \end{enumerate}
    \item The statement $\brak{\sim\brak{p\Leftrightarrow\sim q}}\land q$ is:
    \begin{enumerate}
        \item a tautology
        \item a contradiction
        \item equivalent to $\brak{p\implies q}\land q$
        \item equivalent to $\brak{p\implies q}\land p$\\
    \end{enumerate}
