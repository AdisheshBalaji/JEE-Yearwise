\iffalse
  \title{2022}
  \author{ai24btech11005}
  \section{mcq-single}
\fi


%16
    \item Let $X$ have  a binomial distribution $B\brak{n,p}$ such that the sum and product of the mean and variance of $X$  are $24$ and $128$ respectively. If P$\brak{X>n-3}=\frac{k}{2^n}$
\hfill{[July 2022]}
        \begin{multicols}{4}
            \begin{enumerate}
                \item 528
                \item 529
                \item 629
                \item 630
            \end{enumerate}
        \end{multicols}

%17
    \item A six faced die is biased such that $3\times P\brak{\text{a prime number}}=2\times P\brak{1}=6\times P\brak{\text{a composite number}}.$ Let $X$ be a random variable that counts the number of times one gets a perfect square on some throws of this die. If the die is thrown twice, then the mean of $X$ is :
\hfill{[July 2022]}
		\begin{multicols}{1}
			\begin{enumerate}
				\item $\frac{3}{11}$
				\item $\frac{5}{11}$
				\item $\frac{7}{11}$
				\item $\frac{8}{11}$
			\end{enumerate}
		\end{multicols}

%18
    \item The angle of elevation of the top P of a vertical tower $PQ$ of height 10 from point A on the horizontal ground is 45\textdegree. Let R be a point on $AQ$ and from a point B, the angle of elevation of P is 60\textdegree. If $\angle BAQ$=30\textdegree, $AB=d$ and the area of the trapezium PQRB is $\alpha$, then the ordered pair $\brak{d,\alpha}$ is :
 \hfill{[July 2022]}
        \begin{multicols}{1}
            \begin{enumerate}
                \item $\brak{10\brak{\sqrt{3}-1},25}$
                \item $\brak{10\brak{\sqrt{3}-1},\frac{25}{2}}$
                \item $\brak{10\brak{\sqrt{3}+1},25}$
                \item $\brak{10\brak{\sqrt{3}+1},\frac{25}{2}}$
            \end{enumerate}
        \end{multicols}

%19
    \item Let $S=\{\theta \in \brak{0,\frac{\pi}{2}} :\sum_{m=1}^{9}\sec \brak{\theta+\brak{m-1}\frac{\pi}{6}}\sec \brak{\theta+\frac{m\pi}{6}}\}$
 \hfill{[July 2022]}
		\begin{multicols}{1}
			\begin{enumerate}
				\item $\{\frac{\pi}{6}\}$
				\item $\{\frac{2\pi}{3}\}$
				\item $\sum_{\theta \in S}\theta =\frac{\pi}{2}$
				\item $\sum_{\theta \in S}\theta =\frac{3\pi}{4}$
			\end{enumerate}
		\end{multicols}

%20
    \item If the truth value of the statement $(P\land(\neg R))\rightarrow ((\neg R) \land Q)$ is F, then the truth value of which is of the following is F ?
    \hfill{[July 2022]}
		\begin{multicols}{1}
			\begin{enumerate}
				\item $P \lor Q \rightarrow \neg R$
				\item $R \lor Q \rightarrow \neg P$
				\item $\neg (P \lor Q) \rightarrow \neg R$
				\item $\neg (R \lor Q) \rightarrow \neg P$
			\end{enumerate}
		\end{multicols}


