\iffalse
  \title{2022}
  \author{ai24btech11005}
  \section{integer}
\fi
  \item $A= \myvec{\alpha && \beta && \gamma \\ \alpha^2 && \beta^2 && \gamma^2 \\ \alpha + \beta && \beta +\gamma && \gamma +\alpha}$ . Where $\alpha, \beta,\gamma$ are three distinct natural numbers.If $\frac{det(adj(adj(adj(adjA))))}{(\alpha - \beta)^{16} (\beta - \gamma)^{16} (\gamma -\alpha)^{16}} =2^{32} X 3^{16}$.  Then the number of such 3-tuples $(\alpha , \beta, \gamma)$ is .

    \hfill{[July 2022]}
    \item The number of functions f, from the set $A= \{x \in N: x^2-10x+9 \leq 0 \}$ to the set
    $B= \{ n^2 : n\in N\}$ such that $f\brak{x}\leq
     \brak{x-3}^2+1,$ for every $x \in A, is $
   \hfill{[July 2022]}
%23
    \item Let for the $9^{th}$ term in the binomial expansion of $\brak{3+6x}^n$, in the increasing powers of $6x$, to be the greatest for $x=\frac{3}{2}$, the least value of n is $n_0$.  If $K$ 
    is the ratio of the coefficient of $x^6$ to the coefficient of $x^3$, then $k+n_0$ is equal to:
 \hfill{[July 2022]}

%24
    \item $\frac{2^3-1^3}{1\times7} + \frac{4^3-3^3+2^3-1^3}{2\times 11} +......+\frac{6^3-5^3+4^3-3^3+2^3-1^3}{3\times 15} +\frac{30^3-29^3+28^3-27^3+.....+2^3-1^3}{15\times 63}$ is equal to .
\hfill{[July 2022]}
%25
    \item A water tank has the shape of a right circular cone with axis vertical and vertex downwards. Its semi-vertical angle is $\tan^{-1}\frac{3}{4}$. Water is poured in at a constant rate of 6 cubic  meter per hour. The rate, at which wet curved surface area of the tank is increasing, when the depth of the tank is 4 meters, is:
   \hfill{[July 2022]}

    \item For the curve $C: \brak{x^2+y^2-3}+\brak{x^2-y^2-1}^5=0$, the value of $3y'-y^3y''$, at the point \brak{\alpha,\alpha
    }, $\alpha \geq 0$, on C, is equal to  :
 \hfill{[July 2022]}
    \item Let $f\brak{x}=min\{[x-1],[x-2],......,[x-10]\}$ where $[t]$ denotes the greatest integer $\leq t$. Then $\int_{0}^{10}f\brak{x}dx$+$\int_{0}^{10}(f\brak{x})^2dx$+$\int_{0}^{10}|f\brak{x}|dx$ is equal to :
\hfill{[July 2022]}
    \item Let f be a differentiable function satisfying $f\brak{x}=\frac{2}{\sqrt{3}}\int_{0}^{\sqrt{3}}  f\brak{\frac{\lambda^2x}{3}}d\lambda$, $x\geq0$ and $f\brak{1}=\sqrt{3}.$ If $y=f\brak{x}$ passes through the point \brak{\alpha,6}, then $\alpha$ is equal to :
  \hfill{[July 2022]}
    \item A common tangent $T$ to the curves $C_1 : \frac{x^2}{4}+\frac{y^2}{9}=1 \text{and} C_2:\frac{x^2}{42}-\frac{y^2}{143}=1 $ does not pass through the fourth quadrant. If $T$ touches $C_1$ at \brak{x_1,y_1} and $C_2$ at \brak{x_2,y_2}, then $\abs{2x_1+x_2}$ is equal to :
\hfill{[July 2022]}
    \item Let, $\overrightarrow{a}$,$\overrightarrow{b}$, $\overrightarrow{c}$ be three non-coplanar vectors such that  $\overrightarrow{a}$$\times$ $\overrightarrow{b}$=4$\overrightarrow{c}$,  $\overrightarrow{b}$$\times$ $\overrightarrow{c}$=9$\overrightarrow{a}$, and  $\overrightarrow{c}$$\times$ $\overrightarrow{a}$=$\alpha$$\overrightarrow{b}$, $\alpha >0$
If $\abs{\overrightarrow{a}}+\abs{\overrightarrow{b}}+\abs{\overrightarrow{c}}=\frac{1}{36}$, then the $\alpha$ is equal to :
\hfill{[July 2022]}
