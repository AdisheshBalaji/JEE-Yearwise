\iffalse
\title{2022}
\author{AI24BTECH11006}
\section{mcq-single}
\fi
\item Let $R_1$ and $R_2$ be two relations defined on $\mathbb{R} $, defined as follows:$a R_1 b \iff ab \geq 0$ and $a R_2 b \iff a \geq b.$ Then,
	\hfill{\sbrak{July-2022}}
	\begin{enumerate}
    \item  $R_1$ is an equivalence relation but not $R_2$
    \item  $R_2$ is an equivalence relation but not $R_1$
    \item  Both $R_1$ and $R_2$ are equivalence relation 
    \item  Neither $R_1$ and $R_2$ are equivalence relation
\end{enumerate}
\item Let $f,g : \mathbb{N} = \{1\} \rightarrow \mathbb{N}$ be functions defined by f$\brak{a}=a$, where a is the maximum of the powers of those primes p such that $p^a$ divides a, and $g\brak{a}=a+1$, for all $a \in N-\{1\}$. Then the function f+g is 
\hfill{\sbrak{July-2022}}
	\begin{enumerate}
    \item One-one but not onto
    \item Onto but not one-one
    \item Both one-one and onto
    \item Neither one-one nor onto
\end{enumerate}
\item Let the minimum value $v_0$ of $v= \abs{z}^2 + \abs{z - 3}^2 + \abs{z - 6i}^2, z \in \mathbb {C}$ is attained at z =$z_0$. Then $\abs{2z_0^2 - \bar{z}_0^3 + 3 }^2 + v_0^2$ is equal to
\hfill{\sbrak{July-2022}}
	\begin{enumerate}
    \item $1000$
    \item $1024$
    \item $1105$
    \item $1196$
\end{enumerate}
\item  Let $A = \begin{pmatrix}1 & 2 \\-2 & -5 \\\end{pmatrix}$. Let$\alpha, \beta \in \mathbb {R}$ be such that $
\alpha A^2 + \beta A = 2I$.Then $ \alpha + \beta $ is equal to
\hfill{\sbrak{July-2022}}
		\begin{enumerate}
    \item $-10$
    \item $-6$
    \item $6$
    \item $10$
\end{enumerate}
\item The remainder when$\brak{2021}^{2022}+\brak{2022}^{2021}$ is divided by $7$ is
\hfill{\sbrak{July-2022}}
	\begin{enumerate}
    \item $ 0$
    \item $1$
    \item $2$
    \item $6$
\end{enumerate}
\item Suppose $a_1, a_2, \dots, a_n, \dots$ be an arithmetic progression of natural numbers. If the ration of the sum of first five terms to the sum of first nine terms of the  progression is $5:17$ and $110$ \textless$ a_{15}$ \textless $120$, then the sum of the first ten terms of the progression is equal to
\hfill{\sbrak{July-2022}}
	\begin{enumerate}
    \item $290$
    \item $380$
    \item $460$
    \item $510$
\end{enumerate}
\item Let $\mathbb {R} \rightarrow \mathbb {R}$ be function defined as
$f\brak{x}=a\sin \brak{\frac{\pi \sbrak{x}}{2}} + \sbrak{2-x}, a\in \mathbb{R}$ where $\sbrak{t}$ is the greatest integer less than or equal to t. If $ \lim_{ x \to 1}f\brak{x}$ exists, then the value of$ \int_{0}^{4}f\brak{x} dx $ is equal to
\hfill{\sbrak{July-2022}}
		\begin{enumerate}
    \item $-1$
    \item $-2$
    \item $1$
    \item $2$
\end{enumerate}
\item  Let $I = \int_{\pi / 4}^{\pi /3}\brak{ \frac{8\sin x - \sin 2x}{x} }dx$ Then
\hfill{\sbrak{July-2022}}
	\begin{enumerate}
    \item $ \frac{\pi}{2}$\textless I \textless $\frac{3\pi}{4}$
    \item $ \frac{\pi}{5}$ \textless I \textless $\frac{5\pi}{12}$
    \item $ \frac{5\pi}{12}$ \textless I \textless $\frac{\sqrt{2}\pi}{3}$
    \item $ \frac{3\pi}{4}$ \textless I \textless $\pi$
\end{enumerate}
\item The area of the smaller region enclosed by the curves $y^2=8x+4$ and $x^2 +y^2 + 4\sqrt{3}x-4 =0$ is equal to
\hfill{\sbrak{July-2022}}
	\begin{enumerate}
    \item $ \frac{1}{3}\brak{2-12\sqrt{3} + 8\pi}$
    \item $ \frac{1}{3}\brak{2-12\sqrt{3} + 6\pi}$
    \item $ \frac{1}{3}\brak{4-12\sqrt{3} + 8\pi}$
    \item $ \frac{1}{3}\brak{4-12\sqrt{3} + 6\pi}$
\end{enumerate}
\item  Let $y=y_1\brak{x}$ and $y=y_2\brak{x}$ be two distinct solution of the differential equation $\frac{dy}{dx}=x+y,$ with $y_1\brak{0}=0$ and $y_2\brak{0}=1$  respectively. Then, the number of points of intersection of $y=y_1\brak{x}$ and $y=y_2\brak{x}$ is
\hfill{\sbrak{July-2022}}
	\begin{enumerate}
    \item $0$
    \item $1$
    \item $2$
    \item $3$
\end{enumerate}
\item Let $P\brak{a,b}$  be a point on the parabola $y^2=8x$ such that the tangent at P passes through the centre of the circle $x^2+y^2-10x-14y+65=0$. Let A be the product of all possible values of a and B be the product of all possible values of b. Then the value of A + B is equal to
\hfill{\sbrak{July-2022}}
	\begin{enumerate}
    \item $0$
    \item $25$
    \item $40$
    \item $65$
\end{enumerate}
\item Let $\vec{a} = \alpha \hat{i} + \hat{j} + \beta \hat{k} $and $ \vec{b} = 3 \hat{i} + 5\hat{j} + 4 \hat{k} $ be two vectors, such that $\vec{a} \times \vec{b} = -\hat{i} + 9\hat{j} + 12 \hat{k}. $Then the projection of $ \vec{b}-2\vec{a} $ on $ \vec{b}+ \vec{a}$ is equal to
\hfill{\sbrak{July-2022}}
	\begin{enumerate}
    \item $2$
    \item $\frac{39}{5}$
    \item $9$
    \item $\frac{46}{5}$
\end{enumerate}
\item Let $\vec{a} = 2\hat{i}-\hat{j}+5\hat{k}$ and $ \vec{b} = \alpha \hat{i} + \beta \hat{j}+2\hat{k}.$ If $\brak{\brak{\vec{a}\times \vec{b}} \times \hat{i}}\cdot \hat{k}=\frac{23}{2},$ then$ \abs{\vec{b}\times 2\hat{j}}$ is equal to
\hfill{\sbrak{July-2022}}
	\begin{enumerate}
    \item $4$
    \item $5$
    \item $\sqrt{21}$
    \item $\sqrt{17}$
\end{enumerate}
\item Let S be the sample space of all five digit numbers. It p is the probability that a randomly selected number from S, is multiple of $7$ but not divisible by $5$, then $9p$ is equal to
\hfill{\sbrak{July-2022}}
	\begin{enumerate}
    \item $1.0146$
    \item $1.2085$
    \item $1.0285$
    \item $1.1521$
\end{enumerate}
\item  Let a vertical tower AB of height 2h stands on a horizontal ground. Let from a point P on the ground a man can see upto height h of the tower with an angle of elevation $2\alpha$. When from P, he moves a distance d in the direction of $ \overrightarrow{AP}$ he can see the top B of the tower with an angle of elevation $\alpha$.  if d = $\sqrt{7}$ h, then $\tan \alpha$ is equal to
\hfill{\sbrak{July-2022}}
	\begin{enumerate}
    \item $ \sqrt{5}-2$
    \item $ \sqrt{3}-1$
    \item $ \sqrt{7}-2$
    \item $ \sqrt{7}-\sqrt{3}$
\end{enumerate}
