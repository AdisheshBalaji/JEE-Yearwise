\iffalse
\title{2022}
\author{AI24BTECH11009}
\section{integer}
\fi
\item Let $S$ be the set of all passwords which are six to eight chaacters long, where each character is either an alphabet from $\{A, B, C, D, E\}$ or a number from $\{1, 2, 3, 4, 5\}$ with the repetition of characters allowed. If the number of passwords in $S$ whose at least one character is a number from $\{1, 2, 3, 4, 5\}$ is $\alpha \times 5^6$, then $\alpha$ is equal to \_\_\_\_\_\_ \hfill[July 2022] \\
\item Let $P\brak{-2, -1, 1}$ and $Q\brak{\frac{56}{17}, \frac{43}{17}, \frac{111}{17}}$ be the vertices of the rhombus $PRQS$. If the direction ratios of the diagonal $RS$ are $\alpha$, -1, $\beta$, where both $\alpha$ and $\beta$ are integers of minimum absolute values, then $\alpha^2 + \beta^2$ is equal to \_\_\_\_\_\_ \hfill[July 2022] \\
\item Let $f:\sbrak{0, 1}\rightarrow\textbf{R}$ be a twice differentiable function in \brak{0, 1} such that $f\brak{0} = 3$ and $f\brak{1} = 5$. If the line $y = 2x + 3$ intersects the graph of $f$ at only two distinct points in \brak{0, 1} then the least number of points $x\in\brak{0, 1}$ at which $f''\brak{x} = 0$, is \_\_\_\_\_\_ \hfill[July 2022] \\
\item If 
\begin{align*}
\int_{0}^{\sqrt{3}}\frac{15x^3}{\sqrt{1+x^2+\sqrt{\brak{1+x^2}^3}}}dx= \alpha\sqrt{2} + \beta\sqrt{3}
\end{align*}
where $\alpha$, $\beta$ are integers, then $\alpha+\beta$ is equal to \hfill[July 2022] \\
\item Let $A=\sbrak{\begin{matrix}
    1 & -1 \\ 2 & \alpha
\end{matrix}}$ and $B=\sbrak{\begin{matrix}
    \beta & 1 \\ 1 & 0
\end{matrix}}$, $\alpha, \beta \in R$. Let $\alpha_1$ be the value of $\alpha$ which satisfies 
\begin{align*}
    \brak{A+B}^2 = A^2 + \sbrak{\begin{matrix}
        2 & 2 \\ 2 & 2
    \end{matrix}}
\end{align*}
and $\alpha_2$ be the value of $\alpha$ which satisfies
\begin{align*}
    \brak{A+B}^2 = B^2.
\end{align*}
Then $\abs{\alpha_1 - \alpha_2}$ is equal to \_\_\_\_\_\_ \hfill[July 2022] \\
\item For $p, q \in R$, consider the real valued function $f\brak{x} = \brak{x - p}^2 - q, x \in R$ and $q > 0$. Let $a_1$, $a_2$, $a_3$ and $a_4$ be in an arithmetic progression with mean $p$ and positive common difference. If $\abs{f\brak{a_i}}$ = 500 for all $i = 1, 2, 3, 4$, then the absolute difference between the roots of $f\brak{x} = 0$ is \hfill[July 2022] \\
\item For the hyperbola $H: x^2 - y^2 = 1$ and the ellipse $E: \frac{x^2}{a^2} + \frac{y^2}{b^2} = 1, a > b > 0$, let the
\begin{enumerate}
    \item eccentricity of $E$ be reciprocal of the eccentricity of $H$, and
    \item the line $y = \sqrt{\frac{5}{2}} x + K$ be a common tangent of $E$ and $H$.
\end{enumerate}
Then 4\brak{a^2 + b^2} is equal to \_\_\_\_\_\_ \hfill[July 2022] \\
\item Let $x_1, x_2, x_3, \cdots, x_{20}$ be in geometric progression with $x_1 = 3$ and the common ratio $\frac{1}{2}$. A new data is constructed replacing each $x_i$ by $\brak{x_i - i}^2$. If $\overline{x}$ is the mean of new data, then the greatest integer less than or equal to $\overline{x}$ is  \_\_\_\_\_\_ \hfill[July 2022] \\
\item 
\begin{align*}
    \lim\limits_{x \rightarrow 0}\brak{\frac{\brak{x+2\cos\brak{x}}^3 + 2\brak{x+2\cos\brak{x}}^2 + 3\sin\brak{x+2\cos\brak{x}}}{\brak{x+2}^3 + 2\brak{x+2}^2 + 3\sin\brak{x+2}}}^\frac{100}{x}
\end{align*}
is equal to \_\_\_\_\_\_ \hfill[July 2022] \\
\item The sum of all real value of $x$ for which
\begin{align*}
    \frac{3x^2-9x+17}{x^2+3x+10} = \frac{5x^2-7x+19}{3x^2+5x+12}
\end{align*}
is equal to \_\_\_\_\_\_ \hfill[July 2022] \\
