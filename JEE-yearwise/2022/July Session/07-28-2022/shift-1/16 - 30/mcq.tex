\iffalse
\title{2022}
\author{AI24BTECH11009}
\section{mcq-single}
\fi
\item The foot of the perpendicular from a point on the circle $x^2 + y^2 = 1$, $z = 0$ to the plane $2x + 3y + z = 6$ lies on which one of the following curves? \hfill[July 2022]
\begin{enumerate}
    \item $\brak{6x + 5y - 12}^2 + 4\brak{3x + 7y - 8}^2 = 1$, $z = 6 - 2x - 3y$
    \item $\brak{5x + 6y - 12}^2 + 4\brak{3x + 5y - 9}^2 = 1$, $z = 6 - 2x - 3y$
    \item $\brak{6x + 5y - 14}^2 + 9\brak{3x + 5y - 7}^2 = 1$, $z = 6 - 2x - 3y$
    \item $\brak{5x + 6y - 14}^2 + 9\brak{3x + 7y - 8}^2 = 1$, $z = 6 - 2x - 3y$\\
\end{enumerate}
\item If the minimum value of 
\begin{align*}
    f\brak{x} = \frac{5x^2}{2} + \frac{\alpha}{x^5}, x>0
\end{align*}
is 14, then the value of $\alpha$ is equal to \hfill[July 2022]
  \begin{enumerate}
      \item 32
      \item 64
      \item 128
      \item 256\\
  \end{enumerate}
\item Let $\alpha, \beta, \gamma$ be three positive real numbers. Let $f\brak{x} = \alpha x^5 + \beta x^3 + \gamma x, x \in \mathbb{R}$ and $g:\mathbb{R}\rightarrow\mathbb{R}$ be such that $g\brak{f\brak{x}} = x$ for all $x\in\mathbb{R}$. If $a_1, a_2, a_3, \cdots, a_n$ be in arithmetic progression with mean zero, then the value of 
\begin{align*}
    f\brak{g\brak{\frac{1}{n}\sum_{i=1}^{n}f\brak{a_i}}}
\end{align*}
is equal to \hfill[July 2022]
     \begin{enumerate}
         \item 0
         \item 3
         \item 9
         \item 27\\
     \end{enumerate}
\item Consider the sequence $a_1, a_2, a_3, \cdots$ such that $a_1$ = 1, $a_2$ = 2 and $a_{n+2} = \frac{2}{a_{n+1} + a_n}$ for $n = 1, 2, 3, \cdots$. If 
\begin{align*}
    \brak{\frac{a_1 + \frac{1}{a_2}}{a_3}}\brak{\frac{a_2 + \frac{1}{a_3}}{a_4}}\brak{\frac{a_3 + \frac{1}{a_4}}{a_5}}\cdots\brak{\frac{a_{30} + \frac{1}{a_{31}}}{a_{32}}} = 2^{\alpha}\brak{^{61}C_{31}},
\end{align*}
then $\alpha$ is equal to \hfill[July 2022]
 \begin{enumerate}
     \item -30
     \item -31
     \item -60
     \item -61\\
 \end{enumerate}
\item The minimum value of the twice differentiable function
\begin{align*}
    f\brak{x} = \int_{0}^{x}e^{x-t}f'\brak{t} - \brak{x^2 - x + 1}e^{x}, x\in\mathbb{R}
\end{align*}
is \hfill[July 2022]
\begin{enumerate}
    \item $-\frac{2}{\sqrt{e}}$
    \item $-2\sqrt{e}$
    \item $-\sqrt{e}$
    \item $\frac{2}{\sqrt{e}}$\\
\end{enumerate}
