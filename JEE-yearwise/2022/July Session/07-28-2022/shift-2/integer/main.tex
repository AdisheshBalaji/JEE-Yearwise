\iffalse
\title{2022}
\author{EE24BTECH11012}
\section{integer}
\fi
%\begin{enumerate}
	\item Let the coefficients of the middle terms in the expansion of $\brak{\frac{1}{\sqrt{6}} + \beta x }^4$, $\brak{1 - 3 \beta x}^2$ and $\brak{1 - \frac{\beta}{2} x}^6$, $\brak{\beta \geq 0}$, respectively form the first three terms of an A.P. If d is the common difference of this A.P., then the value of $50 - \frac{2d}{\beta^{2}}$ is equal to :\hfill{[July 2022]}
	\item A class contains b boys and g girls. If the number of ways of selecting 3 boys and 2 girls from the class is 168, then b + 3g is equal to :\hfill{[July 2022]}
	\item Let the tangents at the points P and Q on the ellipse $\frac{x^2}{2} + \frac{y^2}{4} = 1$ meet at the point $\vec{R}\myvec{\sqrt{2}, 2\sqrt{2}-2}$. If S is the focus of the ellipse on its negative major axis, then $\brak{SP}^2 + \brak{SQ}^2$ is equal to :\hfill{[July 2022]}
	\item  If $ 1 + \brak{2 + \comb{49}{1} + \comb{49}{2} + \dots + \comb{49}{49}}\brak{\comb{50}{2} + \comb{50}{4} + \dots + \comb{50}{50}}$ is equal to $2^{n}m$, where m is odd, then n + m is equal to :\hfill{[July 2022]}
	\item Two tangent lines l1 and l2 are drawn from the point $\myvec{2,0}$ to the parabola $2y^2 = x$. If the lines l1 and l2 are also tangent to the circle $\brak{x-5}^2 + y^2 = r$, then 17$r$ is equal to :\hfill{[July 2022]}
	\item If $\frac{6}{3^{12}} + \frac{10}{3^{11}} + \frac{20}{3^{10}} + \frac{40}{3^{9}} + \dots + \frac{10240}{3} = 2^{n}{m}$, where $m$ is odd, then $m\cdot n$ is equal to :\hfill{[July 2022]}
	\item Let $S = \left[{-\pi, \frac{\pi}{2}}\right) - \cbrak{\frac{-\pi}{2}, \frac{-\pi}{4}, \frac{-3\pi}{4}, \frac{\pi}{4}}$. Then the number of elements in the set $$ A = \cbrak{\theta \in S : \tan{\theta}\brak{1 + \sqrt{5}\tan{2\theta}} = \sqrt{5} - \tan{2\theta}}$$ is :\hfill{[July 2022]}
	\item Let $ z = a + ib, b \neq 0$ be complex numbers satisfying $z^2 = \overline{z} 2^{1 - \abs{z}}$ Then the least value of $n \in \vec{N}$ suh that $z^{n} = \brak{z+1}^{n}$ is equal to :\hfill{[July 2022]}
	\item A bag contains  white and 6 black balls. Three balls are drawn at random from the bag. Let X be the number of white balls, among the drawn balls. If $\sigma^2$ is the variance of X, then 100 $\sigma^2$ is equal to\hfill{[July 2022]}
	\item The value of the integral $\int_{0}^{\frac{\pi}{2}} 60 \frac{\sin{6x}}{\sin{x}} dx $ is equal to :\hfill{[July 2022]}

%\end{enumerate}
%\end{document}
