\iffalse
\title{2021}
\author{EE24BTECH11012}
\section{integer}
\fi
%\begin{enumerate}
	\item The number of real roots of the equation $ \brak{x+1}^2 + \abs{x-5} = \frac{27}{4} $ is : \hfill{[Feb 2021]}
	\item The students $S_1, S_2, \dots, S_{10}$ are to be divided into 3 groups A, B and C such that each group has at least one student and the group C has at most 3 students. Then the total number of possibilities of forming such groups is :\hfill{[Feb 2021]}
	\item If $ \emph{a} + \alpha = 1 $, $ \emph{b} + \beta = 2 $ and $af\brak{x} + \alpha\brak{1}{x} = bx + \frac{\beta}{2} $, $ x\neq0 $ then the value of the expression $\frac{\sbrak{f\brak{x} + f\brak{\frac{1}{x}}}}{\brak{x + \frac{1}{x}}}$ :\hfill{[Feb 2021]}
	\item If the variance of 10 natural numbers 1,1,1,\dots,1,\emph{k} is less than 10, then the maximum possible value of \emph{k} is :\hfill{[Feb 2021]}
	\item Let $\lambda$ be an integer. If the shortest distance between the lines $ x - \lambda = 2y - 1 = = -2z $ and $ x = y + 2\lambda = z - \lambda $ is $\frac{\sqrt{7}}{2\sqrt{2}}$, then the value of $\abs{\lambda}$ is :\hfill{[Feb 2021]}
	\item If $ \emph{i} = \sqrt{-1} $. If $ \frac{\brak{-1+\emph{i}\sqrt{3}}^21}{\brak{1-\emph{i}}^24} + \frac{\brak{1+\emph{i}\sqrt{3}}^21}{\brak{1+\emph{i}}^24} = \emph{k}$, and $ \emph{n} = \sbrak{\abs{\emph{k}}}$ be the greatest integral part of $\abs{\emph{k}}$. Then $ \sum_{j=0}^{\emph{n}+5} \brak{j+5}^2 - \sum_{j=0}^{\emph{n+5}} \brak{j+5} $ is equal to : \hfill{[Feb 2021]}
	\item Let a point $\vec{P}$ be such that its distance from the point $\myvec{5,0}$ is thrice the distance of $\vec{P}$ from the point $\myvec{-5,0}$. If the locus of the point $\vec{P}$ is a circle of radius \emph{r}, then $4\emph{r}^2$ is equal to : \hfill{[Feb 2021]}
	\item The maximum value of k for which the sum $ \sum_{i=0}^{k} \comb{10}{i} \comb{15}{k-i} + \sum_{i=0}^{k+1} \comb{12}{i} \comb{13}{k+1-i} $ exists, is equal to : \hfill{[Feb 2021]}

	\item The sum of first four terms of a geometric progression is $\frac{65}{12}$ and the sum of their respective reciprocals is $\frac{65}{18}$. If the product of first three terms of the G.P. is 1, and the third term is $\alpha$ then $2\alpha$ is : \hfill{[Feb 2021]}

	\item If the area of the triangle formed by the positive x-axis, the normal and the tangent to the circle $\brak{x-2}^2 + \brak{y-3}^2 = 25$ at the point $\myvec{5,7}$ is \emph{A}, then 24\emph{A} is equal to :\hfill{[Feb 2021]}
%\end{enumerate}
 %\end{document}
