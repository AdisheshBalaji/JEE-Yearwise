\iffalse
\title{2021}
\author{AI24BTECH11031}
\section{mcq-single}
\fi

\item The value of $\lim\limits_{h \to 0} 2\cbrak{\frac{\sqrt{3}\sin\brak{\frac{\pi}{6} - h} - \cos\brak{\frac{\pi}{6} + h}}{\sqrt{3}h\brak{\sqrt{3}\cos h - \sin h}}}$ is
\hfill{[Feb 2021]}

\begin{multicols}{4}
\begin{enumerate}
    \item $\frac{3}{4}$
    \item $\frac{2}{\sqrt{3}}$
    \item $\frac{4}{3}$
    \item $\frac{2}{3}$
\end{enumerate}
\end{multicols}

\item A fair coin is tossed a fixed number of times. If the probability of
getting 7 heads is equal to the probability of getting 9 heads, then the
probability of getting 2 heads is:
\hfill{[Feb 2021]}

\begin{multicols}{4}
\begin{enumerate}
    \item $\frac{15}{2^{12}}$
    \item $\frac{15}{2^{13}}$
    \item $\frac{15}{2^{14}}$
    \item $\frac{15}{2^{8}}$
\end{enumerate}
\end{multicols}

\item If $(1, 5, 35)$, $(7, 5, 5)$, $(1, \lambda, 7)$ and $(2\lambda, 1, 2)$ are
coplanar, then the sum of all possible values of $\lambda$ is:
\hfill{[Feb 2021]}

\begin{multicols}{4}
\begin{enumerate}
    \item $-\frac{44}{5}$
    \item $\frac{39}{5}$
    \item $-\frac{39}{5}$
    \item $\frac{44}{5}$
\end{enumerate}
\end{multicols}

\item Let $R = \cbrak{(P,Q) \mid \text{P and Q are at the same distance from the origin}}$
be a relation, then the equivalence class of $(1,-1)$ is the set:
\hfill{[Feb 2021]}

\begin{multicols}{2}
\begin{enumerate}
    \item $S = \cbrak{(x, y) \mid x^2 + y^2 = 1}$
    \item $S = \cbrak{(x, y) \mid x^2 + y^2 = 4}$
    \item $S = \cbrak{(x, y) \mid x^2 + y^2 = \sqrt{2}}$
    \item $S = \cbrak{(x, y) \mid x^2 + y^2 = 2}$
\end{enumerate}
\end{multicols}

\item In the circle given below, let $OA$ = 1 unit,
$OB$ = 13 unit and $PQ$ perpendicular to $OB$. Then, the area
of the triangle $PQB$ (in square units) is:
\hfill{[Feb 2021]}

\begin{center}
\begin{tikzpicture}
    \draw [->] (0, -2) -- (0, 2) node[left] {Y};
    \draw [->] (-2, 0) -- (4, 0) node[right] {X};
    \draw (1, 0) circle (1);
    \draw (2, 0) node[below right] {B} --
            (0.5, -0.86) node[below] {Q} --
            (0.5, 0.86) node[above] {P} -- (2, 0);
    \draw (0, 0) node[below left] {O} -- (0.5, 0) node[below right] {A};
\end{tikzpicture}
\end{center}

\begin{multicols}{4}
\begin{enumerate}
    \item $26 \sqrt{3}$
    \item $24 \sqrt{2}$
    \item $24 \sqrt{3}$
    \item $26 \sqrt{2}$
\end{enumerate}
\end{multicols}
