\iffalse
\title{2021}
\author{EE24BTECH11019}
\section{mcq-single}
\fi

%\begin{enumerate}
    %1
    \item The sum of solutions of the equation $\frac{\cos{x}}{1+\sin{x}}=\abs{\tan{2x}}, x \in \brak{-\frac{\pi}{2},\frac{\pi}{2}}-\cbrak{\frac{\pi}{4},-\frac{\pi}{4}}$
    
    \hfill[Aug 2021]
        \begin{enumerate}
            \item $-\frac{11\pi}{30}$
            \item $\frac{\pi}{10}$
            \item $-\frac{7\pi}{30}$
            \item $-\frac{\pi}{15}$
        \end{enumerate}
    
    %2
    \item The mean and standard deviation of $20$ observations were calculated as $10$ and $2.5$ respectively. It was found that by mistake one data value was taken as $25$ instead of $35$. If $a$ and $b$ are the mean and standard deviation respectively for correct data, then $(a,b)$ is :
    
    \hfill[Aug 2021]
        \begin{enumerate}
            \item $(11,26)$
            \item $(10.5,25)$
            \item $(11,25)$
            \item $(10.5,26)$
        \end{enumerate}
    
    %3
    \item On the ellipse $\frac{x^2}{8}+\frac{y^2}{4}=1$ let P be a point in the second quadrant such that the tangent at $\vec{P}$ to the ellipse is perpendicular to the line $x+2y=0$. Let $\vec{S}$ and $\vec{S'}$ be the foci of the ellipse and $e$ be its eccentricity. If $A$ is the area of the triangle $\vec{SPS'}$ then, the value of $\brak{5-e^2}\cdot A$ is :
    
    \hfill[Aug 2021]
        \begin{enumerate}
            \item $6$
            \item $12$
            \item $14$
            \item $24$
        \end{enumerate}

    %4
    \item Let $y=y(x)$ be a solution curve of the differential equation $(y+1)\tan^2{x}\,dx + \tan{x}\,dy + y\,dx=0, x\in\brak{0,\frac{\pi}{2}}$. If $\lim\limits_{x\to0+}xy(x)=1$, then the value of $y\brak{\frac{\pi}{4}}$ is :
    
    \hfill[Aug 2021]
        \begin{enumerate}
            \item $-\frac{\pi}{4}$
            \item $\frac{\pi}{4}-1$
            \item $\frac{\pi}{4}+1$
            \item $\frac{\pi}{4}$
        \end{enumerate}

    %5
    \item Let $A$ and $B$ be independent events such that $P(A) = p, P(B) = 2p$. The largest value of $p$, for which $P$(exactly one of $A, B$ occurs) = $\frac{5}{9}$, is :
    
    \hfill[Aug 2021]
        \begin{enumerate}
            \item $\frac{1}{3}$
            \item $\frac{2}{9}$
            \item $\frac{4}{9}$
            \item $\frac{5}{12}$
        \end{enumerate}

    %6
    \item Let $\theta\in\brak{0,\frac{\pi}{2}}$. If the system of linear equations
    
    \hfill[Aug 2021]
            $$\brak{1+\cos^2{\theta}}x+\sin^2{\theta}y+4\sin{3\theta}z=0$$
            $$\cos^2{\theta}x+\brak{1+\sin^2{\theta}}y+4\sin{3\theta}z=0$$
            $$\cos^2{\theta}x+\sin^2{\theta}y+\brak{1+4\sin{3\theta}}z=0$$
        has a non-trivial solution, then the value of $\theta$ is :
        \begin{enumerate}
            \item $\frac{4\pi}{9}$
            \item $\frac{7\pi}{18}$
            \item $\frac{\pi}{18}$
            \item $\frac{5\pi}{18}$
        \end{enumerate}

    %7
    \item Let $f(x)=\cos\brak{2\tan^{-1}\sin\brak{\cot^{-1}\sqrt{\frac{1-x}{x}}}}, 0<x<1$. Then :
    
    \hfill[Aug 2021]
        \begin{enumerate}
            \item $(1-x)^2f^\prime(x)-2(f(x))^2=0$
            \item $(1+x)^2f^\prime(x)+2(f(x))^2=0$
            \item $(1-x)^2f^\prime(x)+2(f(x))^2=0$
            \item $(1+x)^2f^\prime(x)-2(f(x))^2=0$
        \end{enumerate}


    %8
    \item The sum of the series
        $$\frac{1}{x+1}+\frac{2}{x^2+1}+\frac{2^2}{x^4+1}+\dots+\frac{2^{100}}{x^{2^{100}}+1}$$ when $x=2$ is :
    
    \hfill[Aug 2021]
        \begin{enumerate}
            \item $1+\frac{2^{101}}{4^{101}-1}$
            \item $1+\frac{2^{100}}{4^{101}-1}$
            \item $1-\frac{2^{100}}{4^{100}-1}$
            \item $1-\frac{2^{101}}{4^{101}-1}$
        \end{enumerate}


    %9
    \item If $\comb{20}{r}$ is the coefficient of $x^r$ in the expansion of $(1+x)^{20}$, then the value of $\sum\limits_{r=0}^{20}r^2\,\comb{20}{r}$ is equal to :
    
    \hfill[Aug 2021]
        \begin{enumerate}
            \item $420\times2^{19}$
            \item $380\times2^{19}$
            \item $380\times2^{18}$
            \item $420\times2^{18}$
        \end{enumerate}

    %10
    \item Out of all the patients in a hospital $89\%$ are found to be suffering from heart ailment and $98\%$ are suffering from lungs infection. If $K\%$ of them are suffering from both ailments, then $K$ cannot belong to the set :
    
    \hfill[Aug 2021]
        \begin{enumerate}
            \item \cbrak{80, 83, 86, 89}
            \item \cbrak{84, 86, 88, 90}
            \item \cbrak{79, 81, 83, 85}
            \item \cbrak{84, 87, 90, 93} 
        \end{enumerate}

    %11
    \item The equation $\arg\brak{\frac{z-1}{z+1}}=\frac{\pi}{4}$ represents a circle with
    
    \hfill[Aug 2021]
        \begin{enumerate}
            \item centre at $(0,-1)$ and radius $\sqrt{2}$
            \item centre at $(0,1)$ and radius $\sqrt{2}$
            \item centre at $(0,0)$ and radius $\sqrt{2}$
            \item centre at $(0,1)$ and radius $2$
        \end{enumerate}

    %12
    \item Let $\vec{a}=\hat{i}+\hat{j}+\hat{k}$ and $\vec{b}=\hat{j}-\hat{k}$. If $\vec{c}$ is a vector such that $\vec{a}\times\vec{c}=\vec{b}$ and $\vec{a}\cdot\vec{c}=3$, then $\vec{a}\cdot(\vec{b}\times\vec{c})$ is equal to :
    
    \hfill[Aug 2021]
        \begin{enumerate}
            \item $-2$
            \item $-6$
            \item $6$
            \item $2$
        \end{enumerate}

    %13
    \item If a line along a chord of the circle $4x^2+4y^2+120x+675=0$, passes through the point $(-30, 0)$ and is tangent to the parabola $y^2 = 30x$ then the length of this chord is :
    
    \hfill[Aug 2021]
        \begin{enumerate}
            \item $5$
            \item $7$
            \item $5\sqrt{3}$
            \item $3\sqrt{5}$
        \end{enumerate}

    %14
    \item The value of $\int\limits_{\frac{-1}{\sqrt{2}}}^{\frac{1}{\sqrt{2}}}\brak{\brak{\frac{x+1}{x-1}}^2+\brak{\frac{x-1}{x+1}}^2-2}^{\frac{1}{2}}\,dx$ is :
    
    \hfill[Aug 2021]
        \begin{enumerate}
            \item $\log_{e}4$
            \item $\log_{e}16$
            \item $2\log_{e}16$
            \item $4\log_{e}\brak{3+2\sqrt{2}}$
        \end{enumerate}

    %15
    \item A plane $P$ contains the line $x+2y+3z+1=0=x-y-z-6$, and is perpendicular to the plane $-2x+y+z+8=0$. Then which of the following points lies on $P$ ?
    
    \hfill[Aug 2021]
        \begin{enumerate}
            \item $(-1, 1, 2)$
            \item $(0, 1, 1)$
            \item $(1, 0, 1)$
            \item $(2, -1, 1)$
        \end{enumerate}

%\end{enumerate}
