\iffalse
\title{08-26-2021-shift-1-16-20}
\author{AI24BTECH11011}
\section{mcq-single}
\fi
        \item If $\vec{A}=\myvec{\frac{1}{\sqrt{5}}&\frac{2}{\sqrt{5}}\\\frac{-2}{\sqrt{5}}&\frac{1}{\sqrt{5}}},\vec{B}=\myvec{1&0\\i&1},i=\sqrt{-1},$ and $\vec{Q}=\vec{A^{T}}\vec{B}\vec{A}$, then the inverse of the matrix $\vec{A}\vec{Q^{2021}}\vec{A^{T}}$ is equal to:
		\hfill{(august 2021)}
		\begin{enumerate}
            \item $\myvec{\frac{1}{\sqrt{5}}&-2021\\2021&\frac{1}{\sqrt{5}}}$
            \item $\myvec{1&0\\-2021i&1}$
            \item $\myvec{1&0\\2021i&1}$
            \item $\myvec{1&-2021i\\0&1}$
        \end{enumerate}
        \item If the sum of an infinite GP $a, ar, ar^{2}, ar^{3},\cdots$ is 15 and the sum of the squares of its each term is 150, then the sum of $ar^{2}, ar^{4}, ar^{6}, \cdots$ is :
        \hfill{(august 2021)}  
	\begin{enumerate}
            \item $\frac{5}{2}$
            \item $\frac{1}{2}$
            \item $\frac{25}{2}$
            \item $\frac{9}{2}$
        \end{enumerate}
        \item The value of $\lim_{n \to \infty}\frac{1}{n}\sum_{t=0}^{2n-1}\frac{n^{2}}{n^{2}+4r^{2}}$ is:
         \hfill{(august 2021)}  
		\begin{enumerate}
            \item $\frac{1}{2}\tan^{-1}\brak{2}$
            \item $\frac{1}{2}\tan^{-1}\brak{4}$
            \item $\tan^{-1}\brak{4}$
            \item $\frac{1}{4}\tan^{-1}\brak{4}$
        \end{enumerate}
        \item Let $ABC$ be a triangle with $\vec{A}\brak{-3,1}$ and $\langle ACB=\theta,0<\theta<\frac{\pi}{2}$. If the equation of the median through $\vec{B}$ is $2x+y-3=0$ and the equation of the angle bisector of $\vec{C}$ is $7x-4y-1=0$, then $\tan \theta$ is equal to:
         \hfill{(august 2021)}  
		\begin{enumerate}
            \item $\frac{1}{2}$
            \item $\frac{3}{4}$
            \item $\frac{4}{3}$
            \item 2
        \end{enumerate}
        \item If the truth value of the Boolean expression $\brak{\brak{p \lor q} \land \brak{q\rightarrow r} \land \brak{\neg r}}\rightarrow \brak{p \land q}$ is false, then truth values of the statements $p,q,r$ respectively can be:
         \hfill{(august 2021)}  
	\begin{enumerate}
            \item T F T
            \item F F T
            \item T F F
            \item F T F
        \end{enumerate}

