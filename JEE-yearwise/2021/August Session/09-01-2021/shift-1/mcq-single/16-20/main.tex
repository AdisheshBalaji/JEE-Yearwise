\iffalse
    \title{2021}
    \author{EE24BTECH11001}
    \section{mcq-single}
\fi
\item 
        Let $P_1, P_2 \dots P_{15}$  be 15 points on a circle. The number of distinct triangles formed by points $P_i, P_j, P_k$ such that 
        $i + j +k \ne 0$ is :
        \hfill{\brak{\textnormal{2021-Sep}}}
        \begin{multicols}{4}
            \begin{enumerate}
                \item 12
                    \columnbreak
                \item 419
                    \columnbreak
                \item 443
                    \columnbreak
                \item 455
            \end{enumerate}
        \end{multicols}

    \item The range of the function 
        \begin{align}
            f\brak{x} = \log_{\sqrt{3}} \brak{3 + \cos \brak{\frac{3\pi}{4} + x} + \cos \brak{\frac{\pi}{4} + x}+ \cos \brak{\frac{\pi}{4} - x} +\cos \brak{\frac{3\pi}{4} - x}}
        \end{align} is :
        \hfill{\brak{\textnormal{2021-Sep}}}
        \begin{multicols}{4}
            \begin{enumerate}
                \item $\brak{0,  \sqrt{5}}$ \columnbreak
                \item $\sbrak{-2, 2}$ \columnbreak
                \item $\sbrak{\frac{1}{\sqrt{5}}, \sqrt{5}}$ \columnbreak
                \item $\sbrak{0, 2}$
            \end{enumerate}
        \end{multicols}


    \item Let $a_1, a_2 \dots a_{21}$ be an AP such that 
        \begin{align}
            \sum_{n = 1} ^{20} \frac{1}{a_{n}a_{n+1}} = \frac{4}{9}
        \end{align}. If the sum of the AP is 189, then $a_{6}a_{16}$ is : 
        \hfill{\brak{\textnormal{2021-Sep}}}
        \begin{enumerate}
                \begin{multicols}{2}
                \item 57 \columnbreak
                \item 72
                \end{multicols}
                \begin{multicols}{2}
                \item 48 \columnbreak
                \item 36
                \end{multicols}
        \end{enumerate}

    \item  The function $f\brak{x}$, that satisfies the condition  
        \begin{align}
            f\brak{x} = x + \int_{0} ^ {\frac{\pi}{2}} \sin x \cos y f\brak{y} \, dy
        \end{align} 
        is : 
        \hfill{\brak{\textnormal{2021-Sep}}}
        \begin{enumerate}
                \begin{multicols}{4}
                \item $x + \frac{2}{3}\brak{\pi - 2}\sin x$ \columnbreak
                \item $x + \brak{\pi + 2}\sin x$ \columnbreak
                \item $x + \frac{\pi}{2}\sin x$ \columnbreak
                \item $x + \brak{\pi - 2}\sin x$
                \end{multicols}
        \end{enumerate}

    \item Let $\theta$ be the acute angle between the tangents to the ellipse 
        \begin{align}
            \frac{x^2}{9} + \frac{y^2}{1} = 1
        \end{align} and the circle 
        \begin{align}
            x^2 + y^2 = 3
        \end{align} at their point of intersection in the first quadrant. Then $\tan \theta$ is equal to :

        \hfill{\brak{\textnormal{2021-Sep}}}
        \begin{enumerate}
            \item $\frac{5}{2\sqrt{2}}$ 
            \item $\frac{2}{\sqrt{3}}$ 
            \item $\frac{4}{\sqrt{3}}$  
            \item $2$
        \end{enumerate}
