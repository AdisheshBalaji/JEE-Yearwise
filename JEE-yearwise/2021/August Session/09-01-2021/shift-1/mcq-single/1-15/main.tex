\iffalse
    \title{2021}
    \author{AI24BTECH11030}
    \section{mcq-single}
\fi

    \item Let $f : \mathbb{R} \rightarrow \mathbb{R}$ be a continuous function. Then
    $\lim_{x \to \frac{\pi}{4}} \frac{\frac{\pi}{4} \int_{2}^{\sec^2 x} f(x) dx}{ x^2 - \frac{\pi^2}{16}}$
    is equal to: \hfill [Aug 2021]
    \begin{multicols}{4}
        \begin{enumerate}
            \item $f(2)$
            \item $2f(2)$
            \item $2f(\sqrt{2})$
            \item $4f(2)$
        \end{enumerate}
    \end{multicols}
    
    \item $\cos^{-1}(\cos(-5)) + \sin^{-1}(\sin(6)) - \tan^{-1}(\tan(12))$ is equal to: \hfill [Aug 2021]
    \begin{multicols}{4}
        \begin{enumerate}
            \item $3\pi - 11$
            \item $4\pi - 9$
            \item $4\pi - 11$
            \item $3\pi + 1$
        \end{enumerate}
    \end{multicols}

    \item Consider the system of linear equations:
    \begin{align*}
        -x + y + 2z &= 0 \\
        3x - ay + 5z &= 1 \\
        2x - 2y - az &= 7
    \end{align*}
    Let $S_1$ be the set of all $a \in \mathbb{R}$ for which the system is inconsistent, and $S_2$ the set of all $a \in \mathbb{R}$ for which the system has infinitely many solutions. If $n(S_1)$ and $n(S_2)$ denote the number of elements in $S_1$ and $S_2$ respectively, then: \hfill [Aug 2021]
    \begin{multicols}{2}
        \begin{enumerate}
            \item $n(S_1) = 2$, $n(S_2) = 2$
            \item $n(S_1) = 1$, $n(S_2) = 0$
            \item $n(S_1) = 2$, $n(S_2) = 0$
            \item $n(S_1) = 0$, $n(S_2) = 2$
        \end{enumerate}
    \end{multicols}
    
    \item Let the acute angle bisector of the planes $x - 2y - 2z + 1 = 0$ and $2x - 3y - 6z + 1 = 0$ be the plane $P$. Which of the following points lies on $P$? \hfill [Aug 2021]
    \begin{multicols}{2}
        \begin{enumerate}
            \item $(3, 1, -\frac{1}{2})$
            \item $(-2, 0, -\frac{1}{2})$
            \item $(0, 2, -4)$
            \item $(4, 0, -2)$
        \end{enumerate}
    \end{multicols}

    \item Which of the following is equivalent to the Boolean expression $p \land \neg q$? \hfill [Aug 2021]
    \begin{multicols}{4}
        \begin{enumerate}
            \item $\neg(q \to p)$
            \item $\neg p \to \neg q$
            \item $\neg(p \to \neg q)$
            \item $\neg(p \to q)$
        \end{enumerate}
    \end{multicols}
    
    \item Two squares are chosen at random on a chessboard. The probability that they have a side in common is: \hfill [Aug 2021]
    \begin{multicols}{4}
        \begin{enumerate}
            \item $\frac{2}{7}$
            \item $\frac{1}{18}$
            \item $\frac{1}{7}$
            \item $\frac{1}{9}$
        \end{enumerate}
    \end{multicols}

    \item If $y = y(x)$ is the solution of the differential equation $x^2 dy + \brak{y - \frac{1}{x}} dx = 0$; $x > 0$ and $y(1) = 1$, then $y\brak{\frac{1}{2}}$ is equal to: \hfill [Aug 2021]
    \begin{multicols}{4}
        \begin{enumerate}
            \item $\frac{3}{2} - \frac{1}{\sqrt{e}}$
            \item $3 + \frac{1}{\sqrt{e}}$
            \item $3 + e$
            \item $3 - e$
        \end{enumerate}
    \end{multicols}

    \item If $n$ is the number of solutions of the equation
    $$
    2\cos x\brak{4\sin\brak{\frac{\pi}{4} + x} \sin\brak{\frac{\pi}{4} - x} -1 } = 1, \: x \in [0, \pi],
    $$
    and $S$ is the sum of all these solutions, then the ordered pair $(n, S)$ is: \hfill [Aug 2021]
    \begin{multicols}{4}
        \begin{enumerate}
            \item $(3, \frac{13\pi}{9})$
            \item $(2, \frac{2\pi}{3})$
            \item $(2, \frac{8\pi}{9})$
            \item $(3, \frac{5\pi}{3})$
        \end{enumerate}
    \end{multicols}

    \item The function $f(x) = x^3 - 6x^2 + ax + b$ is such that $f(2) = f(4) = 0$. Consider two statements:
    \begin{itemize}
        \item[(S1)] There exists $x_1, x_2 \in (2, 4)$, $x_1 < x_2$, such that $f'(x_1) = -1$ and $f'(x_2) = 0$.
        \item[(S2)] There exists $x_3, x_4 \in (2, 4)$, $x_3 < x_4$, such that $f$ is decreasing in $(2, x_4)$, increasing in $(x_4, 4)$ and $2f'(x_3) = \sqrt{3}f(x_4)$.
    \end{itemize}
    Then:  \hfill [Aug 2021]
    \begin{multicols}{2}
        \begin{enumerate}
            \item Both (S1) and (S2) are true
            \item (S1) is false and (S2) is true
            \item Both (S1) and (S2) are false
            \item (S1) is true and (S2) is false
        \end{enumerate}
    \end{multicols}

    \item Let 
    $$
    J_{n,m} = \int_0^\frac{1}{2} \frac{x^n}{x^m-1} dx, \forall n > m \quad \text{and} \quad n, m \in \mathbb{N}.
    $$
    Consider a matrix $A = [a_{ij}]_{3 \times 3}$ where 
    $$
    a_{ij} = \begin{cases} 
    J_{6+i,3} - J_{i+3,3}, & \text{if } i \leq j, \\
    0, & \text{if } i > j.
    \end{cases}
    $$
    Then $\abs{\text{adj} A^{-1}}$ is: \hfill [Aug 2021]
    \begin{multicols}{4}
        \begin{enumerate}
            \item $(15)^2 \times 2^{42}$
            \item $(15)^2 \times 2^{34}$
            \item $(105)^2 \times 2^{38}$
            \item $(105)^2 \times 2^{36}$
        \end{enumerate}
    \end{multicols}

    \item The area enclosed by the curves $y = \abs{\cos x - \sin x}$ and $y = \sin x + \cos x$, and the lines $x = 0$ and $x = \frac{\pi}{2}$ is: \hfill [Aug 2021]
    \begin{multicols}{4}
        \begin{enumerate}
            \item $2\sqrt{2} - 2$
            \item $2 + 2\sqrt{2}$
            \item $4 - 2\sqrt{2}$
            \item $2 + 4\sqrt{2}$
        \end{enumerate}
    \end{multicols}

    \item The distance of the line $3y - 2z - 1 = 0 = 3x - z + 4$ from the point $(2, -1, 6)$ is: \hfill [Aug 2021]
    \begin{multicols}{4}
        \begin{enumerate}
            \item $\sqrt{26}$
            \item $2\sqrt{5}$
            \item $2\sqrt{6}$
            \item $4\sqrt{2}$
        \end{enumerate}
    \end{multicols}

    \item Consider the parabola with vertex $\brak{\frac{1}{2}, \frac{3}{4}}$ and the directrix $y = \frac{1}{2}$. Let $P$ be the point where the parabola meets the line $x = -\frac{1}{2}$. If the normal to the parabola at $P$ intersects the parabola again at the point $Q$, then $(PQ)^2$ is equal to: \hfill [Aug 2021]
    \begin{multicols}{4}
        \begin{enumerate}
            \item $\frac{75}{8}$
            \item $\frac{125}{16}$
            \item $\frac{25}{2}$
            \item $\frac{15}{2}$
        \end{enumerate}
    \end{multicols}

    \item The number of pairs $(a, b)$ of real numbers, such that whenever $\alpha$ is a root of the equation $x^2 + ax + b = 0$, $\alpha^2 - 2$ is also a root of the equation, is: \hfill [Aug 2021]
    \begin{multicols}{4}
        \begin{enumerate}
            \item 6
            \item 2
            \item 4
            \item 8
        \end{enumerate}
    \end{multicols}

    \item Let $S_n = 1\cdot(n-1) + 2\cdot(n-2) + \dots + (n-1)\cdot1$, for $n \geq 4$. The sum 
    $$
    \sum_{n=4}^{\infty} \brak{\frac{2S_n}{n!} - \frac{1}{(n-2)!}}
    $$
    is equal to: \hfill [Aug 2021]
    \begin{multicols}{4}
        \begin{enumerate}
            \item $\frac{e - 1}{3}$
            \item $\frac{e - 2}{6}$
            \item $\frac{e}{3}$
            \item $\frac{e}{6}$
        \end{enumerate}
    \end{multicols}

