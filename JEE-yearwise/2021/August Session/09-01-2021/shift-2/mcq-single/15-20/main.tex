\iffalse
  \title{Assignment}
  \author{EE24BTECH11038}
  \section{mcq-single}
\fi  
\item Let $p_1,p_2,p_3,\cdots,p_{15}$ be points on circle. The number of distinct triangles formed by points $p_i,p_j,p_k$ such that i+j+k$\neq$15, is :\hfill{Aug 2021}
\begin{enumerate}
    \item 12
    \item 419
    \item 443
    \item 455
\end{enumerate}
\bigskip
\item The range of the function,
\begin{align*}
    f\brak{x}=\log_{\sqrt{5}} \left(3+\cos{\brak{\frac{3\pi}{4}+x}}+\cos{\brak{\frac{\pi}{4}+x}}+\cos{\brak{\frac{\pi}{4}-x}}-\cos{\brak{\frac{3\pi}{4}-x}}\right)
\end{align*}
is:\hfill{Aug 2021}
\begin{enumerate}
    \item $\brak{0,\sqrt{5}}$\\
    \item $\sbrak{-2,2}$\\
    \item $\sbrak{\frac{1}{\sqrt{5}},\sqrt{5}}$\\
    \item $\sbrak{0,2}$
\end{enumerate}
\bigskip
\item Let $a_1,a_2,a_3,\cdots,a_{21}$ be an A.P such that $\sum_{n=1}^{20} \frac{1}{a_na_{n+1}}=\frac{4}{9}$. If the sum of this A.P is 189, then $a_6a_{16}$ is equal to :\hfill{Aug 2021}
\begin{enumerate}
    \item 57
    \item 72
    \item 48
    \item 36
\end{enumerate}
\bigskip
\item The function f$\brak{x}$, that satisfies the condition f$\brak{x}$=x+$\int_0^{\frac{\pi}{2}} \sin{x}.\cos{y}f\brak{y}\,dy,$ is :\hfill{Aug 2021}
\begin{enumerate}
    \item x+$\frac{2}{3}\brak{\pi-2}\sin{x}$
    \item  x+$\brak{\pi+2}\sin{x}$
     \item x+$\frac{\pi}{2}\sin{x}$
      \item x+$\brak{\pi-2}\sin{x}$
\end{enumerate}
\bigskip
\item Let $\theta$ be the acute angle between the tangents to the ellipse $\frac{x^2}{9}+\frac{y^2}{1}$=1 and the circle $x^2+y^2=3$ at their point of intersection in the first quadrant then $\tan{\theta}$ is equal to:\hfill{Aug 2021}
\begin{enumerate}
    \item $\frac{5}{2\sqrt{3}}$
    \item $\frac{2}{\sqrt{3}}$
    \item $\frac{4}{\sqrt{3}}$
    \item 2
\end{enumerate}


