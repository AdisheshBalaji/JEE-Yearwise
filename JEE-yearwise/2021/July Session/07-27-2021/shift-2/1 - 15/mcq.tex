\iffalse
\title{2021}
\author{AI24BTECH11009}
\section{mcq-single}
\fi
\item The point $P \brak{a, b}$ undergoes the following three transformations successively: \hfill[July 2021]
\begin{enumerate}
    \item reflection about the line $y = x$.
    \item translation through 2 units along the positive direction of x-axis.
    \item rotation through angle $\frac{\pi}{4}$ about the origin in the anti-clockwise direction.
    \end{enumerate}
    If the co-ordinates of the final position of the point $P$ are $\brak{-\frac{1}{\sqrt{2}}, \frac{7}{\sqrt{2}}}$, then the value of $2a + b$ is equal to:
    \begin{enumerate}
        \item 13
        \item 9
        \item 5
        \item 7 \\
    \end{enumerate}
\item A possible value of $'x'$, for which the ninth term in the expansion of 
\begin{align*}
\left\{3^{\log_{3}\sqrt{25^{x-1}+7}}+3^{\brak{-\frac{1}{8}}\log_3\brak{5^{x-1}+1}}\right\}^{10}
\end{align*}
in the increasing powers of $3^{\brak{-\frac{1}{8}}\log_3\brak{5^{x-1}+1}}$ is equal to 180, is: \hfill[July 2021]
\begin{enumerate}
    \item 0
    \item -1
    \item 2
    \item 1 \\
\end{enumerate}
\item For real numbers $\alpha$ and $\beta$ $\neq$ 0, if the point of intersection of the straight lines $\frac{x - \alpha}{1} = \frac{y - 1}{2} = \frac{z - 1}{3}$ and $\frac{x - 4}{\beta} = \frac{y - 6}{3} = \frac{z - 7}{3}$, lies on the plane $x + 2y - z = 8$, then $\alpha - \beta$ is equal to: \hfill[July 2021]
\begin{enumerate}
    \item 5
    \item 9
    \item 3
    \item 7 \\
\end{enumerate}
 \item Let $f:\textbf{R}\rightarrow\textbf{R}$ be defined as $f\brak{x+y} + f\brak{x-y} = 2f\brak{x}f\brak{y}$, $f\brak{\frac{1}{2}} = -1$. Then, the value of $\sum\limits_{k=1}^{20}\frac{1}{\sin{\brak{k}}\sin{\brak{k + f\brak{k}}}}$ is equal to: \hfill[July 2021]
 \begin{enumerate}
     \item $\cosec^{2}\brak{21}\cos\brak{20}\cos\brak{2}$
     \item $\sec^{2}\brak{1}\sec\brak{21}\cos\brak{20}$
     \item $\cosec^{2}\brak{1}\cosec\brak{21}\sin\brak{20}$
     \item $\sec^{2}\brak{21}\sin\brak{20}\sin\brak{2}$\\
 \end{enumerate}
\item Let $\mathbb{C}$ be the set of all complex numbers. Let $S_1 = \left\{z\in\mathbb{C}:\abs{z-2} \leq 1\right\}$ and $S_2 = \left\{z\in\mathbb{C}:z\brak{1+i} + \overline{z}\brak{1-i} \geq 4\right\}$. Then the maximum value of $\abs{z-\frac{5}{2}}^{2}$ for $z \in S_1 \cap S_2$ is equal to: \hfill[July 2021]
\begin{enumerate}
    \item $\frac{3 + 2\sqrt{2}}{4}$
    \item $\frac{5 + 2\sqrt{2}}{2}$
    \item $\frac{3 + 2\sqrt{2}}{2}$
    \item $\frac{5 + 2\sqrt{2}}{4}$\\
\end{enumerate}
\item  A student appeared in an examination consisting of 8 true-false type questions. The student guesses the answers with equal probability. The smallest value of $n$, so that the probability of guessing at least $'n'$ correct answers is less than $\frac{1}{2}$, is : \hfill[July 2021]
\begin{enumerate}
    \item 5
    \item 6
    \item 3
    \item 4 \\
\end{enumerate}
\item If $\tan\brak{\frac{\pi}{9}}, x, \tan\brak{\frac{7\pi}{18}}$ are in arithmetic progression and $\tan\brak{\frac{\pi}{9}}, y, \tan\brak{\frac{5\pi}{18}}$ are also in arithmetic progression, then $\abs{x - 2y}$ is equal to: \hfill[July 2021]
\begin{enumerate}
    \item 4
    \item 3
    \item 0
    \item 1 \\
\end{enumerate}
 \item Let the mean and variance of the frequency distribution
 \begin{table}[h!]
 \centering
	 \begin{tabular}{|c|c|c|c|c|}
\hline
$x$ & $x_1 = 2$ & $x_2 = 6$ & $x_3 = 8$ & $x_4 = 9$ \\
\hline
$f$ & $4$ & $4$ & $\alpha$ & $\beta$ \\
\hline
\end{tabular}
\end{table} \\
be 6 and 6.8 respectively. If $x_3$ is changed from 8 to 7, then the mean for the new data will be: \hfill[July 2021]
 \begin{enumerate}
     \item 4
     \item 5
     \item $\frac{17}{3}$
     \item $\frac{16}{3}$\\
 \end{enumerate}
\item The area of the region bounded by $y - x = 2$ and $x^2 = y$ is equal to: \hfill[July 2021]
\begin{enumerate}
     \item $\frac{16}{3}$
     \item $\frac{2}{3}$
     \item $\frac{9}{2}$
     \item $\frac{4}{3}$\\
 \end{enumerate}
\item Let $y = y\brak{x}$ be the solution of the differential equation $\brak{x - x^3}dy = \brak{y + yx^2 - 3x^4}dx$, $x > 2$. If $y\brak{3} = 3$, then $y\brak{4}$ is equal to: \hfill[July 2021]
\begin{enumerate}
    \item 4
    \item 12
    \item 8
    \item 16\\
\end{enumerate}
\item The value of $\lim\limits_{x \rightarrow 0}\brak{\frac{x}{\sqrt[8]{1-\sin\brak{x}} - \sqrt[8]{1+\sin\brak{x}}}}$ is equal to: \hfill[July 2021]
\begin{enumerate}
    \item 0
    \item 4
    \item -4
    \item -1\\
\end{enumerate}
\item Two sides of a parallelogram are along the lines $4x + 5y = 0$ and $7x+ 2y = 0$. If the equation of one of the diagonals of the parallelogram is $11x + 7y = 9$, then other diagonal passes through the point: \hfill[July 2021]
\begin{enumerate}
    \item \brak{1,2}
    \item \brak{2,2}
    \item \brak{2,1}
    \item \brak{1,3}\\
\end{enumerate}
\item Let $\alpha = \max\limits_{x\in\textbf{R}}\left\{8^{2\sin\brak{3x}} \cdot 4^{4\cos\brak{3x}}\right\}$ and $\beta = \min\limits_{x\in\textbf{R}}\left\{8^{2\sin\brak{3x}} \cdot 4^{4\cos\brak{3x}}\right\}$. If $8x^2 + bx + c = 0$ is a quadratic equation whose roots are $\alpha^{\frac{1}{5}}$ and $\beta^{\frac{1}{5}}$, then the value of $c - b$ is equal to: \hfill[July 2021]
\begin{enumerate}
    \item 42
    \item 47
    \item 43
    \item 50\\
\end{enumerate}
\item Let $f: [0, \infty) \rightarrow \sbrak{0, 3}$ be a function defined by 
\begin{align*}
f\brak{x} =
    \begin{cases}
    \max\left\{\sin\brak{t}:0 \leq t \leq x \right\}, & 0 \leq x \leq \pi \\
    2 + \cos\brak{x}, & x > \pi
    \end{cases}
\end{align*}
Then which of the following is true ? \hfill[July 2021]
\begin{enumerate}
    \item $f$ is continuous everywhere but not differentiable exactly at one point in $\brak{0, \infty}$.
    \item $f$ is differentiable everywhere in $\brak{0, \infty}$.
    \item $f$ is not continuous exactly at two points in $\brak{0, \infty}$.
    \item $f$ is continuous everywhere but not differentiable exactly at two points in $\brak{0, \infty}$.\\
\end{enumerate}
\item Let \textbf{N} be the set of natural numbers and a relation $R$ on \textbf{N} be defined by 
\begin{align*}
 R = \{ \brak{x, y} \in \textbf{N} \times \textbf{N}: x^3 - 3x^2y - xy^2 + 3y^3 = 0 \}.
\end{align*}
Then the relation $R$ is : \hfill[July 2021]
\begin{enumerate}
    \item symmetric but neither reflexive nor transitive
    \item reflexive but neither symmetric nor transitive
    \item reflexive and symmetric, but not transitive
    \item an equivalence relation\\
\end{enumerate}
