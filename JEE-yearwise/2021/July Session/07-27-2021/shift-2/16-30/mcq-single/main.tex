\iffalse
\title{2021}
\author{EE24BTECH11012}
\section{mcq-single}
\fi
%\begin{enumerate}
	\item Which of the following is the negation of the statement "for all M $\geq$ 0, there exists x $\in \vec{S}$ such that $x\geq M$"?\hfill{[July 2021]}
		\begin{enumerate}
			\item there exists M $\geq$ 0 such that x $\leq$ M for all x $\in \vec{S}$
			\item there exists M $\geq$ 0 there exists x $\in \vec{S}$ such that x $\geq$ M
			\item there exists M $\geq$ 0 there exists x $\in \vec{S}$ such that x $\leq$ M
			\item there exists M $\geq$ 0 such that x $\geq$ M for all x $\in \vec{S}$
		\end{enumerate}
	\item Consider a circle C which touches the y-axis at $\myvec{0,6}$ and cuts off an intercept $6\sqrt{5}$ on the x-axis. Then the radius of the circle C is equal to :\hfill{[July 2021]}
		\begin{enumerate}
				\begin{multicols}{4}
				\item $\sqrt{53}$
				\item 9
				\item 8
				\item $\sqrt{82}$
				\end{multicols}
		\end{enumerate}
	\item Let $\vec{a}$, $\vec{b}$ and $\vec{c}$ be three vectors such that $\vec{a} = \vec{b} \times \brak{\vec{b} \times \vec{c}}$. If magnitudes of the vectors $\vec{a}$, $\vec{b}$ and $\vec{c}$ are $\sqrt{2}$, 1 and 2 respectively and the angle between $\vec{b}$ and $\vec{c}$ is $\theta \brak{0 \leq \theta \leq \frac{\pi}{2}}$, then the value of 1 + $\tan{\theta}$ is equal to :\hfill{[July 2021]}
		\begin{enumerate}
				\begin{multicols}{4}
				\item $\sqrt{3}$ + 1
				\item 2
				\item 1
				\item $ \frac{\sqrt{3} + 1}{\sqrt{3}}$
				\end{multicols}
		\end{enumerate}
	\item Let $\vec{A}$ and $\vec{B}$ be two 3 $\times$ 3 real matrices such that $\vec{A^2 - B^2}$ is invertible matrix. If $\vec{A^5} = \vec{B^5}$ and $\vec{A^3B^2} = \vec{A^2B^3}$, then the value of the determinant of the matrix $\vec{A^3 + B^3}$ is equal to :\hfill{[July 2021]}
		\begin{enumerate}
				\begin{multicols}{4}
				\item 2
				\item 4
				\item 1
				\item 0
				\end{multicols}
		\end{enumerate}
	\item Let $ f : \brak{a,b} \to \vec{R}$ be twice differentiable function such that $ f(x) = \int_{a}^{x} g(t) dt$ for a differentiable function g(x). If f(x) = 0 has exactly five distinct roots in \brak{a,b}, then $g(x)g^{\prime}{x} = 0$ has atleast :\hfill{[July 2021]}
		\begin{enumerate}
			\item twelve roots in \brak{a,b}
			\item five roots in \brak{a,b}
			\item seven roots in \brak{a,b}
			\item three roots in \brak{a,b}
		\end{enumerate}
%\end{enumerate}
%\end{document}
