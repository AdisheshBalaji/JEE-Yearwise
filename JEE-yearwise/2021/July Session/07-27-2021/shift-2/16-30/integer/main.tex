\iffalse
\title{2021}
\author{EE24BTECH11012}
\section{integer}
\fi
%\begin{enumerate}
	\item Let $\vec{a} = \vec{i} - \alpha\vec{j} + \beta\vec{k}$, $\vec{b} = 3\vec{i} + \beta\vec{j} - \alpha\vec{k}$ and $\vec{c} = -\alpha\vec{i} - 2\vec{j} + \vec{k}$, where $\alpha$, $\beta$ are integers. If $\vec{a} \cdot \vec{b} = -1$ and $\vec{b} \cdot \vec{c} = 10$, then $\brak{\vec{a} \times \vec{b}} \cdot \vec{c}$ is equal to :\hfill{[July 2021]}
	\item The distance of the point P$\myvec{3,4,4}$ from the point of intersection of the line joining the points Q$\myvec{3,-4,5}$ and R$\myvec{2,-3,1}$ and the plane $2x+y+z=7$, is equal to :\hfill{[July 2021]}
	\item If the real part of the complex number $ z = \frac{3+2i\cos{\theta}}{1-3i\cos{\theta}}$, $\theta \in \brak{0, \frac{\pi}{2}}$ is zero, then the value of $\sin^{2}{3\theta} + \cos^{2}{\theta}$ is equal to :\hfill{[July 2021]}
	\item Let $\vec{E}$ be an ellipse whose axes are parallel to the co-ordinate axes, having its centre at $\myvec{3,-4}$, one focus at $\myvec{4,-4}$ and one vertex at $\myvec{5,-4}$. If $mx - y = 4$, m>0 is a tangent to the ellipse $\vec{E}$, then the value of 5$m^2$ is equal to :\hfill{[July 2021]}
	\item If $\int_{0}^{\pi} \brak{\sin^{3}{x}}e^{-\sin^{2}{x}} dx = \alpha - \frac{\beta}{e} \int_{0}^{1} \sqrt{t}e^{t} dt$, then $\alpha + \beta$ is equal to :\hfill{[July 2021]}
	\item The number of real roots of the equation $ e^{4x} - e^{3x} - 4e^{2x} - e^{x} + 1 = 0$ is equal to :\hfill{[July 2021]}
	\item Let $ y = y(x) $ be the solution of the differential equation $ dy = e^{\alpha x + y} dx $; $\alpha \in \vec{R}$. If $ y\brak{log\brak{2}} = log\brak{2}$ and $y(0) = log\brak{\frac{1}{2}}$, then the value of $\alpha$ is equal to :\hfill{[July 2021]}
	\item Let $n$ be a non-negative integer. Then the number of divisors of the form "4n+1" of the number $\brak{10}^{10}\brak{11}^{11}\brak{13}^{13}$ is equal to :\hfill{[July 2021]}
	\item Let $A = \cbrak{n \in \vec{N} | n^2 \leq n + 10,000}$, $B = \cbrak{3k+1 | k \in \vec{N}}$ and $C = \cbrak{2k | k \in \vec{N}}$, then the sum of all the elements of the set $ A \cap \brak{ B-C}$ is equal to ;\hfill{[July 2021]}
	\item If $A = \myvec{1&1&1\\0&1&1\\0&0&1}$ and $M = A + A^2 + A^3 + \dots + A^{20} $, then the sum of all the elements of the matrix $M$ is equal to :\hfill{[July 2021]}
%\end{enumerate}
%\end{document}
