
\iffalse
\title{Assignment}
\author{EE24BTECH11035}
\section{mcq-single}
\fi

%\begin{enumerate}

\item
If the mean and variance of the following data:  
$ 6, 10, 7, 13, a, 12, b, 12 $ are $ 9 $ and $ \frac{37}{4} $ respectively, then $ (a - b)^2 $ is equal to	\hfill{(July 2021)}
\begin{enumerate}
    \item 24 
    \item 12
    \item 32
    \item 16
\end{enumerate}

\item
The value of  
\begin{equation*}
\lim_{n \to \infty} \frac{1}{n} \sum_{j=1}^{n} \frac{(2j - 1) + 8n}{(2j-1)+4n}
\end{equation*}
is equal to: 	
\hfill{(July 2021)}
\begin{enumerate}
    \item $5 + \log_e \left(\frac{3}{2}\right)$
    \item $2 - \log_e \left(\frac{2}{3}\right)$
    \item $3 + 2 \log_e \left(\frac{2}{3}\right)$
    \item $1 + 2 \log_e \left(\frac{3}{2}\right)$
\end{enumerate}

\item
Let $ \vec{a} = \hat{i} + \hat{j} + 2\hat{k} $ and $ \vec{b} = -\hat{i} + 2\hat{j} + 3\hat{k} $. Then the vector product
\begin{equation*}
    (\vec{a} + \vec{b}) \times \left( \left( \vec{a}\times(\vec{a} - \vec{b}) \times \vec{b} \right) \times \vec{b} \right)
\end{equation*}
is equal to:\hfill{(July 2021)}
\begin{enumerate}
    \item $5(34\hat{i} - 5\hat{j} + 3\hat{k})$
    \item $7(34\hat{i} - 5\hat{j} + 3\hat{k})$
    \item $7(30\hat{i} - 5\hat{j} + 7\hat{k})$
    \item $5(30\hat{i} - 5\hat{j} + 7\hat{k})$
\end{enumerate}

\item
The value of the definite integral
\begin{equation*}
\int_{-\frac{\pi}{4}}^{\frac{\pi}{4}} \frac{dx}{(1 + e^{\cos^3 x})(\sin^4 x + \cos^4 x)}
\end{equation*}
is equal to:\hfill{(July 2021)}
\begin{enumerate}
    \item $-\frac{\pi}{2}$
    \item $\frac{\pi}{2\sqrt{2}}$
    \item $-\frac{\pi}{4}$
    \item $\frac{\pi}{\sqrt{2}}$
\end{enumerate}

\item
Let $ C $ be the set of all complex numbers. Let  
$
S_1 = \{z \in \mathbb{C} \mid |z - 3 - 2i| = 8 \}, \quad
S_2 = \{z \in \mathbb{C} \mid \text{Re}(z) \geq 5\}, \quad
S_3 = \{z \in \mathbb{C} \mid |z - \bar{z}| \geq 8 \}$
Then the number of elements in $ S_1 \cap S_2 \cap S_3 $ is equal to:\hfill{(July 2021)}
\begin{enumerate}
    \item 1
    \item 0
    \item 2
    \item Infinite
\end{enumerate}

\item
If the area of the bounded region
$
R = \{ (x, y) : \max\{0, \log_2 x\} \leq y \leq 2^x, \frac{1}{2} \leq x \leq 2\}
$
is $ \alpha(\log_2 2)^{-1} + \beta(\log_2 2) + \gamma $, then the value of
$
(\alpha + \beta - 2\gamma)^2
$
is equal to:\hfill{(July 2021)}
\begin{enumerate}
    \item 8
    \item 2
    \item 4
    \item 1
\end{enumerate}

\item A ray of light through $ (2, 1) $ is reflected at a point $ P $ on the $ y $-axis and then passes through the point $ (5, 3) $. If this reflected ray is the directrix of an ellipse with eccentricity $ \frac{1}{3} $ and the distance of the nearer focus from this directrix is $ \frac{8}{\sqrt{53}} $, then the equation of the other directrix can be:\hfill{(July 2021)}
\begin{enumerate}
    \item $11x + 7y + 8 = 0$ or $11x + 7y - 15 = 0$
    \item $11x - 7y - 8 = 0$ or $11x + 7y + 15 = 0$
    \item $2x - 7y + 29 = 0$ or $2x - 7y - 7 = 0$
    \item $2x-7y-39=0$ or $2x-7y-7=0$
\end{enumerate}

\item
If the coefficients of $ x^7 $ in $\left( x^2 + \frac{1}{bx^2} \right)^{11}$ and $ x^{-7} $ in  $\left( x - \frac{1}{bx^2} \right)^{11}, \quad b \neq 0$
are equal, then the value of $ b $ is equal to:\hfill{(July 2021)}
\begin{enumerate}
    \item 2
    \item -1
    \item 1
    \item -2
\end{enumerate}
\item The compound statement $(P\vee Q)\wedge(\sim P)\implies Q$ is equivalent to:\hfill{(July 2021)}
\begin{enumerate}
    \item $P\vee Q$
    \item $P \wedge \sim Q$
    \item $\sim (P \implies Q)$
    \item $\sim (P\implies Q)\Leftrightarrow P \wedge \sim Q$
\end{enumerate}
 
\item If $\sin\theta+\cos\theta=\frac{1}{2}$, then $16(\sin2\theta+\cos4\theta+\sin6\theta)$ is equal to:\hfill{(July 2021)}
\begin{enumerate}
    \item $23$
    \item $-27$
    \item $-23$
    \item $27$
\end{enumerate}
  \item $Let \begin{pmatrix}
      1 & 2\\
      -1 & 4
  \end{pmatrix}$
      If $ A^{-1} = \alpha I + \beta A, \quad \alpha, \beta \in \mathbb{R} $. If $ I $ is a $ 2 \times 2 $ identity matrix, then $ 4(\alpha - \beta) $ is equal to:\hfill{(July 2021)}
    \begin{enumerate}
        \item $ 5 $
        \item $ \frac{8}{3} $
        \item $ 2 $
        \item $ 4 $
    \end{enumerate}
\item Let $ f: \left(-\frac{\pi}{4}, \frac{4}{4}\right) \to \mathbb{R} $ be defined as
\begin{equation*}
       f(x) =
    \begin{cases}
        (1 + |\sin x|) \cdot \frac{3}{\pi} & , \quad -\frac{\pi}{4} < x < 0 \\
        b & , \quad x = 0 \\
        e^{\cot(4/x \cdot 2x)} & , \quad 0 < x < \frac{\pi}{4}
    \end{cases}
  \end{equation*}
    If $ f $ is continuous at $ x = 0 $, then the value of $ 6a + b^2 $ is equal to:\hfill{(July 2021)}
    \begin{enumerate}
        \item $ 1 - e $
        \item $ 2e - 1 $
        \item $ 1 + e $
        \item $ 4e $
    \end{enumerate}
 \item Let $y = y(x)$ be the solution of the differential equation 
    \begin{equation*}
    \log_e\left(\frac{dy}{dx}\right) = 3x + 4y, \quad y(0) = 0.
    \end{equation*}
    If $ y\left(-\frac{2}{3} \log_2 2\right) = \alpha \log_2 2$, then the value of $ \alpha $ is equal to:\hfill{(July 2021)}
    \begin{enumerate}
        \item $ -\frac{1}{4} $
        \item $ \frac{1}{4} $
        \item $ 2 $
        \item $ -\frac{1}{2} $
    \end{enumerate}
    \item Let the plane passing through the point $(-1, 0, -2)$ and perpendicular to each of the planes $2x + y - z = 2$ and $x - y - z = 3$ be $ax + by + cz + 8 = 0$. Then the value of $ a + b + c $ is equal to:\hfill{(July 2021)}
    \begin{enumerate}
        \item $ 3 $
        \item $ 8 $
        \item $ 5 $
        \item $ 4 $
    \end{enumerate}

    \item Two tangents are drawn from the point $P(-1, 1)$ to the circle $x^2 + y^2 - 2x - 6y + 6 = 0$. If these tangents touch the circle at points $A$ and $B$, and if $D$ is a point on the circle such that the lengths of the segments $AB$ and $AD$ are equal, then the area of the triangle $ABD$ is equal to:\hfill{(July 2021)}
    \begin{enumerate}
        \item $ 2 $
        \item $ 3\sqrt{2} + 2 $
        \item $ 4 $
	\item $ 3(\sqrt{2} - 1) $
    \end{enumerate}

  
%\end{enumerate}
%\end{document}

