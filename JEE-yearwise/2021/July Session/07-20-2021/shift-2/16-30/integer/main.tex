\iffalse
\title{2021}
\author{EE24BTECH11012}
\section{integer}
\fi
%\begin{enumerate}
	\item Let $ A = \myvec{2 & -1 & 1 \\ -1 & 2 & -1 \\ 1 & -1 & 2}$ then $det\brak{3Adj\brak{2A^{-1}}}$ is equal to :\hfill{[July 2021]}
	\item If $\myvec{\alpha , \beta}$ is a point on $y^2=6x$, that is closest to $\myvec{3,\frac{3}{2}}$ then find 2 $ \brak{\alpha+\beta} $\hfill{[July 2021]}

	\item Let a function  $g : \sbrak{0,4} \rightarrow \vec{R}$ be defined as
		$$ g(x) = \myvec{ max\brak{t^3-6t^2+9t-3}, & 0 \leq x \leq 3 \\
		                  4 - x, & 3 < x \leq 4 } $$
			then the number of points in the interval \brak{0,4} where g(x) is NOT differentiable is :\hfill{[July 2021]}
		\item The number of solutions of the equation $$\log_{x+1}{\brak{2x^2+7x+5}} + \log_{2x+5}{\brak{x+1}^2} - 4 = 0$$, x $\geq$ 0, is :\hfill{[July 2021]}

	\item Let a curve $ y = y(x)$ be givem by the solutio of the differential equation $$\cos{\brak{\frac{1}{2}\cos^{-1}{e^{-x}}}} dx = \sqrt{e^{2x} - 1} dy $$If it intersects y-axis at $y=-1$ and the intersection point of the curve with the x-axis is $\myvec{\alpha , 0}$, then $e^{\alpha}$ is equal to :\hfill{[July 2021]}
	\item For p $\geq$ 0, a vector $\vec{v_2} = 2\vec{i} + \brak{p+1}\vec{j}$ is obtained by rotating the vector $\vec{v_1} = \sqrt{3}p\vec{i} + \vec{j}$ by an angle $\theta$ about the origin in counter clockwise direction. If $\tan{\theta} = \frac{\alpha\sqrt{3} - 2}{4\sqrt{3} + 3}$, then the value of $\alpha$ is equal to : \hfill{[July 2021]}
	\item Consider a triangle with vertices $\vec{A} \myvec{-2,3}, \vec{B} \myvec{1,9}, \vec{C} \myvec{3,8}$. If a line $\vec{L}$ passing through the circumcentre of the triangle ABC, bisects line BC, and intersects y-axis at point $\myvec{0,\frac{\alpha}{2}}$ then the value of real number $\alpha$ is :\hfill{[July 2021]}
	\item For k $\in \vec{N}$, let $$ \frac{1}{\alpha(\alpha +1)(\alpha +2)\dots(\alpha +20)} = \sum_{k=0}^{20} \frac{A_k}{\alpha + k} $$ where $\alpha>0$.Then the value of 100 $\brak{\frac{A_{14} + A_{15}}{A_{13}}}^2 $ is :\hfill{[July 2021]}
	\item Let $\cbrak{a_{n}}_{n=1}^{\infty}$ be a sequence such that $a_1 = 1$, $a_2 = 1$ and $a_{n+2} = 2a_{n+1} + a_{n}$ for all $n \geq 1$. Then the value of $47 \sum_{n=1}^{\infty} \frac{a_{n}}{2^{3n}}$ is equal to :\hfill{[July 2021]}
	\item If $\lim_{x \to 0} \frac{\alpha xe^{x} - \beta \log\brak{1+x} + \gamma x^2e^{-x}}{x \sin^{2}{x}} = 10 $, $\alpha$, $\beta$, $\gamma \in \vec{R}$, then the value of $\alpha + \beta + \gamma$ is : \hfill{[July 2021]}
%\end{enumerate}
%\end{document}
