\iffalse
  \title{Assignment}
  \author{EE24BTECH11038}
  \section{integer}
\fi  
\item Consider a triangle having vertices $\vec{A}\brak{-2,3},\,\vec{B}\brak{1,9},\,\vec{C}\brak{3,8}$. if a line L passing through the circumcentre of triangle ABC, bisects the line BC, and intersects the Y-axis at $\brak{0,\frac{\alpha}{2}}$, then the value of real number $\alpha$ is \hfill{July 2021}
\item Let $\cbrak{a_n}_{n=1}^{\infty}$ be a sequence such that $a_1=1,a_2=1$ and  $a_{n+2}=2a_{n+1}+a_{n}$ for all $n\geq 1$. Then the value of $47\sum_{n=1}^{\infty} \frac{a_n}{2^{3n}}$ is equal to $\cdots$ \hfill{July 2021}
\item The number of solutions of the equation 
\begin{align*}
    \log_{x+1}\left(2x^2+7x+5\right)+\log_{2x+5}\left({x+1}\right)^2-4=0
\end{align*} 
where $x>0$ is \hfill{July 2021}
\item If $\lim_{x\to 0} \frac{\alpha xe^x-\beta\log_{e}^{1+x}+\gamma x^2e^{-x}}{x\sin^2{x}}=10$ ,$\alpha,\,\beta,\,\gamma \in \mathbf{R}$, then the value of $\alpha+\beta+\gamma$ is \hfill{July 2021}
\item For $p>0$, a vector $\vec{v}_2=2\hat{i}+\brak{p+1}\hat{j}$ is obtained by rotating the vector $\vec{v}_1=\sqrt{3}p\hat{i}+\hat{j}$ by an angle $\theta$ about the origin in a counter clock wise direction if $\tan{\theta}=\frac{\alpha \sqrt{3}-2}{4\sqrt{3}+3}$, then the value of $\alpha$ is \hfill{July 2021}
\item Let A=$\cbrak{a_{ij}}$ be a 3 x 3 matrix, where 
\begin{align*}
    a_{ij}=
    \begin{cases}
        \brak{-1}^{j-i} \,\,if \,i<j, \\
        2 \,\, if \,\, i=j, \\
        \brak{-1}^{i+j} \,if\, i>j, 
    \end{cases}
\end{align*}
Then $det\brak{3Adj\brak{2A^{-1}}}$ is equal to  \hfill{July 2021}
\item Let a curve $y=y\brak{x}$ be given by solution of the differential equation 
\begin{align*}
    \cos{\brak{\frac{1}{2}\cos^{-1}{\brak{e^x}}}}\, dx=\sqrt{e^{2x}-1}\,dy
\end{align*}
if it intersects y-axis at y=-1, and the intersection point of the curve with x-axis is $\brak{\alpha,0}$, then $e^{\alpha}$ is equal to \hfill{July 2021}
\item Let a function g:$\sbrak{0,4}\rightarrow \mathbf{R}$ be defined as 
\begin{align*}
    g\brak{x}=
    \begin{cases}
     \max\limits_{\substack{0 \leq t \leq x}} \cbrak{t^3-6t^2+9t-3}, & 0\leq x \leq 3\\
     4-x, & 3 < x\leq 4
    \end{cases}
\end{align*}
then the number of points in the interval $\brak{0,4}$ where $g\brak{x}$ is NOT differentiabe, is  \hfill{July 2021}
\item For k$\in$N, let
\begin{align*}
    \frac{1}{\alpha \brak{\alpha+1} \brak{\alpha+2}\cdots\brak{\alpha+20}}=\sum_{k=0}^{20} \frac{A_k}{\alpha+k}
\end{align*}
where $\alpha>0$ then the value of $100\brak{\frac{A_{14}+A_{15}}{A_{13}}}^2$ is equal to \hfill{July 2021}
\item If the point on the curve $y^2=6x$, nearest to the point $\brak{3,\frac{3}{2}}$ is $\brak{\alpha,\beta}$, then the value of $2\brak{\alpha+\beta}$ is \hfill{July 2021}
   
