\iffalse
\title{2021}
\author{AI24BTECH11006}
\section{integer}
\fi
\item Let P be a plane containing the line $
\frac{\sbrak{x - 1}}{3} = \frac{\sbrak{y + 6}}{4} = \frac{\sbrak{z + 5}}{2} $ and parallel to the line $
		\frac{\sbrak{x - 3}}{4} = \frac{\sbrak{y - 2}}{-3} = \frac{\sbrak{z + 5}}{7} $. If the point $\brak{1,-1,\alpha}$ lies on the plane P, then the value of $\abs{5\alpha}$ is equal to \dots
\hfill{\sbrak{March-2021}}
	\item $\sum_{r=1}^{10} r!\brak{ r^3 + 6r^2 + 2r + 5 } = \alpha \brak{11!}$ . Then the value of $\alpha$ is equal to \dots
\hfill{\sbrak{March-2021}}
	\item The term independent of x in the expansion of $
\sbrak{ \frac{x + 1}{x^{2/3} - x^{1/3} + 1} - \frac{x - 1}{x - x^{1/2}} }^{10} , x \neq 1$ ,is equal to\dots
\hfill{\sbrak{March-2021}}
\item  Let $ \binom{n}{r} $ denote the binomial coefficient of $ x^r $ in the expansion of $ \brak{1+x}^n $. If $    \sum_{k=0}^{10} \sbrak{2^2 + 3k} \binom{n}{k} = \alpha \cdot 3^{10} + \beta \cdot 2^{10}$then $\alpha + \beta$ is equal to\dots
\hfill{\sbrak{March-2021}}
\item  Let P $\brak{x}$ be a real polynomial of degree 3 which vanishes at x =- 3. Let P$\brak{x}$ have local minima at x = 1, local maxima at x = -1 and $\int_{-1}^{1} P(x) \, dx = 18$,then the sum of all the coefficients of the polynomial P $\brak{x}$ is equal to\dots
\hfill{\sbrak{March-2021}}
\item   Let the mirror image of the point $\brak{1, 3, a}$ with respect to the plane r. $\brak{2i - j + k} - b = 0$ be $\brak{-3, 5, 2}$. Then, the value of$ \abs{a + b}$ is equal to\dots
\hfill{\sbrak{March-2021}}
\item If $f\brak{x}$ and $g\brak{x}$ are two polynomials such that the polynomial $P \brak{x} = f \brak{x^3} + x g \brak{x^3}$ is divisible by $^2 + x + 1$, then $P \brak{1}$ is equal to\dots
\hfill{\sbrak{March-2021}}
\item  Let I be an identity matrix of order $2 \times 2$  and $
P = \begin{bmatrix} 
2 & -1 \\ 
5 & -3 
\end{bmatrix}
$ . Then the value of $n\in N$ for which $P^n=5I-8P$ is equal to\dots
\hfill{\sbrak{March-2021}}
\item Let  $f : \mathbb{R} \to \mathbb{R}$ satisfy the equation 
$
f\brak{x + y} = f\brak{x} \cdot f\brak{y}$for all $x, y \in \mathbb{R} $and $ f\brak{x} \neq 0 $for any$ x \in \mathbb{R}
$. If the function f is differentiable at x = 0 and $f\prime\brak{0}=3$,then $
\lim_{h \to 0} \frac{1}{h}  \sbrak{f\brak{h} - 1 }$ is equal to \dots
\hfill{\sbrak{March-2021}}
\item Let $y = y \brak{x}$ be the solution of the differential equation $x \, dy - y \, dx = \sqrt{x^2 - y^2} \, dx, x \geq 1
$ with $y\brak{1}=0$. If the area bounded by the line $x = 1, x = e^\pi, y = 0$ and $y = y\brak{x}$ is $ \alpha e^{2\pi} + \beta $ then the value of $10 \brak{\alpha + \beta}$ is equal to \dots
\hfill{\sbrak{March-2021}}

