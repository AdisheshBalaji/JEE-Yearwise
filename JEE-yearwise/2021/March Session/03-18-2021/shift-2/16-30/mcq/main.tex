\iffalse
\title{2021}
\author{AI24BTECH11006}
\section{mcq-single}
\fi
\item If P and Q are two statements, then which of the following compound statements is a tautology?
\hfill{\sbrak{March-2021}}

	\begin{enumerate}
    
\item  $\brak{\brak{P \Rightarrow Q} \land \neg Q} \Rightarrow P$ 
\item   $\brak{\brak{P \Rightarrow Q} \land \neg Q} \Rightarrow \neg P $
\item $ \brak{P \Rightarrow Q} \land \neg Q$ 
\item  $\brak{\brak{P \Rightarrow Q} \land \neg Q} \Rightarrow Q $


\end{enumerate}
\item  Consider a hyperbola H: $x^2 -2y^2=4$. Let the tangent at a point P $\brak{4,\sqrt{6}}$ meet the x-axis at Q and latus rectum at R $\brak{x_1,y_1} , x_1 $\textgreater 0. If F is a focus of H which is nearer to the point P, then the area of $\triangle{QFR}$ is equal to:
\hfill{\sbrak{March-2021}}
	\begin{enumerate}
\item $\sqrt{6} - 1$
\item  $4\sqrt{6} - 1$
\item  $4\sqrt{6} $
\item $\frac{7}{\sqrt{6}} - 2 $


\end{enumerate}
\item Let $f : \mathbb{R} \to \mathbb{R}$ be a function defined as 
\begin{equation}
	f\brak{x} = 
\begin{cases} 
\frac{\sin\brak{\brak{a+1}x} + \sin\brak{2x}}{2x}, & $if $  x \textless 0 \\ 
b, & $if $ x = 0 \\ 
\frac{\sqrt{x + bx^{3} - \sqrt{x}}}{bx^{5/2}}, & $if $ x \textgreater 0 
\end{cases}
\end{equation}
. If f is continuous at $ x = 0$, then the value of a + b is equal to
\hfill{\sbrak{March-2021}}
		\begin{enumerate}
    \item $-2$
    \item $\frac{-2}{5}$
    \item $\frac{-3}{2}$
    \item $-3$
\end{enumerate}
\item Let y=y$\brak{x}$ be the solution of the differential equation $\frac{dy}{dx}=\brak{y+1}\sbrak{\brak{y+1}e^{x^2/2}-x} ,$0\textless x\textless 2.1,with y$\brak{2}=0$. Then the value of $\frac{dy}{dx}$ at x=1 is equal to:
\hfill{\sbrak{March-2021}}
	\begin{enumerate}
\item $\frac{e^{5/2}}{\brak{1 + e^{2}}^{2}} $
\item $ \frac{5 e^{1/2}}{\brak{e^{2} + 1}^{2}} $
\item $ -\frac{2 e^{2}}{\brak{1 + e^{2}}^{2}} $
\item $ -\frac{e^{3/2}}{\brak{e^{2} + 1}^{2}} $
\end{enumerate}
\item Let a tangent be drawn to the ellipse $\brak{x^2/27}+y^2=1$ at $\brak{3\sqrt{3} \cos \theta, \sin \theta} \quad $where $ \theta \in \brak{0, \frac{\pi}{2}}$.  Then the value of $\theta$ such that the sum of intercepts on axes made by a tangent is minimum is equal to:
\hfill{\sbrak{March-2021}}
	\begin{enumerate}
    \item $\frac{\pi}{8}$
    \item $\frac{\pi}{6}$
    \item $\frac{\pi}{3}$
    \item $\frac{\pi}{4}$
\end{enumerate}
