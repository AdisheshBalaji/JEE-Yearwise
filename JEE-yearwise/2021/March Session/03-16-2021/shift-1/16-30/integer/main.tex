\iffalse
\title{2021}
\author{EE24BTECH11008}
\section{integer}
\fi
%\begin{enumerate}
       \item Consider an arithmetic series and a geometric series having four initial terms from the set $\sbrak{11,8,21,16,26,32,4}.$ If the last terms of these series are the maximum possible four digit numbers, then the number of common terms in these two series is equal to $\dots$
	       \hfill{\brak{2021-Mar}}\\
    \item Let $f :\brak{0,2} \to \textbf{R}$ be defined as $$f\brak{x}=\log_2\brak{1+\tan\brak{\frac{\pi x}{4}}}.$$ Then, $\lim_{n \to \infty}\frac{2}{n}\brak{f\brak{\frac{1}{n}}+f\brak{\frac{2}{n}}+ \dots +f\brak{1}}$ is equal to $\dots$
	   \hfill{\brak{2021-Mar}} \\
    \item Let $ABCD$ be a square of side of unit length. Let a circle $C_{1}$ centered at $A$ with unit radius is drawn. Another circle $C_{2}$ which touches $C_{1}$ and the lines $AD$ and $AB$ are tangent to it, is also drawn. Let a tangent line from the point $C$ to the circle $C_{2}$ meet the side $AB$ at $E$. If the length of $EB$ is $\alpha + \sqrt{3}\beta ,$ where $\alpha,\beta$ are integers, then $\alpha +\beta$ is equal to $\dots$
	    \hfill{\brak{2021-Mar}}\\

    \item If $\lim_{x \to 0}\frac{ae^x-b\cos x+ce^{-x}}{x\sin x}=2,$ then $a+b+c$ is equal to $\dots$
	    \hfill{\brak{2021-Mar}} \\
    \item The total number of $3 X 3$ matrices $A$ having entries from the set \brak{0,1,2,3} such that the sum of all the diagonal entries of $AA^T$ is $9,$ is equal to $\dots$
	    \hfill{\brak{2021-Mar}}\\
    \item Let $$\text{P} = \begin{bmatrix}
-30 & 20  & 56 \\
90  & 140 & 112 \\
120 & 60  & 14
\end{bmatrix} \text{ and A} =
\begin{bmatrix}
2  & 7  & \omega^2 \\
-1 & -\omega & 1 \\
0  & -\omega & -\omega + 1
\end{bmatrix}$$ where $\omega =\frac{-1+i\sqrt{3}}{2},$ and $\textbf{I}_3$ be the identity matrix of order $3.$ If the determinant of the matrix $\brak{P^{-1}AP-\textbf{I}_3}^2$ is $\alpha\omega^2,$ then the value of $\alpha$ is equal to $\dots$
		\hfill{\brak{2021-Mar}}\\
\item If the normal to the curve $y\brak{x}=\int_0^x\brak{2t^2-15t+10t}dt$ at a point $\brak{a,b}$ is parallel to the line $x+3y=-5, a\textgreater 1,$ then the value of $\abs{a+6b}$ is equal to $\dots$
	\hfill{\brak{2021-Mar}}\\
\item Let the curve $y=y\brak{x}$ be the solution of the differential equation, $\frac{dy}{dx}=2\brak{x+1}.$ If the numerical value of area bounded by the curve $y=y\brak{x}$ and $x-$ axis is $\frac{4\sqrt{8}}{3},$ then the value of $y\brak{1}$ is equal to $\dots$
	\hfill{\brak{2021-Mar}}\\
\item Let $f : \textbf{R} \to \textbf{R}$ be a continous function such that $f\brak{x}+f\brak{x+1}=2,$ for all $x \in \textbf{R}.$ If $\textbf{I}_1=\int_0^xf\brak{x}dx$ and $\textbf{I}_2=\int_{-1}^3f\brak{x}dx,$ then the value of $\textbf{I}_1 +2\textbf{I}_2$ is equal to $\dots\dots$ \hfill{\brak{2021-Mar}}\\
\item Let $z$ and $w$ be two complex numbers such that $w=z\overline{z}-2z+2, \abs{\frac{z+i}{z-3i}}=1$ and $Re\brak{w}$ has the minimum value. Then the minimum value of $n \in \textbf{N}$ for which $w^n$ is real, is equal to $\dots$
	\hfill{\brak{2021-Mar}}
%\end{enumerate}

